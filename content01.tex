\section{Représenter une information}
\subsection{Du sens à la mesure}
\begin{exercice}
  \begin{exercicelet}{Qu'est-ce que l'information}
    \begin{questions}
    \item Proposez différents symbolismes utilisés pour noter un
      nombre. Donnez l'exemple de leur notation avec le nombre 9. Donnez des
      inconvénients de votre méthode.
    \item Travaillez en paire (ou triplettes). Proposez une méthode pour
      transmettre d'une personne à une autre le résultat d'un lancer de dé
      (lancer caché par la première personne, la deuxième doit pouvoir énoncer
      le résultat). Votre méthode fonctionne-t-elle si le dé comporte 20
      faces ? Et si le dé est à six faces mais étiqueté par des couleurs ?
    \end{questions}
  \end{exercicelet}
\end{exercice}
\begin{frame}{Qu'est-ce qu'une information ?}
  \begin{block}{Information}
    Une information est une donnée que l'on peut interpréter pour se
    construire une représentation du monde sur laquelle on peut agir.
  \end{block}
  \begin{itemize}
  \item[\dialoginformation] Claude Shannon a été l'un des premiers à
    définir l'information comme une quantité mesurable.
  \item[\dialogwarning] L'information de Shannon n'est pas associé au
    sens ou à la cognition.
  \item Il s'est intéressé à quantifier des sources aléatoires
  \item L'information diminue l'incertitude sur une source aléatoire
  \item On mesure donc la quantité d'information relative à un événement
  \item Par exemple, si on a six possibilités pour un dé, l'information
    permet de savoir quelle face; ou au moins d'éliminer des
    possibilités
  \end{itemize}
\end{frame}
\begin{frame}{La vision cognitive de l'information}
  \begin{itemize}
  \item[\dialogerror] On ne sait pas mesurer le \emph{sens} des choses.
  \item[\dialoginformation] Une même information peut avoir plusieurs représentations très distinctes.
  \item[\dialogsystem] Le nombre quatorze: 14 ou XIV ou IIIIIIIIIIIIII
  \item[\dialoginformation] Une même donnée peut être interprétée de plusieurs façons → informations très distinctes.
  \item[\dialogsystem] Inversement: XIV est un mot ou un nombre
  \end{itemize}
  \centerline{
    \begin{tikzpicture}
      \tikzstyle{every node}=[fill=solarizedRebase02,draw=solarizedRebase0,rounded corners]
      \tikzstyle{data}=[color=solarizedRed,every node]
      \tikzstyle{xlabel}=[color=solarizedMagenta,fill opacity=30,text opacity=1,every node]
      \tikzstyle{snaky}=[decorate,normally,decoration={snake,pre length=5mm}]
      \tikzstyle{normally}=[very thick]
      \tikzstyle{cloudy}=[color=solarizedGreen,every node]
      \node [cloudy] (a) at (0,2) {Concept};
      \node [cloudy] (b) at (6,2) {Concept}
      edge [<-,snaky] node[xlabel,below] {fonction} (a);
      \node [data] (c) at (0,0) {Donnée}
      edge  [<-,normally] node[xlabel] {représentation} (a) (a);
      \node [data] (d) at (6,0) {Donnée}
      edge  [<-,normally] node[xlabel,above] {algorithme}(c)
      edge  [->,normally] node[xlabel] {interprétation} (b);
    \end{tikzpicture}
  }
\end{frame}
\begin{frame}{L'information digitale}
  \begin{itemize}
  \item Les systèmes d'informations (et les ordinateurs en particulier)
    ne sont pas équipés pour traiter n'importe quel type de données.
  \item Toutes les informations sont représentées sous forme de nombres
    pour être traitées par les ordinateurs.
  \item Le monde réel est \emph{analogique}, la représentation des
    ordinateurs est \emph{numérique} (ou \emph{digitale}).
  \item Nous considérerons que nous avons toujours affaire à des
    problèmes représentables par des nombres.
  \end{itemize}
\end{frame}

\begin{exercice}
  \begin{exercicelet}{Digital ou analogique?}
    \begin{questions}
    \item Est-ce que les données suivantes sont digitales ou analogiques :
      \begin{itemize}
      \item Le fait d'avoir un rendez-vous à une certaine heure un certain jour
      \item La pression de l'air
      \item Le résultat (stable) d'un dé
      \item Votre nom de famille
      \item Votre nombre de frères et sœurs
      \item Votre taille
      \item La couleur de vos yeux
      \end{itemize}
    \end{questions}
  \end{exercicelet}
\end{exercice}
\subsection{Mesurer l'information}
\begin{frame}[label=mesure]{L'information mesurée}
  \begin{block}{bit}
    Le bit est la quantité d'information qui permet de choisir
    complètement entre deux issues distinctes d'un événement.

    Le mot de \emph{bit} est l'abréviation de \emph{binary digit}.
  \end{block}
  \begin{block}{Mesure de l'information}
    Pour exprimer $k$ choix possibles distincts, il faut
    $\left\lceil\log_2(k)\right\rceil$ bits distincts. ($\left\lceil
      x\right\rceil$ est l'arrondi par dessus).

    $k$ bits d'information permettent de distinguer $2^k$ choix.
  \end{block}
  \begin{example}[Jeu du fakir]
    Je peux deviner n'importe quel nombre entre 0 et 100 par 7 questions
    à réponse oui ou non ($\left\lceil\log_2(100)\right\rceil=7)$.
    \begin{presentationonly}\hfill\hyperlink{fakira}{\beamergotobutton{Tester}}\end{presentationonly}
  \end{example}
\end{frame}
\begin{frame}{Binaire et décimal: unités}
  \begin{itemize}
  \item Un groupe de 8 bits est désigné par le terme \emph{octet}.
  \item Abréviations: bit=b, octet=o ou B (anglais). À éviter !
  \item Multiples: kilo, mega, giga, tera (voir mémento).
  \end{itemize}
  \begin{alertblock}{Un gros kilo ou un petit ?}
    \begin{itemize}
    \item[\dialogwarning] kilo-octet souvent $1\,024=2^{10}$ octets et non $10^3=1\,000$.
    \item[\dialogsystem] Utilisez le contexte !
    \item[\dialoginformation] Parfois (toujours dans ce cours), préfixe ki ou Mi pour $2^{10}$ et $2^{20}$.
    \end{itemize}
  \end{alertblock}
  \begin{alertblock}{Octet ou byte ?}
    En anglais, octet=\emph{byte}.
    Ne pas confondre un Mb, un MB, un Mib et un MiB.
  \end{alertblock}
\end{frame}
\begin{frame}{Memento: préfixes et unités}
  \begin{center}
    \emph{L'échelle décimale}\par
    \begin{tabular}{|l|>{\rule{0ex}{2.5ex}$}c<{$}|>{\ttfamily}c|l|>{$}c<{$}|>{\ttfamily}c|}\hline
      Préfixe & \mbox{Valeur} & Ab. & Préfixe & \mbox{Valeur} & Ab.\\\hline
      kilo&10^3&k&milli&10^{-3}&m\\\hline
      mega&10^6&M&micro&10^{-6}&\textmu\\\hline
      giga&10^9&G&nano&10^{-9}&n\\\hline
      tera&10^{12}&T&pico&10^{-12}&p\\\hline
      peta&10^{15}&P&exa&10^{-15}&E\\\hline
    \end{tabular}\par\medskip
    \emph{L'échelle binaire}\par
    \begin{tabular}{|l|>{\rule{0ex}{2.5ex}$}c<{$}|>{$}r<{$}|>{\ttfamily}c@{~ou~}>{\ttfamily}c|}\hline
      Préfixe & \multicolumn{2}{c|}{Valeur} & \multicolumn{2}{c|}{Ab.}\\\hline
      kilo&2^{10}&1024&K&ki\\\hline
      mega&2^{20}&1048576&\multicolumn{2}{c|}{\ttfamily Mi}\\\hline
      giga&2^{30}&1073741824&\multicolumn{2}{c|}{\ttfamily Gi}\\\hline
    \end{tabular}
  \end{center}
  Seule exception beaucoup utilisée: kilo-octets souvent 1024 octets.
  Faux pour kilo-bits (toujours
  1000 bits).
\end{frame}
\begin{frame}{Quelques ordres de grandeur}
  \begin{block}{Quantité}
    \begin{itemize}
    \item $10^3$ bits: carte à bande magnétique
    \item $10^6$ bits: un fax d'une page
    \item $10^9$ bits: Capacité d'un CD ou du génome humain
    \item $10^{12}$ bits: Un disque dur moyen en 2008
    \item $10^{15}$ bits: 1/10\ieme\ taille des serveurs de Google
    \item $10^{18}$ bits: Tout ce qui est imprimé dans le monde.
    \end{itemize}
  \end{block}
  \begin{block}{Débit}
    \begin{itemize}
    \item $1$ b/s: vieille sonde spatiale (9 b/s), morse (40 b/s)
    \item $10^3$ b/s: 2G (9,6 kb/s), modems (56 kb/s)
    \item $10^6$ b/s: ADSL (20 Mb/s)
    \item $10^9$ b/s: Réseau local Gigabit (1 Gb/s), USB (0,48 Gb/s), Infiniband (60 Gb/s)
    \item $10^{12}$ b/s: Trafic total USA cumulé sur internet (12 Tb/s)
    \item $10^{15}$ b/s: Trafic total international sur internet (0.5 Pb/s)
      % note : le trafic sur 2008 par seconde dépasse le total de 1993 (année)
    \end{itemize}
  \end{block}
\end{frame}
\begin{exercice}
  \begin{exercicelet}{Conversions}
    \begin{questions}
    \item Convertissez $24\times 10^8$ bits en $\mathrm{Go}$.
      \begin{correction} $24\times 10^8/8=3\times 10^8=0,3 \mathrm{Go}$
      \end{correction}
    \item Convertissez $2^{16}$ octets en $Mib$. Donnez une approximation en
      $Mb$. Quel est l'ordre de grandeur de l'approximation faite ?
      \begin{correction} $8\times2^{16}=2^{19}=0,5 \mathrm{Mib}$, soit
        environ 0,5 Mb, à 5\% près.
      \end{correction}
    \item Un élément d'ordinateur est capable d'émettre 1024 bits en 0,5
      nanosecondes. Quel est le débit (quantité d'information divisée par le
      temps) de cet élément en bits par secondes ? Quelle est la bonne unité
      pour ce débit ?
      \begin{correction} 1024 bits en 0,5
        nanosecondes=$1024/(0,5\times10^{-9})$=$2048\times10^{9})$
        bits/secondes. L'unité appropriée est sans doute le Tb/s (et pas le
        Tib/s, car si on a un 2048, on a pas une pure puissance de 2, et la
        division du résultat par $2^{40}$ n'est pas un entier du tout...).
      \end{correction}
    \end{questions}
  \end{exercicelet}
  % \begin{exercicelet}{Programmation}
  %   \begin{itemize}
  %   \item[\ddialoghome] Écrivez un programme qui permet de transformer un nombre
  %     d'octets en To/Go/Mo/Ko/o d'un côté et Tio/Gio/Mio/Kio/o d'un
  %     autre. Vous pourrez utiliser l'opérateur \verb|>>| pour diviser par
  %     1024 (\verb|x>>10|) au lieu de la classique division. Testez la
  %     différence entre les deux échelles (binaire, décimal) pour le nombre
  %     d'octets suivants: 900, 1000, 1024, 1000000, 20480000, 1000000000,
  %     1073741824.
  %     \begin{correction}
  %       Exercice à rendre.
  %     \end{correction}
  %   \end{itemize}
  % \end{exercicelet}
\end{exercice}
\subsection{De l'analogique au digital}
\begin{frame}{L'information quantifiée}
  L'information n'est pas toujours disponible dans la nature sous forme
  digitale. Il est donc nécessaire, pour la faire traiter par un ordinateur,
  de la digitaliser.
  
  La digitalisation se fait presque toujours de la même façon:
  \begin{itemize}
  \item Filtrage perceptuel physique
  \item Découpage (volumique) (pour les phénomènes multidimensionnels)
  \item Échantillonnage (pour les phénomènes temporels)
  \item Quantification (réduction à un nombre d'états finis)
  \item Filtrage perceptuel numérique
  \end{itemize}

  Nous reverrons un peu mieux ces notions ultérieurement. Les deux premières
  étapes forment la \emph{discrétisation} (spatiale ou temporelle), et la
  troisième la \emph{quantification}.
\end{frame}
\begin{frame}{L'information discrétisée}{Découpage spatial ou pixellisation}
  \begin{tikzpicture}[draw=solarizedRebase0]
    \pgfdeclareimage[height=1cm]{lena}{img/01/lena.png}
    \pgfdeclareimage[height=51.2mm]{biglena}{img/01/lena.png}
    \pgfdeclareimage[height=51.2mm]{biglena4}{img/01/lena4.png}
    \pgfdeclareimage[height=51.2mm]{biglena16}{img/01/lena16.png}
    \pgfdeclareimage[height=51.2mm]{biglena64}{img/01/lena64.png}
    \draw[transparent,use as bounding box] (-3,0) rectangle (5.12,5.12);
    \only<2-| handout:1-> { \draw [dashed] (-3,1)--(0,0); \draw [dashed]
      (-2,1)--(5.12,0); \draw [dashed] (-2,2)--(5.12,5.12); \draw [dashed]
      (-3,2)--(0,5.12); } \only<1,2| handout:1-> { \node at (-2.5,1.5)
      {\pgfbox[center,center]{\pgfuseimage{lena}}}; } \only<3-| handout:0> {
      \node[fill opacity=.5] at (-2.5,1.5)
      {\pgfbox[center,center]{\pgfuseimage{lena}}}; } \only<2|
    handout:0>{\node[fill opacity=.5] at (2.56,2.56)
      {\pgfbox[center,center]{\pgfuseimage{biglena}}};} \only<3-4| handout:0>{
      \node at (2.56,2.56) {\pgfbox[center,center]{\pgfuseimage{biglena}}}; }
    \only<5-6| handout:1>{ \node at (2.56,2.56)
      {\pgfbox[center,center]{\pgfuseimage{biglena16}}}; } \only<4-5|
    handout:1>{ \draw[step=.16] (0,0) grid (5.12,5.12); } \only<7-8|
    handout:2>{ \node at (2.56,2.56)
      {\pgfbox[center,center]{\pgfuseimage{biglena64}}}; } \only<7|
    handout:2>{ \draw[step=.64] (0,0) grid (5.12,5.12); } \only<9-10|
    handout:3>{ \node at (2.56,2.56)
      {\pgfbox[center,center]{\pgfuseimage{biglena4}}}; } \only<9| handout:3>{
      \draw[step=.04] (0,0) grid (5.12,5.12); }
  \end{tikzpicture}
  \begin{itemize}
  \item[\dialogwarning] La résolution d'échantillonnage influe sur la fidélité
    de l'image
  \item[\dialogerror] L'information est perdue: on ne peut pas retrouver la
    précision.
  \item[\dialognetwork] Source de l'image: Image
    Lena\qquad{\tiny\url{http://www.cs.cmu.edu/~chuck/lennapg/}}
  \end{itemize}
\end{frame}
\begin{frame}{L'information discrétisée}{Échantillonnage temporel}
  Rappel: une fréquence, c'est une quantité d'événements par unité de temps
  ($1 Hz=1 s^{-1}$).
  \begin{tikzpicture}[draw=solarizedRebase0]
    \draw[transparent,use as bounding box] (-.5,-2) rectangle (9,2);
    \draw[->](0,0)--(8,0) node[right] {$T$}; \draw(0,-2)--(0,2); \foreach \x
    in {-2,-1.5,...,2} { \draw[ultra thin] (0,\x)--++(-.2,0) node[left]
      {\tiny\x}; } \foreach \x in {0,.5,...,8} { \draw[ultra thin]
      (\x,.2)--++(0,-.2) node[below] {\tiny\x}; }
    
    \only<1-3,5| handout:1,3>{ \draw[smooth,samples=100,domain=0.0:8]
      plot(\x,{sin(\x*180)+0.3*sin(4*\x*180)+0.2*sin(5*\x*180+450)}); }
    \only<4,7-| handout:2,4->{
      \draw[opacity=.5,smooth,dashed,samples=100,domain=0.0:8]
      plot(\x,{sin(\x*180)+0.3*sin(4*\x*180)+0.2*sin(5*\x*180+450)}); }
    \only<5-6| handout:3>{ \draw[ultra
      thin,draw=solarizedRebase3,smooth,samples=100,domain=0.0:8]
      plot(\x,{sin(\x*180)+0.2*sin(5*\x*180+450)}); \node[above] at (4,-2) {Il
        peut y avoir plusieurs reconstructions possibles}; } \only<2-3|
    handout:1>{ \foreach \x in {0,.5,...,8}
      \draw[draw=solarizedRebase3,dashed]
      (\x,0)--(\x,{sin(\x*180)+0.3*sin(4*\x*180)+0.2*sin(5*\x*180+450)});
      \node[above] at (4,-2) {Intervalles réguliers: \emph{fréquence
          d'échantillonnage}}; } \only<3-6| handout:1-3>{ \foreach \x in
      {0,.5,...,8} \fill[fill=solarizedRed]
      (\x,{sin(\x*180)+0.3*sin(4*\x*180)+0.2*sin(5*\x*180+450)}) circle (.05);
    } \only<7| handout:4>{ \foreach \x in {0,.1,...,8}
      \fill[fill=solarizedRed]
      (\x,{sin(\x*180)+0.3*sin(4*\x*180)+0.2*sin(5*\x*180+450)}) circle (.05);
      \node[above] at (4,-2) {On garantit l'unicité en augmentant la fréquence
        d'échantillonnage}; } \only<8| handout:5>{ \foreach \x in {0,.1,...,8}
      { \fill[fill=solarizedRed]
        (\x,{sin(\x*180)+0.3*sin(4*\x*180)+0.2*sin(5*\x*180+450)}) circle
        (.03); \fill[fill=solarizedBlue]
        (\x,{0.2*floor(5*(sin(\x*180)+0.3*sin(4*\x*180)+0.2*sin(5*\x*180+450)))})
        circle (.05); } \node[above] at (4,-2) {Après vient la
        quantification!}; }
  \end{tikzpicture}
  \begin{itemize}
  \item[\ddialogsystem] Pour pouvoir reconstruire exactement un signal
    périodique qui peut être décomposé avec une fréquence maximale, sa
    fréquence d'échantillonnage $f_e$ doit vérifier \emph{(Théorème
      d'échantillonnage de Nyquist-Shannon)}:
    $$f_e\geqslant 2 f_{\mathrm{Max}}$$
  \end{itemize}
\end{frame}
\begin{frame}{L'information quantifiée}
  \begin{block}{Quantification}
    Cette opération réduit un signal à des \emph{quanta} (singulier
    \emph{quantum}) en nombre limité. Le nombre de \emph{quanta} possibles
    s'appelle la \emph{valence}.

    La reconstruction exacte du signal n'est plus possible, mais reste souvent
    proche de l'original.
  \end{block}
  \begin{example}{Signal électrique}
    Une tension électrique compris entre 0 (large) et 10 V (strict) peut ainsi
    être réduit à 10 quanta : 0~V, 1~V,\dots,9~V.
  \end{example}
  Le nombre de bits nécessaires pour coder un état du signal peut être exprimé
  par $$k=\left\lceil\log_2 V\right\rceil.$$

  \dialoginformation Beaucoup plus sur la quantification des images plus tard.

\end{frame}
\begin{exercice}
  \begin{exercicelet}{Signal électrique}
    \begin{questions}
    \item Un signal électrique qui va de 0 à 2,559 V est quantifié sur un
      quantum de 0,01 V. Quel est le nombre de quanta ? Quelle quantité
      d'information est transportée par un quantum ?
      \begin{xcorrection}
        8 bits par échantillon. La valence est de 256.
      \end{xcorrection}
    \item Ce signal est périodique, et se décompose avec des fréquences
      maximales qui vont jusqu'à 10 kHz. Quelle est le débit d'information
      nécessaire pour reconstituer ce signal à l'identique ?
      \begin{xcorrection}
        Chaque échantillon prend un octet. $f_e$ doit être plus grand que
        $2\times 10^3$ échantillons/s. Donc le débit 20 ko/s (binaires).
      \end{xcorrection}
    \item Quelle est la taille de l'information nécessaire pour enregistrer ce
      signal pendant une heure ?
      \begin{xcorrection}
        3600 s*20 ko/s=72 Mo (décimaux)
      \end{xcorrection}
    \end{questions}
  \end{exercicelet}
  \begin{exercicelet}{CD audio}
    \begin{questions}
    \item Un CD audio contient de la musique échantillonnée en stéréo sur 16
      bits par piste à 44100 Hz (nombre d'échantillons par seconde). Il dure
      environ 80 minutes. Calculez (de tête) l'ordre de grandeur de la
      quantité d'information écrite dans un CD audio.
      \begin{xcorrection}
        Environ 800 Mo. En vrai: 807,5 Mio ou 846 Mo.  88200 octets par
        seconde par piste, soit 167400 octets/s, soit ~10 Mo/minute.
      \end{xcorrection}
    \end{questions}
  \end{exercicelet}
\end{exercice}


% Local Variables:
% TeX-master: "archi01"
% TeX-PDF-mode: t
% fill-column: 78
% coding: utf-8-unix
% mode-require-final-newline: t
% mode: latex
% mode: flyspell
% ispell-local-dictionary: "francais"
% End:
