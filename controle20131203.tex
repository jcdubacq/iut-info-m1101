\documentclass[a4paper]{iutvexam}
% compile-latex options:--jobname controle20131118
% compile-latex options:--jobname correction20131118
\usepackage{textcomp}
\usepackage{dirtree}
\usepackage{listings}
\usepackage{caption}
\newcommand{\DTd}[1]{\textbf{#1/}}
\newcommand{\DTfcomment}[1]{\DTcomment{\emph{#1}}}

\title{Contrôle 2}
\date{18/11/2013}
\begin{document}
\conditions{ Vous disposez de 2 heures pour faire ce contrôle. Aucun
  document autorisé. Toute tentative de communication avec un voisin ou
  l'extérieur peut être sanctionnée. Toutes les réponses doivent être
  faites sur l'énoncé. La taille de la réponse attendue dépend de la
  taille allouée pour répondre. }
\begin{questions}
  \titledquestion{Scripts}
  \begin{parts}
    \part[\half] Quelle variable désigne \texttt{ijkl} dans la commande \texttt{./toto.sh abcd efgh ijkl mnop}?
    \begin{solutionordottedlines}[.25in]%
      \verb|$3|%$
    \end{solutionordottedlines}
    \part[\half] Que met-on comme première ligne d'un script shell ?
    \begin{solutionordottedlines}[.25in]%
      \verb|#!/bin/sh| (ou bash).
    \end{solutionordottedlines}
    \part[\half] Que contient la variable \verb|PATH|?
    \begin{solutionordottedlines}[.5in]%
      Le chemin du répertoire courant
    \end{solutionordottedlines}
    \part[\half] Quelle est la commande shell pour faire afficher le
    résultat de l'opération $2187\times 1309$?
    \begin{solutionordottedlines}[.25in]%
      \verb|echo $((2187*1309))|%$
    \end{solutionordottedlines}
    \part[\half] Quelle est la commande shell pour faire afficher le
    résultat de l'opération $2187\times a$, sachant que $a$ est une
    variable qui contient un entier?
    \begin{solutionordottedlines}[.25in]%
      \verb|echo $((2187*a))|%$
    \end{solutionordottedlines}
    \part[\half] Que fait la commande \verb|cd $PWD|.
    \begin{solutionordottedlines}[.25in]%
      \verb|mv ~/toto/toto.sh ~/bin;chmod 755 ~/bin/toto.sh| ou 
      \verb|PATH="~/toto:$PATH";chmod 755 ~/toto/toto.sh|
    \end{solutionordottedlines}
    \part[\half] Comment afficher en une seule ligne de test le texte \texttt{ok} si \texttt{toto} est un répertoire ou si \texttt{titi} est exécutable?
    \begin{solutionordottedlines}[.75in]%
      \verb|if [ -d toto ]&& ! [ -x titi ]; then echo "ok"; fi|
    \end{solutionordottedlines}
    \part[1] Faites un programme qui affiche chaque seconde un mot de la
    phrase « bonjour amis terriens qui venez en paix » (donc qui dure
    sept secondes). Il attend une seconde \emph{avant} d'afficher, pas
    \emph{après}.
    \begin{solutionordottedlines}[.75in]%
      \verb|for i in bonjour amis terriens qui venez en paix; do echo $i; done| %$
    \end{solutionordottedlines}
    \part[2] Faites un script shell qui affiche, s'il existe, le
    contenu du fichier \texttt{README.txt} du répertoire \textbf{parent}
    du répertoire courant, et affiche «~Aucune aide n'est disponible,
    contactez "root"~» sinon. Attention, il y a une apostrophe dans
    cette phrase et deux guillemets.
    \begin{solutionordottedlines}[1in]%
      \begin{lstlisting}[language=sh,caption=Programme]
        if [ -f ../README.txt ]; then # 0,5 pour le if correct avec else et fi
        cat ../README.txt # 0,5 pour le chemin correct
        else
        echo "Aucune aide n'est disponible, contactez" '"root"'
        fi # 0,5 pour les guillemets et 0,5 pour l'apostrophe
      \end{lstlisting}
    \end{solutionordottedlines}
  \end{parts}
  \newpage
  \titledquestion{Script de statistiques}
  \begin{parts}
    \part[3]
    Écrivez un script qui donne pour \textbf{tous} les fichiers *.txt
    présents dans le répertoire courant le nombre de lignes de ce
    fichier sous la forme: «~Fichier toto.txt:\textbackslash n72\textbackslash n~». (le
    \texttt{\textbackslash n} est un retour à la ligne). Les
    statistiques doivent être mises dans un fichiers \texttt{stats.out}
    également. Les messages d'erreurs (par exemple, si \texttt{titi.txt}
    est en fait un répertoire, la commande qui compte les lignes émet un
    message d'erreur) doivent être mis dans \texttt{stats.err}. À la fin
    du programme, affichez toutes les lignes de \texttt{stats.err} qui
    comportent le mot \texttt{dossier}.

    Ce programme acceptera en \textbf{paramètre optionnel} un nom de répertoire,
    dans lequel il faudra aller s'il est donné (et s'il n'est pas donné,
    on travaillera dans le répertoire courant).
    \begin{solutionordottedlines}[3in]%
      \half pour penser à effacer; \half pour le 2>, \half pour le >,
      \half pour le append; \half pour le paramètre optionnel; \half
      pour le reste (boucle for, wc, etc.).
      \begin{lstlisting}[language=sh,caption=Programme]
        if [ -n "$1" ]; then
        cd "$1"
        fi
        rm -f stats.out stats.err
        for i in *.txt; do
        echo Fichier $i: >> stats.out
        wc -l $i >> stats 2>> stats.err
        done
        cat stats.err|grep dossier # ou grep dossier stats.err
      \end{lstlisting} %
    \end{solutionordottedlines}
    \bonuspart[\half] Modifiez votre programme pour avoir
    «~Fichier toto.txt: 72\textbackslash n~» dans \texttt{stats.out}.
    \begin{solutionordottedlines}[.25in]%
      Il suffit de faire un echo -n au lieu de echo.
    \end{solutionordottedlines}
    \part[\half] Rajoutez deux lignes à la fin de votre script pour
    afficher aussi les trois premières lignes de résultat
    \texttt{stats.out}, la chaîne \texttt{...} et les trois dernières
    lignes du résultat.

    \begin{solutionordottedlines}[.5in]%
      head -3 stats.out ; echo ... ; tail -3 stats.out
    \end{solutionordottedlines}
  \end{parts}
  \titledquestion{Données}
  \begin{parts}
    \part[\half] Le nombre hexadécimal \texttt{0x12345678} a été mis à
    l'adresse \texttt{0xC0800008} en mode \texttt{little-endian}. Quel
    est l'autre mode possible ? Donnez le contenu et l'adresse de chacun
    des quatre octets qui contient une partie du nombre.
    \begin{solutionordottedlines}[.5in]%
      0,25 seulement s'ils mettent les mauvaises adresses, 0,5 si bon
      (0x78, 0x56, 0x34, 0x12).
    \end{solutionordottedlines}
    \part[1] Quelle est la taille habituelle d'un \texttt{int} en C ? Quelle est sa valeur maximale ?
    \begin{solutionordottedlines}[.5in]%
      \half pour 4 octets, \half pour $2^{31}-1$.
    \end{solutionordottedlines}
    \part[1] À quoi sert la compression de données ? Est-ce qu'on peut
    récupérer la qualité d'origine ? Citez au moins un algorithme de
    compression de données.
    \begin{solutionordottedlines}[.75in]%
      \half pour une explication qui tienne, \half pour un nom d'algorithme.
    \end{solutionordottedlines}
    \part[1] Un bus fonctionne à 0,5 GHz avec une largeur de 64
    bits. Chaque échange nécessite un cycle pour transmettre une
    adresse, et 1 cycle pour transmettre une valeur. Quel est le débit
    maximal de données qui circule sur ce bus ?
    \begin{solutionordottedlines}[.5in]%
      $0,25\times10^9\times 8 = 2 Go/s$.
    \end{solutionordottedlines}
    \part[1] À quoi sert un codec ? Dans quel contexte en trouve-t-on ?
    Quel est le lien avec un format ?
    \begin{solutionordottedlines}[1in]%
      \half pour une explication qui tienne, \half pour dire que
      certains formats imposent un codec mais que d'autres ne l'imposent
      pas ; un codec vidéo accepte en général plusieurs formats.
    \end{solutionordottedlines}
  \end{parts}
  \titledquestion{Images et couleurs}
  \begin{parts}
    \part[1\half] Pour les trois couleurs suivantes: rouge, cyan et noir
    définissez (en pourcentage du maximum) la quantité de chaque
    composante RVB et CMJN qui est nécessaire.
    \begin{center}
      \begin{tabular}{l|ccc|cccc}
        \textbf{Couleur} & R & V & B & C & M & J & N\\\hline
        \rule{0cm}{.25in} Rouge & \hspace{1cm} & \hspace{1cm} & \hspace{1cm} & \hspace{1cm} & \hspace{1cm} & \hspace{1cm} & \hspace{1cm}\\\hline
        \rule{0cm}{.25in} Cyan &&&&&&&\\\hline
        \rule{0cm}{.25in} Noir &&&&&&&\\
      \end{tabular}
    \end{center}
    \part[1] Une photo de 5 pouces sur 4 a été scannée en CMJN sur 12
    bits par composante à 300 points par pouce (dans chaque
    dimension). Quelle est la quantité d'information extraite de cette
    image ? Formule, application numérique (en Mo).
    \begin{solutionordottedlines}[.75in]%
      10800000 octets, soit 10,8 Mo.
    \end{solutionordottedlines}
  \end{parts}
  \titledquestion{Processeur}
  \begin{parts}
    \part[1] Un programme de 1000 instructions machine comporte 200
    instructions flottantes et 800 instructions ordinaires. Les
    instructions flottantes mettent en moyenne 8 cycles pour s'exécuter,
    et les instructions ordinaires mettent 3 cycles. Calculer le nombre
    de cycles nécessaire à l'exécution de ce programme et son IPC moyen
    (dont vous rappellerez la définition).
    \begin{solutionordottedlines}[.75in]%
      IPC moyen=0,25 (4000 cycles).
    \end{solutionordottedlines}
    \part[2] Expliquez ce qu'est un langage compilé et un langage
    interprété, en mettant en lumière les différences entre les deux.
    \begin{solutionordottedlines}[2in]%
    \end{solutionordottedlines}
  \end{parts}
\end{questions}
\end{document}
\begin{parts}
  \part[1] Quelle est la différence entre une variable du shell et une variable d'environnement ?
  \begin{solutionordottedlines}[.5in]%
    Une variable du shell n'est pas transmise aux sous-processus du
    shell. Compter \half au moins pour un simple « il faut mettre
    \texttt{export} devant ».
  \end{solutionordottedlines}
  \part[1] Est-ce que deux processus peuvent être exécutés réellement
  en même temps sur un processeur simple ? Expliquez
  \begin{solutionordottedlines}[1in]%
    Le processeur étant partagé, pas vraiment; il y a exécution concurrente avec priorités des 2 processus; Alternance rapide; Illusion de simultanéité
  \end{solutionordottedlines}
  \part[1] Citez au moins quatre systèmes d'exploitation différents
  \begin{solutionordottedlines}[.25in]%
    Linux, Windows, OS X, Android, ...
  \end{solutionordottedlines}
  \part[1] Comment le shell trouve-t-il où est une commande ?
  \begin{solutionordottedlines}[1in]%
    Il utilise la variable \texttt{PATH} qui est une liste de
    répertoires séparés par des deux-points. Il cherche dans ces
    répertoires si un fichier du nom de la commande existe et s'arrête
    au premier trouvé.
  \end{solutionordottedlines}
\end{parts}

\newpage
\titledquestion{Arborescence}

\framebox{\begin{minipage}{.5\linewidth}%
    \dirtree{%
      .1 \DTd{}.  .2 \DTd{bin}.  .3 (\dots)\DTfcomment{(1)}.  .2
      \DTd{home}.  .3 \DTd{jvaljean}\DTfcomment{Votre répertoire
        personnel}.  .4 \DTd{Documents}\DTfcomment{Vous êtes ici}.  .5
      \DTd{Ancien}. .6 \DTd{Paris}.  .7 greve.jpg.  .5
      condamnation.pdf. .5 document1.txt.  .5 doc2.txt.  .5
      facture.txt.  .5 mission.pdf.  .5 mission.jpg.  .4 \DTd{Coffre}.
      .3 \DTd{gavroche}.  .4 avis.txt.  .2 (\dots)\DTfcomment{(2)}.  }
  \end{minipage}}\hfill%
\begin{minipage}{.45\linewidth}
  Voilà ci-contre un extrait de l'arborescence de votre système. Il y
  a d'autres utilisateurs : \textit{\texttt{gavroche}} est dans votre
  groupe, \textit{\texttt{thenardier}} et \textit{\texttt{cosette}} ne
  le sont pas.

  Vous devrez faire tout cet exercice sans \textbf{jamais} utiliser la
  commande \texttt{cd}. Dans l'extrait d'arborescence, le répertoire
  courant est indiqué. Il est possible que \textbf{deux} commandes
  soient nécessaires pour certaines manipulations.
\end{minipage}
\begin{parts}
  \bonuspart[\half] Répondez à \textnormal{toutes} les questions avec
  une seule commande pour ce bonus!
  \part[\half] Listez le contenu du répertoire \texttt{Ancien}
  \begin{solutionordottedlines}[.25in]%
    \texttt{ls Ancien}
  \end{solutionordottedlines}
  \part[\half] Créez un répertoire \texttt{Nouveau/Vacances}
  \begin{solutionordottedlines}[.25in]%
    \texttt{mkdir -p Nouveau/Vacances} ou en deux commandes. J'ai
    finalement accepté en une, en donnant 0,75 (la moitié du bonus de
    \emph{une seule commande}) à ceux qui avaient mis le \texttt{-p}
    ou fait en deux commandes.
  \end{solutionordottedlines}
  \part[1] Lisez le contenu du fichier \texttt{avis.txt} chez
  l'utilisateur \textit{\texttt{gavroche}}.
  \begin{solutionordottedlines}[.25in]%
    \texttt{cat \~gavroche/avis.txt} ou autre solution à base de \texttt{..}
  \end{solutionordottedlines}
  \part[1] Déplacez tous les documents du répertoire courant dans \texttt{Coffre} sauf \texttt{condamnation.pdf}
  \begin{solutionordottedlines}[.25in]%
    \texttt{mv *.txt mission.pdf ../Coffre} ou autre solution à base de \texttt{..}
  \end{solutionordottedlines}
  \part[1] Copiez \texttt{condamnation.pdf} et le répertoire \texttt{Paris} dans le répertoire \texttt{Nouveau/Vacances}
  \begin{solutionordottedlines}[.25in]%
    \texttt{cp -r condamnation.txt Ancien/Paris Nouveau/2013} (.5 pour le -r).
  \end{solutionordottedlines}
  \part[1] Sachant que les fichiers étaient lisibles par tout le
  monde, changez les modes de \texttt{Coffre} pour que
  \texttt{\textit{gavroche}} puisse connaître le contenu du
  répertoire, que \texttt{\textit{thenardier}} ne puisse pas, et que
  \texttt{\textit{cosette}} puisse (si vous lui en donnez le chemin)
  lire un des fichiers. Utilisez les notations \textbf{numériques}
  pour cette commande.
  \begin{solutionordottedlines}[.25in]%
    \texttt{chmod 751 \~/Coffre}
  \end{solutionordottedlines}
  \part[\half] Le mode de \texttt{Ancien} est au départ
  \texttt{rwxr-xr-x}. En utilisant les notations \textbf{symboliques}
  et non numériques pour cette commande, faites en sorte que plus
  personne (même vous) ne puisse savoir quoi que ce soit sur le
  répertoire \texttt{Paris} stocké dans \texttt{Ancien}.
  \begin{solutionordottedlines}[.25in]%
    \texttt{chmod a-rwx Ancien/Paris}
  \end{solutionordottedlines}
  \part[\half] Donnez ici deux exemples de répertoires ou fichiers que l'on trouve traditionnellement dans les parties non listées numéro 1 et 2.
  \begin{solutionordottedlines}[.25in]%
    1. ls et 2. usr
  \end{solutionordottedlines}
\end{parts}
\newpage
\titledquestion{Codage}
\begin{parts}
  \part[\half] Quel est le nombre \textbf{minimal} de bits nécessaire
  pour coder l'information suivante: un numéro de page de l'annexe
  d'un livre, qui vaut entre 1400 et 1912.
  \begin{solutionordottedlines}[.25in]
    1912-1400=513 possibilités, donc 10 bits.
  \end{solutionordottedlines}
  \part[\half] Quel est le plus grand entre $48\times 10^6$
  bits et $6$ Mio ? Prouvez-le
  \begin{solutionordottedlines}[.5in]
    $48\times10^6 bits < 6Mio$ car $48\times10^6 bits = 6\times10^6 octets = 6Mo$ or $6Mo < 6Mio$ car \\
    $1Mo=10^6=(1000)^2$ et $1Mio=2^{20}=(2^10)^2=1024^2$ et
    $1000<1024$ (éventuellement moins de détails)
  \end{solutionordottedlines}
  \part[1\half] Convertissez en hexadécimal les 3 nombres suivants:
  48, 226, 2251.
  \begin{solutionordottedlines}[.75in]
    0x30, 0xE2, 0x8CB
  \end{solutionordottedlines}
  \part[1] Convertissez en C2 sur 8 bits le nombre $-61$.
  \begin{solutionordottedlines}[.25in]
    0xC3 ou $1100\,0011$\\
  \end{solutionordottedlines}
  \part[1] Faites l'opération suivante:
  $0b1101\,1011+0b1111\,0101+1$. Donnez le résultat en binaire puis en
  décimal.
  \begin{solutionordottedlines}[.25in]
    $0b1\,1101\,0001=465_{10}$
  \end{solutionordottedlines}
  \part[1] Voici le récapitulatif du tableau de conversion UTF8:
  \begin{center}\renewcommand{\tabcolsep}{1mm}
    \begin{tabular}{|c|c|c|c|}\hline
      Valeurs & Écriture binaire & Codage UTF-8 (binaire) & octets\\\hline
      0x0--0x7F & abc\,defg & 0abc\,defg & 1 \\
      0x80--0x7FF & abc\,defg\,hijk & 110a\,bcde~10fg\,hijk & 2 \\
      0x800--0xFFFF & abcd\,efgh\,ijkl\,mnop & 1110\,abcd~10ef\,ghij~10kl\,mnop & 3 \\
      0x10000--0x1FFFFF& a\,bcde\,fghi\,jklm\,nopq\,rstu & 1111\,0abc~10de\,fghi~10jk\,lmno~10pq\,rstu &	4 \\\hline
    \end{tabular}
  \end{center}
  Donnez la séquence d'octets correspondant au texte «~1€~», dont les
  valeurs unicodes sont 0x49 et 0x20AC.
  \begin{solutionordottedlines}[.25in]
    0x49, puis 0xE2 0x82 0xAC.\\
    $0x49\in [0x0;0x7F]$ $\rightarrow$ codage sur 1 octet. Le code ascii est le code UTF-8 solution:$0x49$\\
    $0x20AC\in [0x800;0xFFFF]$ $\rightarrow$ codage sur 3 octets selon le motif $111\,abcd\,\,10ef\,ghij\,\,10kl\,mnop$ avec $abcd\,efgh\,ijkl\,mnop=0x20AC=0011\,0000\,1010\,1100$ soit: solution: $1110\,0010\,\,1000\,0010\,\,1010\,1100=0xE282AC$
  \end{solutionordottedlines}
  \part[\half] Un signal audio a une fréquence d'échantillonage de
  8000 Hz. Quelle est la fréquence maximale qui peut être reproduite
  fidèlement avant quantification ? Justifiez
  \begin{solutionordottedlines}[.25in]
    D'après le Théorème d'échantillonnage de Nyquist-Shannon on a
    $f_{echan}>2*f_{Max}$ donc or ici
    $f_{Max}<f_{chan}/2=8000Hz/2=4000Hz$
  \end{solutionordottedlines}
  \part[1] Un signal audio (mono-voie) à 8000 Hz utilise 1024 niveaux
  d'intensité par échantillon. Combien de bits sont nécessaires pour
  chaque échantillon ? (justifiez) En déduire la taille totale d'un
  fichier qui code une minute de signal.
  \begin{solutionordottedlines}[.25in]
    $8000\times 10\times 60=4\,800\,000$ bits ou bien 600 ko.
  \end{solutionordottedlines}
\end{parts}
\titledquestion{Flottants courts} On veut modifier la norme IEEE
754. On définit une nouvelle catégorie de nombres flottants sur 12
bits:
\begin{itemize}
\item 1 bit de signe;
\item 4 bits pour noter $E$ avec $E=e+7$;
\item 7 bits pour la partie fractionnaire $M$ de la valeur $v$.
\end{itemize}
On a $x=(-1)^s\times v\times 2^e$. On fait les mêmes
\textbf{exceptions} que dans la norme IEEE 754: pour
$E=\texttt{0b0000}$, on a $x=0$ et pour $E=\texttt{0b1111}$, on a
$x=\pm\infty$. NB: $2^{-6}=0,015\,625, 2^{-12}=0,000\,244\,140\,625$.
\begin{parts}
  \part[1] Codez en \emph{flottant court} le nombre 3,5.
  \begin{solutionordottedlines}[.5in]%
    $3,5=11,1=1,11\times2$ donc $0100 0110 0000$ soit 0x460.
  \end{solutionordottedlines}
  \part[1] Quel est le plus grand nombre non infini que l'on peut
  représenter? Donnez son \emph{codage} sous forme hexadécimale, et sa
  valeur en décimal.
  \begin{solutionordottedlines}[.5in]%
    Codage $011101111111=0x77F$ (1/2), $E=0b1110=14$, $e=7$ et
    $v=0b1,1111111$. Soit $x=0b1111\,1111=255$ (1/2). Notes partielles
    si essentiellement bon.
  \end{solutionordottedlines}
  \part[1] Soit deux nombres $a$ et $b$, flottants courts, qui ont le
  même exposant $e$. Quelle est la plus petite différence possible non
  nulle entre ces deux nombres ?
  \begin{solutionordottedlines}[.5in]%
    La différence de $v$ vaut 0.0000001 donc différence totale
    de $2^e\times2^{-7}$ donc $2^{e-7}$.
  \end{solutionordottedlines}
  \part[1] Transformez en décimal les deux nombres \emph{flottants
    courts} codés par les chaînes de bits suivantes:
  $\texttt{0b0011\,1010\,0000}$ et
  $\texttt{0b0110\,0101\,1010}$.
  \begin{solutionordottedlines}[.5in]%
    1,25 et 54,5
  \end{solutionordottedlines}
\end{parts}

\titledquestion{Programme étrange}%
Considérez le programme suivant:

\begin{lstlisting}[language=C,caption=Programme]
  fonction(int a) {
    int b=0;
    while (a!=0) {
      a=a>>1;
      b=b+(a&1)
    }
    return(b);
  }
\end{lstlisting}

\begin{parts}
  \part[1] Pourquoi ce programme s'arrête-t-il ? Expliquez-bien.
  \begin{solutionordottedlines}[.75in]
    Le décalage à droite n'introduit que des zéros dans le nombre a,
    donc a finit par être nul (division entière par 2).
  \end{solutionordottedlines}
  \part[1] On appelle la fonction avec la valeur 21. Détaillez les
  valeurs successives de \texttt{a} et \texttt{b} jusqu'à l'arrêt de
  la fonction.

  \begin{minipage}{.3\linewidth}
    \begin{solutionordottedlines}[1in]
      21,0 (facultatif)\\10,1
    \end{solutionordottedlines}
  \end{minipage}\hfill
  \begin{minipage}{.3\linewidth}
    \begin{solutionordottedlines}[1in]
      5,1\\2,2
    \end{solutionordottedlines}
  \end{minipage}\hfill
  \begin{minipage}{.3\linewidth}
    \begin{solutionordottedlines}[1in]
      1,2\\0,3
    \end{solutionordottedlines}
  \end{minipage}
  \part[1] Que calcule cette fonction ?
  \begin{solutionordottedlines}[.25in]
    Le nombre de bits à 1 dans le codage binaire de l'entier a.
  \end{solutionordottedlines}
\end{parts}
\end{questions}
\end{document}
