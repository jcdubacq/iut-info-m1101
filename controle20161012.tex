\documentclass[a4paper]{iutvexam}
% compile-latex options:--jobname controle20161012
% compile-latex options:--jobname correction20161012
\usepackage{textcomp}
\usepackage{dirtree}
\usepackage{listings}
\usepackage{caption}
\newcommand{\DTd}[1]{\textbf{#1/}}
\newcommand{\DTfcomment}[1]{\DTcomment{\emph{#1}}}

\title{Contrôle 1}
\date{8/10/2013}
\begin{document}
\conditions{ Vous disposez de 2 heures pour faire ce contrôle. Aucun
  document autorisé. Toute tentative de communication avec un voisin ou
  l'extérieur peut être sanctionnée. Toutes les réponses doivent être
  faites sur l'énoncé. La taille de la réponse attendue dépend de la
  taille allouée pour répondre. }
\begin{questions}
  \titledquestion{Système}

  Lorsqu'une commande vous est demandée, \textbf{donnez un exemple
    d'utilisation} qui répond à la question.
  \begin{parts}
    \part[3] Complétez le tableau:\par
    \def\spacer{\rule{0mm}{5mm}\rule{4cm}{.5pt}}
    \begin{tabular}{p{42mm}l}
      \spacer & Lister les fichiers (pour un répertoire, le contenu du répertoire)\\
      \spacer & Obtenir de l'aide sur une commande\\
      \spacer & Créer un raccourci vers un autre chemin, qui peut être brisé\\
      \spacer & Déplacer ou renommer un fichier\\
      \spacer & Greffer un nouveau système de fichier dans l'arborescence des fichiers\\
      \spacer & Obtenir les méta-données sur un fichier\\
    \end{tabular}

    \part[1] Quelle est l'interprétation du symbole spécial \verb|~|
    dans le shell ? Donnez \textbf{deux} sens (légèrement) différents de
    ce symbole.
    \begin{solutionordottedlines}[.5in]%
      On peut désigner son propre répertoire (symbole seul) ou celui de
      quelqu'un d'autre.
    \end{solutionordottedlines}
    \part[1\half] Donnez au moins trois métadonnées bien différentes qui sont stockés dans l'i-nœud rattaché à un fichier.
    \begin{solutionordottedlines}[.5in]%
      Droits d'accès, propriétaire, dates de dernière ... (modification, lecture), taille en octets et en blocs...
    \end{solutionordottedlines}
    \part[1] Citez au moins deux ressources gérées par le système
    d'exploitation.
    \begin{solutionordottedlines}[.5in]%
      Le système d'exploitation arbitre au moins l'accès au processeur
      et l'accès à la mémoire. Les périphériques aussi, mais pas la ROM
      ou le GRUB. .25 par ressource.
    \end{solutionordottedlines}
    \part[\half] Quelle est la commande qui permet de visualiser
    l'espace libre de chaque point de montage avec leurs
    caractéristiques sous Linux ?
    \begin{solutionordottedlines}[.25in]%
      \texttt{df}
    \end{solutionordottedlines}
    \part[1\half] Un système de fichiers \emph{case-insensitive} est un
    système ou la portion de chemin qui le parcourt ne tient pas compte
    de la différence majuscule/minuscule. Un tel système a été monté au
    point de montage \texttt{/media/divers/WIN}. Regroupez les chemins suivants
    selon qu'ils désignent ou pas un même fichier:
    \begin{enumerate}
    \item \texttt{/media/Divers/win/Usage/Document.txt}
    \item \texttt{/media/divers/WIN/Usage/document.txt} %
    \item \texttt{/media/divers/WIN/Usage/../USAGE/document.txt} %
    \item \texttt{/media/divers/WIN/Usage/.../Usage/document.txt}
    \item \texttt{/media/divers/WIN/Usage/../document.txt}
    \item \texttt{/media/divers/WIN/./USAGE/Document.TXT} %
    \item \texttt{/media/divers/WIN/Usage/.././Usage/document.txt}
    \end{enumerate}
    \begin{solutionordottedlines}[.25in]%
      2,3 et 6 sont les mêmes fichiers ; les autres sont tous différents.
    \end{solutionordottedlines}
  \end{parts}

  \newpage
  \titledquestion{Arborescence}

  \framebox{\begin{minipage}{.5\linewidth}%
      \dirtree{%
        .1 \DTd{}.  .2 \DTd{bin}.  .3 (\dots)\DTfcomment{(1)}.  .2
        \DTd{home}.  .3 \DTd{jvaljean}\DTfcomment{Votre répertoire
          personnel}.  .4 \DTd{Documents}\DTfcomment{Vous êtes ici}.  .5
        \DTd{Ancien}. .6 \DTd{Paris}.  .7 greve.jpg.  .5
        condamnation.pdf. .5 document1.txt.  .5 doc2.txt.  .5
        facture.txt.  .5 mission.pdf.  .5 mission.jpg.  .4 \DTd{Coffre}.
        .3 \DTd{gavroche}.  .4 avis.txt.  .2 (\dots)\DTfcomment{(2)}.  }
    \end{minipage}}\hfill%
  \begin{minipage}{.45\linewidth}
    Voilà ci-contre un extrait de l'arborescence de votre système. Il y
    a d'autres utilisateurs : \textit{\texttt{gavroche}} est dans votre
    groupe, \textit{\texttt{thenardier}} et \textit{\texttt{cosette}} ne
    le sont pas.

    Vous devrez faire tout cet exercice sans \textbf{jamais} utiliser la
    commande \texttt{cd}. Dans l'extrait d'arborescence, le répertoire
    courant est indiqué. Répondez à \textnormal{toutes} les questions
    avec une seule commande.

  \end{minipage}
  \begin{parts}
    \part[\half] Listez le contenu du répertoire \texttt{Ancien}
    \begin{solutionordottedlines}[.25in]%
      \texttt{ls Ancien}
    \end{solutionordottedlines}
    \part[\half] Créez un répertoire \texttt{Nouveau/Vacances}
    \begin{solutionordottedlines}[.25in]%
      \texttt{mkdir -p Nouveau/Vacances} ou \texttt{mkdir Nouveau Nouveau/Vacances}.
    \end{solutionordottedlines}
    \part[1] Créez un lien sur le fichier \texttt{facture.txt} dans le répertoire courant sous le nom \texttt{TODO}. Est-ce que l'information existera toujours si on exécute \texttt{rm facture.txt} ? Et si on l'édite et qu'on supprime toutes les lignes (et qu'on sauvegarde, bien sûr) ?
    \begin{solutionordottedlines}[.25in]%
      \texttt{ln facture.txt TODO}. Oui, et non.
    \end{solutionordottedlines}
    \part[\half] Lisez le contenu du fichier \texttt{avis.txt} chez
    l'utilisateur \textit{\texttt{gavroche}}.
    \begin{solutionordottedlines}[.25in]%
      \texttt{cat \~gavroche/avis.txt} ou autre solution à base de \texttt{..}
    \end{solutionordottedlines}
    \part[1] Déplacez tous les documents du répertoire courant dans \texttt{Coffre} sauf \texttt{condamnation.pdf}
    \begin{solutionordottedlines}[.25in]%
      \texttt{mv *.txt mission.pdf ../Coffre} ou autre solution à base de \texttt{..}
    \end{solutionordottedlines}
    \part[1] Copiez \texttt{condamnation.pdf} et le répertoire \texttt{Paris} dans le répertoire \texttt{Nouveau/Vacances}
    \begin{solutionordottedlines}[.25in]%
      \texttt{cp -r condamnation.txt Ancien/Paris Nouveau/2013} (.5 pour le -r).
    \end{solutionordottedlines}
    \part[1] Donnez la commande qui permet d'archiver tous vos documents dans une archive située dans \texttt{coffre} (nom au choix).
    \begin{solutionordottedlines}[.25in]%
      \texttt{tar cvf ~/Coffre/archive.tar ~/Documents}
    \end{solutionordottedlines}
  \end{parts}
  \newpage
  \titledquestion{Codage}
  \begin{parts}
    \part[1] Quel est le nombre \textbf{minimal} de bits nécessaire
    pour coder l'information suivante: un numéro de page de l'annexe
    d'un livre, qui vaut entre 1900 et 2412 (bornes comprises).
    \begin{solutionordottedlines}[.25in]
      2412-1900=513 possibilités, donc 10 bits.
    \end{solutionordottedlines}
    \part[\half] Quel est le plus grand entre $24\times 10^6$
    bits et $3$ Mio ? Prouvez-le
    \begin{solutionordottedlines}[.5in]
      $24\times10^6 bits < 3Mio$ car $24\times10^6 bits = 3\times10^6 octets = 6Mo$ or $3Mo < 3Mio$ car \\
      $1Mo=10^6=(1000)^2$ et $1Mio=2^{20}=(2^10)^2=1024^2$ et
      $1000<1024$ (éventuellement moins de détails)
    \end{solutionordottedlines}
    \part[1\half] Convertissez en hexadécimal les 3 nombres suivants:
    96, 212, 2251.
    \begin{solutionordottedlines}[.75in]
      0x60, 0xE2, 0x8CB
    \end{solutionordottedlines}
    \part[1] Convertissez en C2 sur 8 bits le nombre $-61$.
    \begin{solutionordottedlines}[.25in]
      0xC3 ou $1100\,0011$\\
    \end{solutionordottedlines}
    \part[1] Convertissez en nombre décimal le nombre C2 sur 8 bits écrit en hexal 0xF0.
    \begin{solutionordottedlines}[.25in]
      -16\\
    \end{solutionordottedlines}
    \part[1] Faites l'opération suivante:
    $0b1101\,1011+0b1111\,0101+1$. Donnez le résultat en binaire puis en
    décimal.
    \begin{solutionordottedlines}[.25in]
      $0b1\,1101\,0001=465_{10}$
    \end{solutionordottedlines}
    \part[\half] Faites l'opération suivante:
    $0x1028+0x988$. Donnez le résultat en hexadécimal.
    \begin{solutionordottedlines}[.25in]
      $0b1\,1101\,0001=465_{10}$
    \end{solutionordottedlines}
    \part[\half] Un signal audio a une fréquence d'échantillonage de
    8000 Hz. Quelle est la fréquence maximale qui peut être reproduite
    fidèlement avant quantification ? Justifiez
    \begin{solutionordottedlines}[.25in]
      D'après le Théorème d'échantillonnage de Nyquist-Shannon on a
      $f_{echan}>2*f_{Max}$ donc or ici
      $f_{Max}<f_{chan}/2=8000Hz/2=4000Hz$
    \end{solutionordottedlines}
    \part[1] Un signal audio (mono-voie) à 8000 Hz utilise 1024 niveaux
    d'intensité par échantillon. Combien de bits sont nécessaires pour
    chaque échantillon ? (justifiez) En déduire la taille totale d'un
    fichier qui code une minute de signal.
    \begin{solutionordottedlines}[.25in]
      $8000\times 10\times 60=4\,800\,000$ bits ou bien 600 ko.
    \end{solutionordottedlines}
  \end{parts}
  \titledquestion{Flottants courts} On veut modifier la norme IEEE
  754. On définit une nouvelle catégorie de nombres flottants sur 12
  bits:
  \begin{itemize}
  \item 1 bit de signe;
  \item 4 bits pour noter $E$ avec $E=e+7$;
  \item 7 bits pour la partie fractionnaire $M$ de la valeur $v$.
  \end{itemize}
  On a $x=(-1)^s\times v\times 2^e$. On fait les mêmes
  \textbf{exceptions} que dans la norme IEEE 754: pour
  $E=\texttt{0b0000}$, on a $x=0$ et pour $E=\texttt{0b1111}$, on a
  $x=\pm\infty$. NB: $2^{-6}=0,015\,625, 2^{-12}=0,000\,244\,140\,625$.
  \begin{parts}
    \part[1] Codez en \emph{flottant court} le nombre 3,5.
    \begin{solutionordottedlines}[.5in]%
      $3,5=11,1=1,11\times2$ donc $0100 0110 0000$ soit 0x460.
    \end{solutionordottedlines}
    \part[1] Quel est le plus grand nombre non infini que l'on peut
    représenter? Donnez son \emph{codage} sous forme hexadécimale, et sa
    valeur en décimal.
    \begin{solutionordottedlines}[.5in]%
      Codage $011101111111=0x77F$ (1/2), $E=0b1110=14$, $e=7$ et
      $v=0b1,1111111$. Soit $x=0b1111\,1111=255$ (1/2). Notes partielles
      si essentiellement bon.
    \end{solutionordottedlines}
    \part[1] Soit deux nombres $a$ et $b$, flottants courts, qui ont le
    même exposant $e$. Quelle est la plus petite différence possible non
    nulle entre ces deux nombres ?
    \begin{solutionordottedlines}[.5in]%
      La différence de $v$ vaut 0.0000001 donc différence totale
      de $2^e\times2^{-7}$ donc $2^{e-7}$.
    \end{solutionordottedlines}
    \part[1] Transformez en décimal les deux nombres \emph{flottants
      courts} codés par les chaînes de bits suivantes:
    $\texttt{0b0011\,1010\,0000}$ et
    $\texttt{0b0110\,0101\,1010}$.
    \begin{solutionordottedlines}[.5in]%
      1,25 et 54,5
    \end{solutionordottedlines}
  \end{parts}

  \titledquestion{Programme étrange}%
  Considérez le programme suivant:

  \begin{lstlisting}[language=C,caption=Programme]
    fonction(int a) {
      int b=0;
      while (a!=0) {
        a=a>>1;
        b=b+(a&1)
      }
      return(b);
    }
  \end{lstlisting}

  \begin{parts}
    \part[1] Pourquoi ce programme s'arrête-t-il ? Expliquez-bien.
    \begin{solutionordottedlines}[.75in]
      Le décalage à droite n'introduit que des zéros dans le nombre a,
      donc a finit par être nul (division entière par 2).
    \end{solutionordottedlines}
    \part[1] On appelle la fonction avec la valeur 21. Détaillez les
    valeurs successives de \texttt{a} et \texttt{b} jusqu'à l'arrêt de
    la fonction.

    \begin{minipage}{.3\linewidth}
      \begin{solutionordottedlines}[1in]
        21,0 (facultatif)\\10,1
      \end{solutionordottedlines}
    \end{minipage}\hfill
    \begin{minipage}{.3\linewidth}
      \begin{solutionordottedlines}[1in]
        5,1\\2,2
      \end{solutionordottedlines}
    \end{minipage}\hfill
    \begin{minipage}{.3\linewidth}
      \begin{solutionordottedlines}[1in]
        1,2\\0,3
      \end{solutionordottedlines}
    \end{minipage}
    \part[1] Que calcule cette fonction ?
    \begin{solutionordottedlines}[.25in]
      Le nombre de bits à 1 dans le codage binaire de l'entier a.
    \end{solutionordottedlines}
  \end{parts}
\end{questions}
\end{document}
