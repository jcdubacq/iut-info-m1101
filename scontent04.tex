\section{Compléments sur l'arborescence}
\subsection{Droits sur les fichiers}
\begin{frame}{Propriété des fichiers}
  \begin{block}{Identifications des utilisateurs dans un environnement
      multi-utilisateurs}
    \begin{enumerate}
    \item[UID] (\textbf{U}ser \textbf{ID}entifier) numéro unique associé à
      chaque utilisateur lors de la création de son compte.
    \item[GID] (\textbf{G}roup \textbf{ID}entifier) numéro unique d'un groupe
      d'utilisateurs. Chaque utilisateur peut être associé à un ou plusieurs
      groupes.
    \end{enumerate}
  \end{block}
  \begin{alertblock}{Utilité}
    \begin{itemize}
    \item Chaque fichier (ou répertoire) et chaque processus du système est
      associé à un utilisateur: cet utilisateur est le propriétaire du fichier
      (ou répertoire) ou celui qui a lancé le processus.
    \item Être propriétaire d'un fichier ou d'un processus confère des droits
      sur ceux-ci.
    \end{itemize}
  \end{alertblock}
  \begin{block}{Connaitre l'identité du propriétaire d'un processus ou d'un
      fichier}
    \begin{itemize}
    \item Les commandes \lin{top} et \lin{ps} affichent le nom du propriétaire
      des processus.
    \item La commande \lin{ls} avec l'option \lin{-l} affiche le nom et le
      groupe du propriétaire d'un fichier ou d'un répertoire.
    \item Les UID et GID sont enregistrés dans le fichier d'administration
      \lin{/etc/passwd} ou d'autres mécanismes
    \end{itemize}
  \end{block}
\end{frame}

\manpage{ls(ter)}
\begin{frame}{Les droits sur les fichiers et répertoires}
  \begin{block}{3 catégories d'utilisateurs}
    \begin{center}
      \begin{tabular}{c|ccc|ccc|ccc}
        \Huge{-}&\Huge{r}&\Huge{w}&\Huge{x}&\Huge{r}&\Huge{w}&\Huge{x}&\Huge{r}&\Huge{w}&\Huge{x}\\
        &&&&&&&&&\\
        Type de&\multicolumn{3}{c|}{Doits du}&\multicolumn{3}{c|}{Doits du}&\multicolumn{3}{c}{Doits des}\\
        Fichier&\multicolumn{3}{c|}{propriétaire}&\multicolumn{3}{c|}{groupe}&\multicolumn{3}{c}{autres}\\
        &\multicolumn{3}{c|}{(\textbf{U}ser)}&\multicolumn{3}{c|}{(\textbf{G}roup)}&\multicolumn{3}{c}{(\textbf{O}ther)}\\
      \end{tabular}
    \end{center}
  \end{block}
  \begin{columns}
    \begin{column}{3.5cm}
      \begin{block}{Types de fichiers}
        \begin{center}
          \begin{tabular}{ll}
            \hline
            &Types\\
            \hline
            -&Fichier ordinaire\\
            d&Répertoire\\
            l&lien symbolique\\
            \hline
          \end{tabular}
        \end{center}
      \end{block}
    \end{column}
    \begin{column}{8cm}
      \begin{block}{Droits/Permissions}
        \begin{center}
          \begin{tabular}{llll}
            \hline
            &&Fichier&Répertoire\\
            \hline
            \lin{r}&(\textbf{R}ead)&lire&lister le contenu\\
            \lin{w}&(\textbf{W}rite)&écrire et modifier&modifier le contenu\\
            \lin{x}&(e\textbf{X}ecute)&exécution&traverser\\
            \hline
          \end{tabular}
        \end{center}
      \end{block}
    \end{column}
  \end{columns}
  \begin{block}{Types d'utilisateurs}
    \begin{center}
      \begin{tabular}{lll}
        \hline
        && Cible\\
        \hline
        u&(\textbf{U})ser&Propriétaire du fichier/répertoire\\
        g&(\textbf{G})roup&Membre du même groupe que le propriétaire\\
        o&(\textbf{O})ther&Tous les autres\\
        a&(\textbf{A})ll&Tous les utilisateurs (réunion de \lin{'u'}, \lin{'g'} et \lin{'o'}).\\
        \hline
      \end{tabular}
    \end{center}
  \end{block}
\end{frame}

\manpage{chmod} \manpage{chmod(bis)}

\begin{exercice}
  \begin{exercicelet}{Identification et droits}
    \begin{questions}
    \item Au moyen de la commande \lin{id}, affichez votre UID et votre GID?
      Comparez-le avec celui de votre voisin de table. Qu'en concluez-vous?
      Comparez-les avec celui de l'utilisateur \lin{root}. Qu'en
      concluez-vous?
    \item Quels sont vos droits sur le répertoire racine \lin{/}, \lin{root},
      \lin{/tmp}, sur votre répertoire \lin{\~\//}, et celui de votre voisin
      de table \lin{\~\//../login\_voisin}.
    \item Pouvez-vous lire les données contenue dans le répertoire de votre
      voisin. Quelle commande permettrait de le faire? Qui doit lancer la
      commande?
    \item Donnez les commandes octale et alphanumérique de changement de
      droits permettant:
      \begin{itemize}
      \item d'autoriser aux membres de votre groupe et aux "autres" l'accès en
        lecture aux images du répertoire \lin{donnees\_tdtp2/images}.
      \item de donnez les droits d'écriture aux membres de votre groupe sur le
        fichier \lin{donnees\_tdtp2/command\_line.txt}
      \item de vous (le propriétaire) retirer toute possibilité de supprimer
        le fichier \lin{donnees\_tdtp2/0readme}
      \end{itemize}
    \item Imaginez comment donner à votre voisin un accès sous votre
      répertoire personnel à un répertoire dans lequel il aurait les droits
      d'écriture sur un fichier spécifique, que vous ne pourriez vous que lire
      (mais pas modifier). Il ne doit pas pouvoir créer un autre fichier chez
      vous. Comment faites vous pour effacer ce fichier ?
    \end{questions}
  \end{exercicelet}
\end{exercice}
\begin{exercice}
  \begin{exercicelet}{Remise en état}
    Après toutes les modifications pouvant impliquer votre répertoire
    personnel, n'oubliez pas \lin{chmod 711 ~} pour remettre les modes de
    votre répertoire à leur état d'origine.
  \end{exercicelet}
\end{exercice}
\subsection{Arborescence du système Linux}
\begin{frame}{Les principaux répertoires et leur contenu}
  \begin{block}{Une structure plus ou moins normalisée}
    \begin{itemize}
    \item Les fichiers nécessaires au fonctionnement du système sont organisés
      en arborescence,
    \item Cette arborescence est commune à presque toutes les distribution
      linux,
    \item Cette organisation rationalisée facilite l'installation de nouveaux
      programmes qui savent où trouver les fichiers dont ils peuvent avoir
      besoin.
    \end{itemize}
  \end{block}
  \begin{block}{Une organisation qui permet un cloisonnement}
    \begin{itemize}
    \item Les fichiers et les répertoires systèmes sont protégés par des
      restrictions de droits,
    \item De nombreux fichiers ne peuvent être modifiés par un utilisateur
      «~normal~»,
    \item Seul l'utilisateur \lin{root}, ou les utilisateur faisant partie du
      groupe \lin{admin} peuvent avoir la permission de modifier certains
      fichiers.
    \item Il s'agit d'une protection. Pour réaliser une action susceptible
      d'affecter le comportement du système il faut montrer "patte blanche" et
      prendre conscience de ce que l'on fait. Entrer le mot de passe
      \lin{root} doit être un signal d'alerte.
    \end{itemize}
  \end{block}
\end{frame}

\begin{frame}{Les principaux répertoires et leur contenu}
  \begin{center}
    \begin{tabular}{ll}
      \hline
      Répertoire&Contenu\\[3pt]
      \hline
      /		&Répertoire racine: Toutes les données accessibles par le système\\[3pt]
      /bin		&Binaires exécutables des commandes de bases (cd, ls, mkdir, \dots)\\[3pt]
      /dev		&Fichiers spéciaux correspondants aux périphériques\\[3pt]
      /etc		&Fichiers de configuration (profile, passwd,fstab... )\\[3pt]
      /home	&Les répertoires personnels des utilisateurs\\[3pt]
      /lib		&Librairies partagées et modules du noyeau\\[3pt]
      /mnt		&Points de montage des périphériques\\[3pt]
      /root		&Répertoire personnel de l'administrateur\\[3pt]
      /tmp		&Données temporaires\\[3pt]
      /usr		&Ressources accessibles par les utilisateurs\\[3pt]
      /var		&Fichiers de log ou fichiers changeant fréquemment\\[3pt]
      \hline
    \end{tabular}
  \end{center}
  L'essentiel est synthétisé dans
  \url{https://fr.wikipedia.org/wiki/Filesystem_Hierarchy_Standard}
\end{frame}
\begin{exercice}
  \begin{exercicelet}{Hiérarchie du système}
    \begin{itemize}
    \item[\ddialoginformation] Astuce: si la sortie d'une commande est trop
      longue, on peut ajouter \verb/|less/ à la fin de la ligne pour
      l'afficher par morceaux. Ceci vous sera expliqué dans quelques
      séances...
    \end{itemize}
    \begin{questions}
    \item Identifiez le propriétaire, le groupe et les différents droits des
      fichiers contenus dans le répertoire \lin{/bin}? Quels sont vos droits
      sur ces fichiers?
    \item Ces fichiers on le droit \lin{x}. Que pouvez-vous en conclure?
    \item A votre avis, que se passe-t-il en fait lorsque vous saisissez une
      commande telle que \lin{ls} ?
    \end{questions}
  \end{exercicelet}
  \begin{exercicelet}{FHS}
    \begin{questions}
    \item Identifiez, à l'aide de la FHS, la fonction de
      \lin{/usr/include}. Confirmez votre hypothèse en regardant quelques
      fichiers.
    \end{questions}
  \end{exercicelet}
\end{exercice}

\subsection{Interprétation ou Compilation}
\begin{frame}{Langages Compilés Vs Langages Interprétés}
  \begin{block}{Caractéristiques des Langages
      \only<1|handout:1>{Compilés}\only<2|handout:2>{Interprétes}}
    \only<1|handout:1>{%
      \begin{itemize}
      \item L'ensemble du code source est compilé une seule fois avant
        l'exécution en instructions machine (contenues dans un fichier:
        exécutable).
      \item Le compilateur n'est pas nécessaire lors de l'exécution.
      \item Le compilateur est spécifique à la machine.
      \item L'exécutable (code compilé) est spécifique à la machine.
      \end{itemize}
    } \only<2|handout:2>{%
      \begin{itemize}
      \item Les instructions du code source sont converties en instructions
        machine lors de l'exécution du programme
      \item L'interpréteur est nécessaire lors de l'exécution.
      \item L'interpréteur est spécifique à la machine,
      \item L'exécutable (le code source) n'est pas spécifique à la machine.
      \end{itemize}
    }
  \end{block}
  \begin{columns}
    \begin{column}{5.5cm}
      \begin{block}{Inconvenients}
        \only<1|handout:1>{%
          \begin{itemize}
          \item Il faut recompiler pour prendre en compte une modification du
            code.
          \item L'exécutable n'est pas portable sur d'autres machines.
          \end{itemize}
        } \only<2|handout:2>{%
          \begin{itemize}
          \item Moins rapide.
          \item Plusieurs fichiers (et librairies) servent à l'exécution.
          \end{itemize}
        }
      \end{block}
    \end{column}
    \begin{column}{5.5cm}
      \begin{block}{Avantages}
        \only<1|handout:1>{%
          \begin{itemize}
          \item Plus rapide (spécifique à la machine qui exécute les
            instructions).
          \item L'ensemble des instructions sont regroupées dans un seul
            fichier.
          \end{itemize}
        } \only<2|handout:2>{%
          \begin{itemize}
          \item Modifications du code source immédiatement prises en compte
            lors de la réexécution.
          \item Le code est portable sur d'autres machine
          \end{itemize}
        }
      \end{block}
    \end{column}
  \end{columns}
  \begin{block}{Exemples de langages \only<1>{Compilés}\only<2>{Interprétés}}
    \only<1|handout:1>{%
      \begin{itemize}
      \item C, C++, ADA, Pascal, Fortran, Cobol,
      \end{itemize}
    } \only<2|handout:2>{%
      \begin{itemize}
      \item Java, Python, Bash, Lisp, PHP, Prolog, Perl, Javascript
      \end{itemize}
    }
  \end{block}
\end{frame}

\subsection{Exécution des commandes}
\begin{frame}{Lancer un programme/une commande}
  \begin{block}{Cas général}
    \begin{itemize}
    \item Pour exécuter un programme il suffit saisir sur la ligne de commande
      le chemin menant au fichier contenant les instructions,
    \item Si le fichier présente la permission \lin{"X"} pour exécutable, les
      instructions qu'il contient sont exécutées.
    \end{itemize}
  \end{block}
  \begin{columns}
    \begin{column}{.45\textwidth}
      \begin{block}{Script {bash} exécutable}
        \begin{itemize}
        \item Un script \lin{bash} est un fichier texte contenant des
          instructions \lin{bash}
        \item La première ligne contient le chemin menant à l'exécutable de
          l'interpréter précédé des caractère \lin{\#!} (par exemple \lin{\#!
            /bin/bash}),
        \item La seconde ligne est souvent vide,
        \item Les lignes suivantes comportent des instructions.
        \end{itemize}
      \end{block}
    \end{column}
    \begin{column}{.45\textwidth}
      
      \fileform[.8\textwidth]{test\_bash.sh}{%
        \#!/bin/bash\\ \\instruction 1 ;\\instruction 2 ;\\...\\instruction N
        ; }
    \end{column}
  \end{columns}
\end{frame}

\begin{exercice}
  \begin{exercicelet}{Lancer un programme/une commande}
    \begin{questions}
    \item Après avoir créé un répertoire \lin{bin} dans votre répertoire
      personnel, définissez créez dans ce répertoire un script nommé
      \lin{listintro.sh}. Ce script comporte une unique commande permettant de
      lister le contenu du répertoire de travail
      \lin{Intro\_Systeme} dans lequel vous avez l'habitude de travailler.
    \item Attribuez les droits d'exécution sur ce fichier. Il est normalement
      devenu un exécutable.
    \item Quelle commande permet d'exécuter ce script si le répertoire courant
      es le répertoire \lin{\~\//bin} qui le contient? Idem, si le répertoire
      courant est votre répertoire personnel. Vous vérifierez que le script se
      comporte comme attendu (il vous place dans une autre répertoire).
    \item la commande \lin{echo} permet d'afficher une message à
      l'écran. Modifiez le script pour qu'il avertisse l'utilisateur du
      changement de répertoire par un message explicite.
    \end{questions}
  \end{exercicelet}
\end{exercice}

\manpage{echo}

\begin{frame}{Lancer un programme/une commande}
  \begin{block}{Cas général}
    \begin{itemize}
    \item Pour exécuter un programme il suffit saisir sur la ligne de commande
      le chemin menant au fichier contenant les instructions,
    \item Si le fichier présente la permission \lin{"X"} pour exécutable, les
      instructions qu'il contient sont exécutées.
    \end{itemize}
  \end{block}
  \begin{block}{Cas particulier : les commandes}
    \begin{itemize}
    \item Une commande (\lin{ls}, \lin{gedit}, \lin{firefox}, \dots) est un
      programme comme un autre,
    \item Les instructions qui doivent être évaluées sont écrites dans un
      fichier (\lin{/bin/ls}, \lin{/usr/bin/python},
      \lin{/usr/share/bin/firefox}, \dots),
    \item Pourtant \dots
    \end{itemize}
  \end{block}
  \begin{alertblock}{Des chemins qui mènent nulle part !!!}
    \begin{itemize}
    \item les noms des commandes (\lin{ls}, \lin{gedit}, \lin{firefox} \dots)
      sont toujours saisies comme des chemins relatifs (pas de \lin{/bin/...}
      devant le nom du fichier), alors que le fichier de commande n'est pas
      dans le répertoire courant !\dots
    \item On donne donc un chemin vers un fichier qui n'existe pas \dots
    \end{itemize}
  \end{alertblock}
\end{frame}

\subsection{Chemins par défaut et variable d'environnement}
\begin{frame}{Chemins par défaut et variable d'environnement}
  \begin{alertblock}{Lorsqu'on donne une commande au terminal, on ne spécifie
      pas le chemin vers le fichier qui contient l'exécutable, on donne juste
      le nom du fichier\dots}
  \end{alertblock}
  \begin{center}
    \mprompt{ \prompt{ls}{%
        Mes\_Documents/ Etoiles/ astronomie.txt cv.pdf } \prompt{\cursor}{} }
  \end{center}
  \begin{alertblock}{\dots alors, comment le système trouve-t-il le fichier a
      exécuter correspondant à la commande ?\dots}
  \end{alertblock}
  \begin{block}{Un mécanisme propre aux commandes}
    \begin{itemize}
    \item Le premier mot tapé sur la ligne de commande est toujours
      interprétée comme le nom d'un fichier exécutable,
    \item Le système recherche donc dans une liste de répertoires contenant
      les exécutables si un fichier porte le nom de cette commande,
    \item Dès qu'il trouve dans ces répertoires un tel fichier, il l'exécute
      \dots
    \end{itemize}
  \end{block}
\end{frame}

\begin{frame}{Chemins par défaut et variable d'environnement}
  \begin{block}{Les variables d'environnement}
    \begin{itemize}
    \item Comme les variables d'un script, les variables d'environnement sont
      associées à une valeur,
    \item De telles variables sont définies par le système d'exploitation pour
      son fonctionnement, ce sont les variables d'environnement,
    \item ces variables peuvent être utilisées par les programmes.
    \end{itemize}
  \end{block}
  \begin{block}{La variable d'environnement \lin{\$PATH}}
    \begin{itemize}
    \item Sa valeur est une liste de répertoires séparés par le signe \lin{':'}\\
      \begin{center}
        \linbox{PATH=repertoire1\NoAutoSpaceBeforeFDP:repertoire2\NoAutoSpaceBeforeFDP:...\NoAutoSpaceBeforeFDP:RepertoireN}
      \end{center}
    \item Lors de chaque appel de commande, l'interpréteur parcourt cette
      liste dans l'ordre à la recherche d'un fichier portant le nom de la
      commande,
    \item Dès qu'il rencontre un tel fichier, il met fin à sa recherche et
      exécute le fichier.
    \end{itemize}
  \end{block}
  \begin{alertblock}{Rôle de \lin{\$PATH}}
    \begin{itemize}
    \item[\textrightarrow] Il s'agit d'une liste de répertoires que
      l'interpréteur parcours automatiquement et séquentiellement (par défaut)
      si aucun chemin n'est donné pour trouver le fichier exécutable.
    \end{itemize}
  \end{alertblock}
\end{frame}

%%%%%%%%%%%%%% 
\manpage{which}
%%%%%%%%%%%%%% 
\begin{frame}{Chemins par défaut et variable d'environnement}
  \begin{block}{La commande \lin{export} pour modifer la variable
      \lin{\$PATH}}
    \begin{itemize}
    \item Définir la variable \lin{\$PATH}
      \begin{center}
        \small{ \mprompt{ \prompt[\~\//]{export
              PATH=monDir1\NoAutoSpaceBeforeFDP:monDir2}{} } }
      \end{center}
    \item Ajouter un répertoire à \lin{\$PATH}
      \begin{center}
        \small{ \mprompt{ \prompt[\~\//]{export
              PATH=\$PATH\NoAutoSpaceBeforeFDP:monDir2}{} } }
      \end{center}
    \end{itemize}
  \end{block}
\end{frame}
\begin{exercice}
  \begin{exercicelet}{Environnement}
    \begin{questions}
    \item Au moyen de la commande \lin{env}, donnez la liste des répertoires
      contenus dans \lin{\$PATH}.
    \item Au moyen de la commande \lin{which}, afficher la localisation des
      exécutables correspondants aux commandes \lin{mv}, \lin{cd}, \lin{man},
      \lin{cat}, \lin{firefox}, \lin{acroread}.
    \item Vérifiez que ces répertoires font partie de la liste contenue dans
      la variable \lin{\$PATH}? Que se passerait-il si ce n'était pas le cas?
    \item Ajouter le répertoire \lin{\~\//bin} à la liste des répertoires
      \lin{\$PATH}.
    \item Maintenant que \lin{\~\//bin} est parcoure par défaut lors de
      l'appel d'une commande, comment invoque-t-on désormais l'exécution du
      script \lin{listintro.sh}? Vérifiez le comportement attendu.
    \end{questions}
  \end{exercicelet}
\end{exercice}

\subsection{Configuration des variables d'environnement}
\begin{frame}{Fichiers de configuration}
  \begin{block}{Fichiers systèmes et utilisateurs}
    \begin{itemize}
    \item Les variables d'environnement (et d'autres variables de
      configuration) sont définis dans divers fichiers.
    \item On distingue les fichiers système qui définissent des comportements
      pour tous les utilisateurs (stockés dans le répertoire \lin{/etc/}) des
      fichiers de configuration propres à un utilisateur (stockés dans le
      répertoire personnel)
    \end{itemize}
  \end{block}
  \begin{center}
    \begin{tabular}{llll}
      \hline
      fichier&Propriétaire&Applicable à& Évalué lors\\
      \hline
      \lin{/etc/profile}&\lin{root}&Tous&Au début de chaque shell de login\\
      \lin{/home/chez\_moi/.profile}&\lin{utilisateur}&utilisateur&Au début de chaque shell de loginl\\
      \lin{/etc/bashrc}&\lin{root}&Tous&Au début de chaque shell\\
      \lin{/home/chez\_moi/.bashrc}&\lin{utilisateur}& utilisateur&Au début de chaque shell\\
      \hline
    \end{tabular}
  \end{center}
  \begin{block}{Configurer son environnement}
    \begin{itemize}
    \item Chaque utilisateur peut redéfinir ses variables d'environnement,
    \item Pour cela il peut modifier le contenu des fichiers \lin{.bashrc} et
      \lin{.profile} dans son répertoire personnel,
    \item Ce sont des fichiers cachés (leur nom commence par un point:
      \lin{.}). Pour voir si ils existent il faut utiliser l'option \lin{-a}
      de la commende \lin{ls}.
    \end{itemize}
  \end{block}
\end{frame}
%%%%%%%%%%%%%% 
\begin{frame}{Fichiers de configuration}
  \begin{block}{Contenu d'un fichier \lin{.bashrc}}
    \begin{itemize}
    \item Redéfinition des variables d'environnement,
    \item Définition des alias,
    \item Définition des fonctions,
    \item et de façon générale toutes les instructions que l'on souhaite
      évaluer lors de l'ouverture d'un nouveau shell.
    \end{itemize}
  \end{block}
  \begin{center}
    \fileform[6cm]{.bashrc}{\# Mes aliases\\
      alias ll='ls -l'\\
      alias df='df -h'\\
      alias rm='rm -i'\\
      \# Mes variables\\
      PATH=\$PATH\NoAutoSpaceBeforeFDP:\$HOME/bin }
  \end{center}
  \begin{alertblock}{Autres variables d'environnement}
    \begin{description}
    \item[\lin{\$HOME}] le chemin du répertoire personnel de l'utilisateur,
    \item[\lin{\$PWD}] le chemin du répertoire courant.
    \end{description}
  \end{alertblock}

\end{frame}

\manpage{alias}

\begin{exercice}
  \begin{exercicelet}{Chemins par défaut et variable d'environnement}
    \begin{questions}
    \item Copiez l'exécutable de la commande \lin{ls} dans le répertoire
      \lin{\~\//bin}. Deux versions de la même commande existe dans 2
      répertoires différents listés sans \lin{\$PATH}. Quelle commande est
      exécutée? Comment en être sur et pourquoi?
    \item Si vous modifiez la variable \lin{\$PATH}, de la façon suivante,
      quelle commande est alors exécutée?
      \begin{center}
        \small{ \mprompt{ \prompt[\~\//]{export
              PATH=monDir2\NoAutoSpaceBeforeFDP:\$PATH}{} } }
      \end{center}
    \item Modifiez/créez un fichier \lin{\~\//.bashrc} pour ajouter le
      répertoire \lin{\~\//bin} de façon stable à votre variable \lin{\$PATH}.
    \item ajoutez dans le même fichier les alias qui vous paraissent
      intéressants.
    \end{questions}
  \end{exercicelet}    
\end{exercice}

% Local Variables:
% TeX-master: "sys04"
% TeX-PDF-mode: t
% fill-column: 78
% coding: utf-8-unix
% mode-require-final-newline: t
% mode: latex
% mode: flyspell
% ispell-local-dictionary: "francais"
% End:
