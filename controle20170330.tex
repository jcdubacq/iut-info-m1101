\documentclass[a4paper]{iutvexam}
% compile-latex options:--jobname controle20170330
% compile-latex options:--jobname correction20170330
\usepackage{textcomp}
\usepackage{dirtree}
\usepackage{listings}
\usepackage{caption}
\newcommand{\DTd}[1]{\textbf{#1/}}
\newcommand{\DTfcomment}[1]{\DTcomment{\emph{#1}}}

\title{Contrôle 2}
\date{30/03/2017}
\begin{document}
\conditions{ Vous disposez de 90 minutes pour faire ce contrôle. Aucun
  document autorisé. Toute tentative de communication avec un voisin ou
  l'extérieur peut être sanctionnée. Toutes les réponses doivent être
  faites sur l'énoncé. La taille de la réponse attendue dépend de la
  taille allouée pour répondre. }
\lstset{
  extendedchars=true,
  literate={è}{{\`e}}1 {é}{{\'e}}1 {î}{{\^\i}}1
}


L'ensemble de ce contrôle tourne autour d'un format de police de caractères\footnote{C'est un format imaginaire}. Une police au format NFP (nouveau format de police) est composé par les structures qui sont détaillées ci-après. Nous allons étudier un certain nombre de détails de ce format.

Tout d'abord, une police est un assemblage de blocs de données qui
sont d'un des trois types suivants : méta-données, données bitmap pour
une taille fixée, table de correspondance numéro de caractère vers
glyphe.

Les fichiers sont construits comme des copies de la mémoire d'une
machine en mode LE 32bits. L'adresse \texttt{0x0000} correspond au
début du fichier.
\begin{questions}

  \titledquestion{Les méta-données}  
  Un bloc de méta-données permet de stocker un couple
  \texttt{(cle,valeur)} qui est une propriété du genre
  \texttt{("style","italic")}. Le format des méta-données est le suivant
  (syntaxe similaire au C):
  
  \begin{lstlisting}[numbers=left,language=C,caption=Programme]
    struct blocMeta {
      int longueur; /* longueur en octets du bloc de méta-données */
      char version; /* La valeur 1 (tout le temps) */
      char typebloc; /* type de bloc, valeur 0 pour un bloc de méta-données */
      char l_longueur; /* longueur réelle de la clé (max 16) */
      char l_valeur; /* longueur réelle de la valeur (max 16) */
      char cle[16]; /* nom de la propriété */
      char valeur[16]; /* nom de la valeur */
    }
  \end{lstlisting}
  
  \begin{parts}
    \part[1] Il y a une limite au nombre de méta-données que l'on peut inclure dans un fichier : laquelle ? Est-ce que cette limite peut être dépassée de façon réaliste ?
    \begin{solutionordottedlines}[.25in]
      126-32=94 caractères. $94^{16}+94^{15}+\dots+94^{0}$ est un nombre très grand.
    \end{solutionordottedlines}
    \part[\half] Est-ce qu'il est nécessaire de mettre un caractère nul
    à la fin du nom de la propriété ? Pourquoi ?
    \begin{solutionordottedlines}[.25in]
      Non, puisque la longueur est déjà indiquée dans la structure.
    \end{solutionordottedlines}
    \part[2] Est-ce qu'il y a des contraintes d'alignement dans cette
    structure ? Quelle est la longueur de cette structure ? Faites un
    schéma de la structure et des places occupées par les valeurs.
    \begin{solutionordottedlines}[.5in]
      Non, pas de contrainte. 4 cases, 4 fois une case, 16 cases, 16
      cases
    \end{solutionordottedlines}
    \part[1] Représentez les octets associés au bloc de méta-données qui
    sert à coder l'information \texttt{("FAMILY","Times New")} (vous
    noterez les lettres "A", "B", etc. au lieu de \texttt{0x41},
    \texttt{0x42}, etc.).
    \begin{solutionordottedlines}[.25in]
      ...
    \end{solutionordottedlines}
  \end{parts}
  \titledquestion{Les blocs de correspondance}

  Un bloc de correspondance permet de faire le lien entre des numéros et
  des noms de glyphe (une chaîne d'au plus 32 caractères). On peut
  mettre plusieurs blocs de correspondances dans une même police pour
  qu'elle puisse servir avec plusieurs jeux de caractères (qui se
  recouvrent).

  La structure d'un bloc est la suivante:
  \begin{lstlisting}[numbers=left,language=C,caption=Programme]
    struct blocTable {
      int longueur; /* longueur en octets du bloc de correspondance, y compris tableau et chaîne de caractères */
      char version; /* La valeur 1 (tout le temps) */
      char typebloc; /* type de bloc, valeur 1 pour un bloc de correspondances */
      char* characterset; /* nom du jeu de caractères */
      char num_glyphes; /* nombre de glyphes existants pour le jeu de caractères */
      struct glyphe* glyphe; /* table des glyphes */
    }

    struct glyphe {
      int glyphe; /* numéro du glyphe dans la table des glyphes */
      int character; /* numéro du caractère dans le jeu de caractère */
    }
  \end{lstlisting}

  Les chaînes de caractères se trouvent derrière le bloc qui s'en sert. Ce sont des chaînes de caractères de type C.
  \begin{parts}
    \part[1] Refaites le schéma qui explique les relations entre touches, jeu de caractères, etc.
    \begin{solutionorbox}[2in]
      
    \end{solutionorbox}
    \part[2] Est-ce qu'il y a des contraintes d'alignement dans la première
    structure ? Quelle est la longueur de cette structure ? Faites un
    schéma de la structure et des places occupées par les valeurs.
    \begin{solutionordottedlines}[.5in]
      Oui, un trou de deux ou trois octets avant les pointeurs dans la première. 4 cases, 2 fois une case, trou, 4 cases, 1 case, trou de trois octets, 4 cases.
      cases
    \end{solutionordottedlines}
    \part[1] Est-ce qu'il y a des contraintes d'alignement dans la deuxième
    structure ? Quelle est la longueur de cette structure ? Faites un
    schéma de la structure et des places occupées par les valeurs.
    \begin{solutionordottedlines}[.5in]
      Non, aucune contrainte.
    \end{solutionordottedlines}
    \part[4] On va représenter à partir de l'adresse 100 un tel
    bloc. Comme la police n'est pas finie, il n'y a que deux caractères
    qui correspondent au jeu de caractères "ASCII". Le premier caractère
    est le "A" (code 0x41 en ASCII, glyphe numéro 1 dans la table des
    glyphes), le deuxième est le "B" (code 0x42 en ASCII, glyphe numéro
    2). Représentez successivement le bloc de correspondance, la chaîne
    de caractères "ASCII", le tableau de correspondance proprement dit.
    \begin{solutionorbox}[1in]
      
    \end{solutionorbox}
  \end{parts}

  \titledquestion{Les glyphes}  
  Un bloc de glyphe permet de stocker un glyphe (qui est un dessin point par point, dit \emph{bitmap}).
  Une police de caractères comporte donc un bloc de glyphe pour chaque glyphe. Chaque bloc est suivi des données bitmap: un pixel blanc est représenté par un bit à 0, un pixel noir par un bit à 1. Lorsqu'il y a un nombre de pixels qui n'est pas un multiple de 8, on complète par des (faux) pixels \emph{à gauche} pour arriver à un multiple de 8.
  \begin{lstlisting}[numbers=left,language=C,caption=Programme]
    struct blocGlyphe {
      int longueur; /* longueur en octets du bloc de méta-données */
      char version; /* La valeur 1 (tout le temps) */
      char typebloc; /* type de bloc, valeur 2 pour un bloc de glyphes */
      char wd; /* dimension 1 */
      char ht; /* dimension 2 */
      int num; /* numéro de glyphe */
      char* bitmap; /* données */
    }
  \end{lstlisting}
  \begin{parts}
    \part[\half] Quel est le nom anglais et le nom français des deux dimensions d'un glyphe ?
    \begin{solutionordottedlines}[.25in]
    \end{solutionordottedlines}
    \part[1] Est-ce qu'il y a des contraintes d'alignement dans cette
    structure ? Quelle est la longueur de cette structure ? Faites un
    schéma de la structure et des places occupées par les valeurs.
    \begin{solutionordottedlines}[.5in]
    \end{solutionordottedlines}
    \part[2] Voilà les octets d'une structure qui commence à l'adresse 0x200. Faites le dessin qui convient
\begin{verbatim}
00000200    18 00 00 00  01 02 06 08  0F 02 00 00  01 00 00 00
00000210    00 08 08 14  1C 22 22 36
\end{verbatim}
    \begin{solutionorbox}[3in]
      
    \end{solutionorbox}
  \end{parts}
\end{questions}
\end{document}
