\begin{frame}{Organisation du module}
  \begin{alertblock}{Remerciements}
    \begin{itemize}
    \item Les cours et exercices de ce module sont directement inspirés
      des documents de \textbf{M. Bosc}, \textbf{J.-C. Dubacq} et \textbf{G. Santini}.
    \item D'autres intervenants ont participé à l'élaboration des supports.
    \end{itemize}
  \end{alertblock}
  \begin{block}{Les enseignements}
    \begin{itemize}
    \item 12 sessions de 4h et du travail personnel \dots
    \item 6 sessions pour la présentation générale du système
      d'exploitation Linux,
    \item 6 sessions pour la théorie de base du codage informatique 
    \end{itemize}
  \end{block}
  \begin{alertblock}{Votre présence est obligatoire}
    \begin{itemize}
    \item Contrôle des présences.
    \item Rapport des absences.
    \end{itemize}
  \end{alertblock}
  \begin{block}{L'évaluation}
    \begin{itemize}
    \item Une composition après la sixième session (sur papier ou sur
      ordinateur).
    \item Une composition à la fin du module (sur papier ou sur
      ordinateur).
    \end{itemize}
  \end{block}
\end{frame}
