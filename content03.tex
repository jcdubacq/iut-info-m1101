\section{Les opérations}
\subsection{Les entiers}
\begin{exercice}
  \begin{exercicelet}{De la fonction à l'algorithme}
    \objectif{Maîtriser la différence entre fonction et
      algorithme. Concevoir un algorithme simple. Examiner l'influence de la
      représentation sur l'algorithme utilisé pour une même fonction.}

    La numération grecque (simple) est proche de la numération romaine que
    vous connaissez : on note les nombres comme suit:
    \[\begin{array}{c|c|c|c|c|c|c|c|c|c}
      1&5&10&50&100&500&1\,000&5\,000&10\,000&50\,000\\\hline
      \Iota&\Gamma&\Delta&\Gamma_\Delta&\Chi&\Gamma_\Chi&\Eta&\Gamma_\Eta&\Mu&\Gamma_\Mu
    \end{array}
    \] C'est à la différence près que l'on a pas de règle soustractive: le
    nombre $4$ s'écrit $\Iota\Iota\Iota\Iota$, pas $\Iota\Gamma$. La
    position des chiffres n'a théoriquement aucune importance, mais on les
    classait dans l'ordre décroissant de valeur.
    \begin{enumerate}
    \item Ce système est-il un système de numération positionnelle ?
      \begin{correction}Non\end{correction}
    \item Écrivez votre âge et votre date de naissance en numération
      grecque.
      \begin{xcorrection}$\Eta\Gamma_\Chi\Chi\Chi\Chi\Chi\Gamma_\Delta\Delta\Delta\Iota\Iota\Iota$
        et $\Delta\Delta\Delta\Gamma\Iota\Iota\Iota$ pour moi.\end{xcorrection}
    \item Écrivez un algorithme d'addition des nombres représentés en
      numération grecque. Est-ce que cet algorithme est le même qu'en décimal
      ?
      \begin{xcorrection}On prend les nombres à additionner, on les colle
        les uns aux autres, on rassemble les chiffres identiques. Ensuite on
        répète en partant de la droite : tant que l'on a cinq (ou deux) chiffres
        identiques, on les enlève et on fabrique un chiffre suivant à leur
        place.

        C'est un exemple, il y en a plusieurs possibles.\end{xcorrection}
    \item Faites l'addition de votre âge et de votre année de naissance
      avec votre algorithme (vous devriez obtenir
      $\Eta\Eta\Delta\Iota\Iota\Iota$ ou $\Eta\Eta\Delta\Iota\Iota$). De
      quelle représentations partez-vous ?
      \begin{correction}On part des représentations grecques, bien
        sûr. Ou alors des représentations en décimal, que l'on
        convertit, mais 1-- ça a déjà été fait et 2-- ça ne doit pas
        faire partie de l'algorithme.\end{correction}
    \item Faites la même chose en décimal. De quelles représentations
      partez-vous ? Est-ce que l'algorithme est le même ? Est-ce que
      la fonction calculée est la même ?
      \begin{xcorrection} La représentation en décimal sert de départ,
        l'algorithme est différent (même si un peu similaire), mais le résultat
        (et donc la fonction calculée) est identique.
      \end{xcorrection}
    \end{enumerate}
  \end{exercicelet}
\end{exercice}
\begin{frame}{Addition dans les systèmes positionnels}
  \begin{columns}[c]
    \begin{column}{0.45\linewidth}
      \begin{methode}[Addition]
        L'addition en base $B$ se fait de la droite vers la gauche, colonne
        par colonne, en utilisant le fait que la somme de \emph{chiffres}
        s'écrit sous la forme $s+Br$, où $s$ est la \emph{somme partielle}
        (un chiffre unique) et $r$ est la \emph{retenue}. Le poids de $r$
        est $B$ fois plus important, et $r$ est donc remise dans la colonne
        d'à côté.
      \end{methode}
    \end{column}
    \begin{column}{0.5\linewidth}
      \begin{example}[Addition en base 2]
        \begin{center}
          \begin{tikzpicture}[scale=0.5]
            \tikzstyle{hig}=[circle,fill=solarizedYellow]
            \tikzstyle{higx}=[rounded corners,fill=solarizedRed]
            \tikzstyle{higr}=[circle,fill=solarizedRebase0]
            \only<12->{\node (pointj) at (0,0) {$1$} ;}
            \only<10->{\node (pointk) at (1,0) {$0$} ;}
            \only<8->{\node (pointl) at (2,0) {$1$} ;}
            \only<6->{\node (pointm) at (3,0) {$0$} ;}
            \only<4->{\node (pointn) at (4,0) {$0$} ;}
            \only<2->{\node (pointo) at (5,0) {$1$} ;}

            \only<beamer|beamer:2>{\node[hig] (higa) at (5,2.5) {} ;}
            \only<beamer|beamer:2>{\node[hig] (higb) at (5,1.5) {} ;}
            \only<beamer|beamer:2>{\node[hig] (higf) at (5,3.5) {} ;}
            \only<beamer|beamer:2>{\node[higx] (higc) at (5,0) {$1$} ;}

            \only<beamer|beamer:3-4>{\node[hig] (higa) at (4,2.5) {} ;}
            \only<beamer|beamer:3-4>{\node[hig] (higb) at (4,1.5) {} ;}
            \only<beamer|beamer:3-4>{\node[hig] (higf) at (4,3.5) {} ;}
            \only<beamer|beamer:3>{\node[higx] (higc) at (4,0) {$10$} ;}
            \only<beamer|beamer:4>{\node[higx] (higc) at (4,0) {$0$} ;}
            \only<beamer|beamer:4>{\node[higr] (higd) at (3,4.5) {\small $1$} ;}

            \only<beamer|beamer:5-6>{\node[hig] (higa) at (3,2.5) {} ;}
            \only<beamer|beamer:5-6>{\node[hig] (higb) at (3,1.5) {} ;}
            \only<beamer|beamer:5-6>{\node[hig] (hige) at (3,4.5) {} ;}
            \only<beamer|beamer:5-6>{\node[hig] (higf) at (3,3.5) {} ;}
            \only<beamer|beamer:5>{\node[higx] (higc) at (3,0) {$100$} ;}
            \only<beamer|beamer:6>{\node[higx] (higc) at (3,0) {$0$} ;}
            \only<beamer|beamer:6>{\node[higr] (higd) at (2,4.5) {\small $10$} ;}

            \only<beamer|beamer:7-8>{\node[hig] (higa) at (2,2.5) {} ;}
            \only<beamer|beamer:7-8>{\node[hig] (higb) at (2,1.5) {} ;}
            \only<beamer|beamer:7-8>{\node[hig] (hige) at (2,4.5) {} ;}
            \only<beamer|beamer:7>{\node[higx] (higc) at (2,0) {$11$} ;}
            \only<beamer|beamer:8>{\node[higx] (higc) at (2,0) {$1$} ;}
            \only<beamer|beamer:8>{\node[higr] (higd) at (1,4.5) {\small $1$} ;}

            \only<beamer|beamer:9-10>{\node[hig] (higa) at (1,2.5) {} ;}
            \only<beamer|beamer:9-10>{\node[hig] (hige) at (1,4.5) {} ;}
            \only<beamer|beamer:9>{\node[higx] (higc) at (1,0) {$10$} ;}
            \only<beamer|beamer:10>{\node[higx] (higc) at (1,0) {$0$} ;}
            \only<beamer|beamer:10>{\node[higr] (higd) at (0,4.5) {\small $1$} ;}
            
            \only<beamer|beamer:11>{\node[hig] (hige) at (0,4.5) {} ;}
            \only<beamer|beamer:11>{\node[higx] (higc) at (0,0) {$1$} ;}

            \only<beamer|beamer:4,6,8,10>{\path[<-] (higd) edge[dashed] (higc);}
            \only<beamer|beamer:2-10>{\path[<-] (higc) edge[bend right, bend angle=20] (higa);}
            \only<beamer|beamer:2-10>{\path[<-] (higc) edge[bend right, bend angle=20] (higa);}
            \only<beamer|beamer:2-6>{\path[<-] (higc) edge[bend left, bend angle=20] (higf);}
            \only<beamer|beamer:2-8>{\path[<-] (higc) edge (higb);}
            \only<beamer|beamer:5-11>{\path[->] (hige) edge[bend right, bend angle=20] (higc);}
            \only<4->{\node[color=red] (pointp) at (3,4.5) {\small $1$} ;}
            \only<6->{\node[color=red] (pointq) at (2,4.5) {\small $10$} ;}
            \only<8->{\node[color=red] (pointr) at (1,4.5) {\small $1$} ;}
            \only<10->{\node[color=red] (points) at (0,4.5) {\small $1$} ;}
            \only<beamer|beamer:12>{\node (pointzzz) at (0,0) {} ;}
            \foreach \poids/\etape in {1/2,2/3-4,4/5-6,8/7-8,16/9-10,32/11}
            \only<beamer|beamer:\etape>{\node[starburst,starburst point height=0.2cm,inner sep=2pt,text=solarizedRebase03,fill=solarizedRebase3,draw=solarizedYellow,very thick] at (5,4.5) {\tiny poids $\poids$};}
            \node (pointz) at (6.5,6) {} ;
            \node (pointzz) at (0,-0.3) {} ;
            \node (pointa) at (2,1.5) {$1$} ;
            \node (pointb) at (3,1.5) {$1$} ;
            \node (pointc) at (4,1.5) {$1$} ;
            \node (pointd) at (5,1.5) {$1$} ;
            \node (add1) at (0,2.5) {$+$} ;
            \node (add2) at (0,1.5) {$+$} ;
            \node (pointe) at (1,2.5) {$1$} ;
            \node (pointf) at (2,2.5) {$0$} ;
            \node (pointg) at (3,2.5) {$1$} ;
            \node (pointh) at (4,2.5) {$1$} ;
            \node (pointi) at (5,2.5) {$0$} ;
            \node (pointp) at (3,3.5) {$1$} ;
            \node (pointq) at (4,3.5) {$0$} ;
            \node (pointr) at (5,3.5) {$0$} ;
            \draw[very thick] (-1,1)--(6,1) ;
          \end{tikzpicture}
        \end{center}
        \uncover<6-7>{On peut aussi marquer la retenue $10$ avec $0$ dans
          la colonne suivante et $1$ dans la colonne d'ordre encore
          supérieur.}
      \end{example}
    \end{column}
  \end{columns}
  \begin{itemize}
  \item[\ddialoginformation] Truc: pour additionner une colonne, on peut
    bien sûr le faire en décimal, à condition de repasser au binaire
    pour le reste et les retenues.
  \end{itemize}
\end{frame}
\begin{frame}{Multiplication}
  \begin{columns}[c]
    \begin{column}{0.65\linewidth}
      \begin{methode}[Multiplication]
        À l'instar de l'addition, la multiplication se fait comme pour les
        nombres décimaux. Les tables sont justes différentes (il faut les
        écrire dans la bonne base !).

        En binaire, c'est très facile: on multiplie par $0$ ou par $1$, donc
        on se contente d'additionner des copies du multiplicande décalées là
        où le multiplicateur a des $1$.

        \textbf{Très important:} décaler à gauche de $n$ colonnes un nombre,
        c'est le multiplier par $B^n$. Le décaler à droite, c'est le diviser
        par $B^n$.
      \end{methode}
    \end{column}
    \begin{column}{0.35\linewidth}
      $$\begin{array}{cccccc}
        &&&1&1&0\\
        \times&&&1&0&1\\\hline
        &&&1&1&0\\
        (+&&&&0&)\\
        +&1&1&0&&\\\hline
        &1&1&1&1&0
      \end{array}
      $$
    \end{column}
  \end{columns}
\end{frame}
\begin{frame}{Négatifs et réels}
  Par analogie avec les techniques maîtrisées en base 10, il est
  possible de faire également:
  \begin{itemize}
  \item Des soustractions: lorsque le chiffre duquel on soustrait n'est
    pas suffisant, on ajoute 1 au chiffre à soustraire dans la colonne
    d'ordre immédiatement supérieur, et en compensation, on ajoute la
    base dans la colonne courante.
  \item Des divisions: en fait, on fait plein de multiplications
    enchaînées avec des soustractions.
  \item Additionner un négatif, c'est soustraire un positif.
  \item Des opérations en virgule fixe: les règles de placement de la
    virgule sont les mêmes qu'en base 10.
  \item Des décalages qui sont des multiplications ou divisions par une
    puissance de la base.
  \item Pour la virgule flottante, les multiplications sont simples ;
    les additions nécessitent de recoder en virgule fixe.
  \end{itemize}
\end{frame}
\begin{exercice}
  \begin{exercicelet}{Calcul en binaire et hexadécimal}
    \objectif{Maîtriser les conversions et les additions en base 2 et 16,
      maîtriser la multiplication en base 2, comprendre la notion de poids
      des chiffres dans un système positionnel}
    \begin{enumerate}
    \item Faites les additions en binaire:
      \texttt{0b1101\,0101}+\texttt{0b1110\,0101};
      \texttt{0b1,1}+\texttt{0b110}+\texttt{0b100,1}+\texttt{0b111,1}+%
      \texttt{0b1010,1}+\texttt{0b100,1}.
      \begin{xcorrection}$1\,1011\,1010$ et $10\,0010,1$ (ils peuvent
        vérifier en décimal).\end{xcorrection}
    \item Faites les opérations suivantes en hexadécimal: $0x122+0x233$;
      $0x87+0x54$; $0x18+0x9$; $0xED+0xED$; $0x100-0x3$.
      \begin{xcorrection}355,DB,21,1DA,FD\end{xcorrection}
    \item Faites la multiplication suivante: $17\times129$ à la fois en
      décimal et en binaire.
      \begin{xcorrection}2193 et 100010010001.\end{xcorrection}
    \item Faites la multiplication suivante en binaire: $110110\times 1101$.
      \begin{xcorrection}1010111110 (54*13=702)\end{xcorrection}
    \end{enumerate}
  \end{exercicelet}
\end{exercice}
\subsection{Addition et codage}
\begin{frame}{Les algorithmes d'addition}
  \begin{itemize}
  \item L'addition de codage se fait par un algorithme
    (représentation/information). Si l'algorithme fait ce qu'on attend
    de lui (identique à la fonction), il est \emph{correct}.
  \item Tous les nombres ne sont pas représentables. Si le résultat de
    l'addition (fonction) donne un nombre non représentable,
    l'algorithme ne peut pas être correct.
  \item L'algorithme d'addition pour NAT et C2 est très similaire à la
    méthode usuelle (décrite avant); on coupe juste le résultat à la
    taille du codage.
  \end{itemize}
  \begin{theorem}
    En C2 et NAT, si le résultat est représentable, l'addition
    (classique) est correcte.
  \end{theorem}
\end{frame}
\begin{frame}{Addition correcte ou pas ?}
  \begin{example}[Addition en NAT]
    \[\begin{array}{ccc}
      131+85 & \stackrel{\text{code}}{\Longrightarrow} & 1000\,0011_{NAT}+0101\,0101_{NAT}\\
      \multicolumn{2}{c}{\only<2->{\alert<beamer:2>{\text{Correct !}}}} & \stackrel{\text{algo}}{=}\\
      216 & \stackrel{\text{decode}}{\Longleftarrow} & 1101\,1000_{NAT}\\
      \uncover<3->{187+101} & \uncover<3->{\stackrel{\text{code}}{\Longrightarrow}} & \uncover<3->{1011\,1011_{NAT}+0110\,0101_{NAT}}\\
      \multicolumn{2}{c}{\only<4->{\alert<beamer:4>{\text{Incorrect !}}}} & \uncover<3->{\stackrel{\text{algo}}{=}}\\
      \uncover<3->{32(\neq288)} & \uncover<3->{\stackrel{\text{decode}}{\Longleftarrow}} & \uncover<3->{0010\,0000_{NAT}}\\
    \end{array}\]
  \end{example}
  \begin{itemize}
  \item En C1 et en VA+S, il suffit de mettre un négatif pour opération
    incorrecte
  \item En plus, problème du double zéro pour C1 et VA+S.
  \item C2: comme si le bit de poids fort était de poids $-2^{n-1}$ au lieu de
    $2^{n-1}$, arithmétique modulo $2^n$.
  \end{itemize}
\end{frame}
\begin{frame}{Notes sur la multiplication et l'hexadécimal}
  \begin{itemize}
  \item La multiplication est extrêmement simple en binaire. Elle se comporte comme une série d'additions.
  \item La multiplication est rarement correcte si on a autant de
    chiffres en entrée qu'en sortie. La plupart des ordinateurs
    multiplient en mettant le résultat dans deux mots de code distincts
    (partie haute et partie basse).
  \end{itemize}
\end{frame}
\begin{exercice}
  \begin{exercicelet}{Limites de la multiplication}
    Expliquez pourquoi le résultat d'une multiplication de deux nombres
    représentés dans l'un des 4 codes classiques est toujours
    représentable à condition de doubler la taille du code.
    \begin{xcorrection}
      Il suffit de regarder la valeur maximale autorisée par un code et
      de l'élever au carré (produit le plus grand possible).
    \end{xcorrection}
  \end{exercicelet}
  \begin{exercicelet}{Addition en C2}
    Faites les opérations suivantes en transformant les nombres au
    préalable en codage C2 sur 8 bits (résultat aussi en C2 sur 8 bits):
    \begin{itemize}
    \item 45+17
    \item 45-17 (soit 45+(-17))
    \item -17-17
    \item 17-45
    \item 221+45
    \end{itemize}
    Dites aussi si le résultat obtenu est correct et s'il est représentable.
    \begin{xcorrection}
      00101101+00010001=...\\
      00101101+11101111=...\\
      11101111+11101111=11011110\\
      ...+...=...\\ %TODO
      11011101+00101101=00001010\\
      Ils sont tous corrects s'ils sont représentables, donc tous sauf le dernier.
    \end{xcorrection}
  \end{exercicelet}
\end{exercice}
\subsection{Les champs de bits}
\begin{frame}{La logique booléenne}
  \begin{itemize}
  \item Il est possible d'interpréter les deux valeurs binaires comme
    représentant respectivement \emph{vrai} (1) ou \emph{faux} (0).
  \item On peut définir des opérations correspondantes à la conjonction (et),
    la disjonction (ou) et la négation (opposé de).
  \item[\dialogerror] Ce ne sont pas des opérations arithmétiques classiques.
  \item[\dialogsystem] $c=ab$ ou $c=a\wedge b$ pour le \emph{et} (en C:
    $a=b\&c$)
  \item[\dialogsystem] $c=a+b$ ou $c=a\vee b$ pour le \emph{ou} (en C:
    $a=b|c$)
  \item[\dialogsystem] $b=\overline{a}$ ou $b=\neg a$ pour le \emph{non} (en
    C: $a=\sim c$)
  \end{itemize}
  \begin{example}[chemin sous UNIX]
    Soit $\mathcal A(p)$ la propriété « $p$ est un chemin existant » et
    $\mathcal B(p)$ la propriété « $p$ est un chemin qui désigne un répertoire
    ». Si on suppose qu'il n'y a que deux type d'objets (répertoires et
    fichiers), la propriété $\mathcal C(p)$ « $p$ est un chemin qui désigne un
    fichier » s'exprime par $\mathcal A(p)\overline{\mathcal B(p)}$.
  \end{example}
\end{frame}
\begin{frame}{Les opérateurs booléens}{AND et OR}
  \begin{block}{AND}
    L'opérateur binaire AND, noté $a\times b$, renvoie $1$ si et seulement si
    ses deux arguments sont égaux à $1$.

    L'opérateur général AND renvoie $1$ si et seulement si tous ses arguments
    sont égaux à $1$.

    Ils sont équivalents à la fonction \emph{minimum}.
  \end{block}
  \begin{block}{OR}
    L'opérateur binaire OR, noté $a+b$, renvoie $0$ si et seulement si ses
    deux arguments sont égaux à $0$.

    L'opérateur général OR renvoie $0$ si et seulement si tous ses arguments
    sont égaux à $0$.

    Ils sont équivalents à la fonction \emph{maximum}.
  \end{block}
\end{frame}
\begin{frame}{Les opérateurs booléens}{XOR et NOT}
  \begin{block}{XOR}
    L'opérateur binaire XOR, noté $a\oplus b$, renvoie $1$ si et seulement si
    ses deux arguments sont différents.

    Ils sont équivalents à la fonction \emph{différent}.
  \end{block}
  \begin{block}{NOT}
    L'opérateur unaire NOT, noté $\overline{a}$, renvoie $0$ si et seulement
    si son argument est égal à $1$.

    Il est équivalent à la fonction \emph{complémentation}.

    Il y a aussi les opérateurs généraux NOR et NAND qui sont en fait
    NOT(OR(...)) et NOT(AND(...)). Ils sont rarement implémentés dans les
    langages de programmation.
  \end{block}
\end{frame}
\begin{exercice}
  \begin{exercicelet}{Tables de vérité}
    Faites une table qui montre toutes les paires d'arguments possibles pour
    les opérateurs AND, OR, XOR et qui montre le résultat à côté.
    \centerline{\LARGE $\begin{tabular}{|cc|c|}\hline A&B&A$\times$B\\\hline
        0&0&\begin{correction}0\end{correction}\\
        0&1&\begin{correction}0\end{correction}\\
        1&0&\begin{correction}0\end{correction}\\
        1&1&\begin{correction}1\end{correction}\\
        \hline\end{tabular}$\hfill $\begin{tabular}{|cc|c|}\hline
        A&B&A$+$B\\\hline
        0&0&\begin{correction}0\end{correction}\\
        0&1&\begin{correction}1\end{correction}\\
        1&0&\begin{correction}1\end{correction}\\
        1&1&\begin{correction}1\end{correction}\\
        \hline\end{tabular}$\hfill $\begin{tabular}{|cc|c|}\hline
        A&B&A$\oplus$B\\\hline
        0&0&\begin{correction}0\end{correction}\\
        0&1&\begin{correction}1\end{correction}\\
        1&0&\begin{correction}1\end{correction}\\
        1&1&\begin{correction}0\end{correction}\\
        \hline\end{tabular}$ }
  \end{exercicelet}
\end{exercice}
\begin{frame}{Les champs de bits}
  \begin{itemize}
  \item La notion de variable booléenne stockant une valeur vraie ou faux est
    très souvent intégrée directement dans les langages
  \item[\dialogerror] Ce n'est pas le cas dans le langage C
  \item On appelle parfois ces variables des \emph{flags} (drapeaux).
  \item Quand on réunit plusieurs de ces variables dans une même entité, on
    appelle le résultat un champ de bits (anglais \emph{bit field}).
  \item Ces champs de bits peuvent être stockés dans une seule variable (selon
    leur nombre). On les considère comme un entier codé en NAT ou C2.
  \item On définit des opérations sur les champs de bits: le \emph{et
      bit-à-bit}, le \emph{ou bit-à-bit}, le \emph{not bit-à-bit} et le
    \emph{xor bit-à-bit}.
  \item[\dialogsystem] Il s'agit de faire sur les bits de même position dans
    deux champs de bits (un pour la négation) l'opération booléenne
    correspondante.
  \end{itemize}
\end{frame}
\begin{exercice}
  \begin{exercicelet}{Opérations booléennes}
    \begin{itemize}
    \item Que vaut 0b10000110 AND 0b11101001 ?
      \begin{correction}10000000\end{correction}
    \item Que vaut 0b10000110 OR 0b11101001 ?
      \begin{correction}11101111\end{correction}
    \item Que vaut 0b10000110 XOR 0b11101001 ?
      \begin{correction}01101111\end{correction}
    \item Que se passe-t-il si on calcule ($a$ est une variable booléenne):
      $a+0$ ? $a+1$ ? $a\times 0$ ? $a\times 1$ ? $a+a$ ? $a+a+a+a+a+a$ ?
    \item Démontrez que $a+ab=a$;
      \begin{correction}Vu en cours. $a+ab=a(1+b)=a$.\end{correction}
    \item Démontrez que $a+bc=(a+b)(a+c)$;
      \begin{correction}
        $(a+b)(a+c)=a+ab+ac+bc=a+ac+bc=a+bc$
      \end{correction}
    \item Démontrez que $a+\overline{a}b=a+b$;
      \begin{correction}Vu en
        cours. $a+\overline{a}b=a(b+\overline{b})+\overline{a}b=ab+a\overline{b}+
        \overline{a}b+ab=a(b+\overline{b})+b(a+\overline{a})=a+b$. NB:
        $x=x+x$.\end{correction}
    \end{itemize}
  \end{exercicelet}
\end{exercice}
\begin{frame}{Masquage}
  Lorsqu'un champ de bits est représenté par un entier, on peut accéder à un
  bit particulier en procédant à un ET:
  $$b_x=(B\&(1\mathop{<\!<}x))\mathop{>\!>}x$$
  On obtient 1 si le bit numéro $x$ est à 1, 0 sinon.
  
  On peut aussi mettre à 1 le bit numéro $x$:
  $$B=B|(1\mathop{<\!<}x)$$
  ou à zéro:
  $$B=B\&(\sim(1\mathop{<\!<}x))$$

  On peut aussi inverser le bit numéro $x$:
  $$B=B\oplus(1\mathop{<\!<}x)$$

  On peut tester si par exemple le bit 1 ou 3 sont à 1: \texttt{if (a\&0b1010
    != 0) ...}

  L'ensemble de ces techniques pour manipuler un champ de bits sous la forme
  d'un entier est appelé \emph{masquage}.
  
  Souvent, les valeurs $(1\mathop{<\!<}x)$ sont nommées pour qu'on puisse
  simplement utiliser leur nom au lieu de se souvenir de leur position.
\end{frame}
\begin{exercice}
  \begin{exercicelet}{Analyse d'un masquage}
    Dans un champ de bits qui contient $a=0b11001001$, on veut faire les
    choses suivantes:
    \begin{itemize}
    \item On veut vérifier si le bit 0 est actif ou non. Décomposez
      l'opération.
    \item On veut changer le bit 1 en 1 et le bit 3 en 0. Décomposez les
      opérations qui permettent de le faire.
    \item Changez le bit 5, en expliquant les valeurs intermédiaires.
    \end{itemize}
  \end{exercicelet}
  \begin{exercicelet}{Analyse de touches}
    Dans un système, la fonction \texttt{keyEvent()} renvoie une valeur
    entière sur 16 bits (dont 5 ignorés):
    \begin{itemize}
    \item Les 8 premiers bits correspondent au numéro de la touche sur le
      clavier (pour les touches ordinaires)
    \item Le 9\ieme{} bit correspond à la touche SHIFT (1: pressée, 0: pas
      pressée)
    \item Le 10\ieme{} bit correspond à la touche CONTROL (1: pressée, 0: pas
      pressée)
    \item Le 11\ieme{} bit correspond au fait d'appuyer sur une touche (1) ou
      de l'avoir juste relachée (0)
    \end{itemize}
    \begin{itemize}
    \item[\ddialoghome] Écrivez un programme qui appelle cette fonction
      (\texttt{a=keyEvent()}) puis qui en fonction de \texttt{a} affiche un
      texte du genre: « Vous venez de lâcher la touche 27 en ayant SHIFT
      appuyé et CONTROL lâché »
    \end{itemize}
  \end{exercicelet}
\end{exercice}
% Local Variables:
% TeX-master: "archi03"
% TeX-PDF-mode: t
% fill-column: 78
% coding: utf-8-unix
% mode-require-final-newline: t
% mode: latex
% mode: flyspell
% ispell-local-dictionary: "francais"
% End:
