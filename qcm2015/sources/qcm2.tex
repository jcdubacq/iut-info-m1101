\element{conversionbits}{
  \qp{12}
  \begin{questionmult}{conversion1}
    Un appareil a émis \unitfrac[100]{b}{ms} pendant \unit[80]{s}. À la fin, il a émis:
    \begin{rep}[3]
      \bonne{\unit[1]{Mo}}
      \mauvaise{\unit[64]{Mo}}
      \mauvaise{\unit[64]{Mb}}
      \bonne{\unit[8]{Mb}}
      \mauvaise{\unit[1]{ko}}
      \mauvaise{\unit[8]{kb}}
      \mauvaise{\unit[64]{ko}}
    \end{rep}
  \end{questionmult}
}

\element{conversionbits}{
  \qp{6}
  \begin{question}{bits1}
    De combien de bits ai-je besoin pour coder le résultat d'un dé à six faces ?
    \begin{rep}[4]
      \mauvaise{$2$}
      \bonne{$3$}
      \mauvaise{$2^6=64$}
      \mauvaise{$6$}
    \end{rep}
  \end{question}
}

\element{conversionbits}{
  \qp{6}
  \begin{question}{bits2}
    De combien de bits ai-je besoin pour coder le résultat d'un dé à vingt faces ?
    \begin{rep}[4]
      \mauvaise{$4$}
      \bonne{$5$}
      \mauvaise{$2^{20}$}
      \mauvaise{$20$}
    \end{rep}
  \end{question}
}

\element{conversionbits}{
  \qp{12}
  \begin{questionmult}{unites}
    Je convertis 2 MiB en kib ou en bits (2 bonnes réponses exactement):
    \bareme{mz=12}
    \begin{rep}[3]
      \bonne{$2^{24}=16777216$ bits}
      \bonne{$2^{14}=16384$ kib}
      \mauvaise{$2^{20}=1048576$ bits}
      \mauvaise{$2048$ kib}
      \mauvaise{$2000$ kib}
      \mauvaise{$2000000$ bits}
      \mauvaise{$16$ kib}
      \mauvaise{$16000$ bits}
      \mauvaise{$250$ kib}
      \mauvaise{$250000$ bits}
      \mauvaise{$2^{18}=262144$ bits}
      \mauvaise{$0,25$ kib}
    \end{rep}
  \end{questionmult}
}

\element{signaux-out}{
  \qp{12}
  \begin{questionmult}{periode1}
    Un signal périodique de fréquence maximale 1 kHz est échantillonné
    pour être discrétisé. Quelle sont les fréquences d'échantillonnage
    qui garantissent qu'il n'y aura aucune perte avant la quantification
    ?
    \begin{rep}[3]
      \bonne{4 kHz}
      \bonne{2 kHz}
      \mauvaise{1 kHz}
      \mauvaise{0,5 kHz}
      \mauvaise{Il y a toujours des pertes}
    \end{rep}
  \end{questionmult}
}

\element{signaux}{
  \qp{6}
  \begin{question}{periode2}
    Un signal périodique de fréquence maximale 1 kHz est échantillonné
    pour être discrétisé. Quelle est la fréquence la plus petite qui
    garantit qu'il n'y aura aucune perte avant la quantification ?
    \begin{rep}[3]
      \mauvaise{4 kHz}
      \bonne{2 kHz}
      \mauvaise{1 kHz}
      \mauvaise{0,5 kHz}
      \mauvaise{Il y a toujours des pertes}
    \end{rep}
  \end{question}
}

\element{signaux}{
  \begin{question}{quanta}
    On mesure la pression de l'air à 1000 Pa (pascals) près (on arrondit donc au millier la plus proche). L'instrument permet de mesurer de 80000 Pa à 124000 Pa. Quel est le nombre de quanta de cet instrument ?
    \begin{rep}[4]
      \bonne{45}
      \mauvaise{44}
      \mauvaise{44000}
      \mauvaise{45000}
      \mauvaise{124000}
      \mauvaise{124}
      \mauvaise{80}
      \mauvaise{80000}
    \end{rep}
  \end{question}
}

\element{boiteaoutils}{
  \qp{6}
  \begin{question}{pwd}
    Quelle commande permet d'afficher le chemin du répertoire
    courant ?
    \ttfamily
    \begin{rep}[3]
      \bonne{pwd}
      \mauvaise{cd}
      \mauvaise{cd\ \textasciitilde}
      \mauvaise{pwd\ \textasciitilde}
      \mauvaise{pwd .}
      \mauvaise{cd .}
    \end{rep}
  \end{question}
}

\element{linux1}{
  \qp{6}
  \begin{question}{chemin1}
    Est-ce que le chemin \texttt{/bin/ls} est un chemin... :
    \begin{rep}[4]
      \bonne{...absolu}
      \mauvaise{...relatif}
      \mauvaise{...personnel}
      \mauvaise{...d'escapade}
    \end{rep}
  \end{question}
}

\element{linux1}{
  \qp{6}
  \begin{question}{chemin2}
    Est-ce que le chemin \texttt{../toto.txt} est un chemin... :
    \begin{rep}[4]
      \mauvaise{...absolu}
      \bonne{...relatif}
      \mauvaise{...personnel}
      \mauvaise{...d'escapade}
    \end{rep}
  \end{question}
}
\element{linux1}{
  \qp{6}
  \begin{question}{chemin3}
    Est-ce que le chemin \texttt{README.md} est un chemin... :
    \begin{rep}[4]
      \mauvaise{...absolu}
      \bonne{...relatif}
      \mauvaise{...personnel}
      \mauvaise{...d'escapade}
    \end{rep}
  \end{question}
}
\element{boiteaoutils}{
  \qp{6}
  \begin{question}{mkdir}
    Quelle option de \texttt{mkdir} permet de créer tous les répertoires manquants intermédiaires?
    \begin{rep}[4]
      \mauvaise{\texttt{-rf}}
      \bonne{\texttt{-p}}
      \mauvaise{\texttt{-f}}
      \mauvaise{\texttt{-i}}
    \end{rep}
  \end{question}
}

\element{boiteaoutils}{
  \qp{24}
  \begin{questionmult}{commandes}
    Quel(s) nom(s) de commande est un nom de commande UNIX connue?
    \ttfamily
    \begin{rep}[4]
      \mauvaise{copy}
      \mauvaise{del}
      \mauvaise{tch}
      \mauvaise{remove}
      \mauvaise{move}
      \mauvaise{list}
      \bonne{cp}
      \bonne{touch}
      \bonne{rm}
      \bonne{ls}
      \bonne{rmdir}
      \bonne{mv}
    \end{rep}
  \end{questionmult}
}

