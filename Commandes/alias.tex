			% Syntaxe d'utilisation
			\synt{alias}{}{nom\_de\_la\_commande=expression}
			
			% Description de la commande
			\desc{\begin{itemize}
					\item créet un alias entre un nom de commande et une expression.
					\item l'expression est donnée entre \it{quotes}: \lin{'expression \dots'}
				\end{itemize}
				}
			
			% Exemple d'uitlisation
			\expl{%
				\begin{columns}
					\begin{column}{4.5cm}
						\dirtree{%
							.1 \DTd{chez\_moi}\DTfcomment{{\color{red}Répertoire Courant}}.
							.2 \DTd{public\_html}.
							.3 index.html.
							.2 astronomie.txt.
							}
					\end{column}
					\begin{column}{7.5cm}
						\mprompt{
							\prompt{ll}{-bash: ll: command not found}
							\prompt{alias ll='ls -l'}{}
							\prompt{ls -l}{%
								\begin{tabular}{@{ }r@{ }r@{ }r@{ }r@{ }r@{ }r@{ }r@{ }r@{ }r}
									\multicolumn{9}{l}{total 32}\\
									{\color{Green}drwxr-xr-x}&2&{\color{red}santini}&{\color{blue}ensinfo}&4096&20&jui&15:50&public\_html\\
									{\color{Green}-rw-r--r--}&1&{\color{red}santini}&{\color{blue}ensinfo}&25&20&jui&15:49&telluriques.txt
								\end{tabular}
							}
						}
					\end{column}
				\end{columns}
			}
