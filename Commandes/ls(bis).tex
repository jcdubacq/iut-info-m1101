			% Syntaxe d'utilisation
			\synt{ls}{-a}{<source>}
			
			% Description de la commande
			\desc{\begin{itemize}
					\item Affiche le contenu d'un répertoire y compris les fichiers et répertoires cachés.
					\item Les fichiers et répertoires cachés ont un nom dont le premier caractère est un point.
					\item Les fichiers et répertoires cachés sont utilisés par le système ou certaines applications.
				\end{itemize}
				}
			
			% Exemple d'uitlisation
			\expl{
				\begin{columns}
					\begin{column}{4cm}
						\dirtree{%
						.1 \DTd{chez\_moi}\DTfcomment{{\color{red}Rép. Courant}}.
						.2 \DTd{./ssh}.
						.3 id\_rsa.
						.3 id\_rsa.pub.
						.3 known\_hosts.
						.2 .bashrc.
						.2 astronomie.txt.
						.2 \DTd{Etoiles}.
						.3 soleil.jpg.
						}
					\end{column}
					\begin{column}{4cm}
						\begin{center}
							Sans option \it{-a}\\
							\mpromptT{
								\promptT{ls}{astronomie.txt\\Etoiles/}
								\promptT{\cursor}{}
							}
						\end{center}
					\end{column}
					\begin{column}{4cm}
						\begin{center}
							Avec option \it{-a}\\
							\mpromptT{
								\promptT{ls -a}{.\\..\\.ssh/\\.bashrc\\astronomie.txt\\Etoiles/}
								\promptT{\cursor}{}
							}
						\end{center}
					\end{column}
				\end{columns}
			}
