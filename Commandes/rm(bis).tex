% Syntaxe d'utilisation
\synt{rm}{-r}{chemin <chemin\_2 \dots>}

% Description de la commande
\desc{\begin{itemize}
  \item L'option \lin{-r} (comme \textbf{r}écursif) permet de supprimer un répertoire et tout son contenu.
  \item[\dialogwarning] L'option \lin{-f} (comme \textbf{f}orce) permet d'ignorer certaines questions.
  \end{itemize}
}

% Exemple d'uitlisation
\expl{
  \dirtree{%
    .1 \DTd{chez\_moi}\DTfcomment{{\color{red}Répertoire Courant}}.
    .2 astronomie.txt.
    .2 \DTd{{\color{solarizedGreen}Etoiles}}\DTfcomment{{\color{solarizedGreen}Supprimé par la commande \#1}}.
    .3 {\color{solarizedGreen}soleil.jpg}\DTfcomment{{\color{solarizedGreen}Supprimé par la commande \#1}}.
    .3 \DTd{{\color{solarizedGreen}Galaxie}}\DTfcomment{{\color{solarizedGreen}Supprimé par la commande \#1}}.
    .4 {\color{solarizedGreen}Andromede.pdf}\DTfcomment{{\color{solarizedGreen}Supprimé par la commande \#1}}.
    .2 aldebaran.gif.
  }
  \mprompt{
    \prompt{rm -r Etoiles}{}
  }
}
