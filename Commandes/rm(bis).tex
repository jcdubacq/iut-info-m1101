			% Syntaxe d'utilisation
			\synt{rm}{-r}{chemin <chemin\_2 \dots>}
			
			% Description de la commande
			\desc{\begin{itemize}
					\item L'option \lin{-r} (\textbf{R}écursif) permet de supprimer un répertoire et tout son contenu.
				\end{itemize}
				}
			
			% Exemple d'uitlisation
			\expl{
				\dirtree{%
				.1 \DTd{chez\_moi}\DTfcomment{{\color{red}Répertoire Courant}}.
				.2 astronomie.txt.
				.2 \DTd{{\color{solarizedGreen}Etoiles}}\DTfcomment{{\color{solarizedGreen}Supprimé par la Commande \#1}}.
				.3 {\color{solarizedGreen}soleil.jpg}\DTfcomment{{\color{solarizedGreen}Supprimé par la Commande \#1}}.
				.3 \DTd{{\color{solarizedGreen}Galaxie}}\DTfcomment{{\color{solarizedGreen}Supprimé par la Commande \#1}}.
				.4 {\color{solarizedGreen}Andromede.pdf}\DTfcomment{{\color{solarizedGreen}Supprimé par la Commande \#1}}.
				.2 aldebaran.gif.
				}
				Commande \#1:
				\mprompt{
					\prompt{rm -r Etoiles}{}
				}
			}
