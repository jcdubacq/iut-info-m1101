% Syntaxe d'utilisation
\synt{mkdir}{-p}{chemin <chemin\_2 \dots>}

% Description de la commande
\desc{\begin{itemize}
  \item Création d'un ou de plusieurs répertoires aux endroits spécifiés par les chemins.
  \item Si depuis la racine en suivant un chemin, on rencontre un fichier, il y a un message d'erreur.
  \item Si depuis la racine en suivant un chemin, il n'y pas de répertoire, il est créé.
  \end{itemize}
}

% Exemple d'uitlisation
\expl{
  \dirtree{%
    .1 \DTd{chez\_moi}\DTfcomment{{\color{red}Répertoire Courant}}.
    .2 astronomie.txt.
    .2 \DTd{Etoiles}.
    .2 \DTd{{\color{solarizedGreen}Galaxies}}\DTfcomment{{\color{solarizedGreen}Création Commande \#1}}.
    .3 \DTd{{\color{solarizedGreen}M91}}\DTfcomment{{\color{solarizedGreen}Création Commande \#1}}.
    .4 \DTd{{\color{solarizedGreen}highres}}\DTfcomment{{\color{solarizedGreen}Création Commande \#1}}.
  }
  \mprompt{
    \prompt{mkdir -p Galaxies/M91/highres}{}
  }
}
