% Exemple d'uitlisation
\expl[Déplacer un Répertoire]{
  \begin{columns}
    \begin{column}{6cm}
      État Initial de l'arborescence:\\
      \dirtree{%
        .1 \DTd{{\color{solarizedBlue}moi}}\DTfcomment{{\color{red}Répertoire courant}}.
        .2 {\color{red} astronomie.txt}\DTfcomment{{\color{red}Fichier source}}.
        .2 \DTd{{\color{solarizedBlue}Etoiles}}\DTfcomment{{\color{solarizedBlue}Répertoire cible}}.
      }
    \end{column}
    \begin{column}{6cm}
      État Final de l'arborescence:\\
      \dirtree{%
        .1 \DTd{{\color{solarizedBlue}moi}}\DTfcomment{{\color{red}Répertoire courant}}.
        .2 \DTd{{\color{solarizedBlue}Etoiles}}\DTfcomment{{\color{solarizedBlue}Répertoire cible}}.
        .3 {\color{solarizedGreen} astronomie.txt}\DTfcomment{{\color{solarizedGreen}Fichier déplacé}}.
      }
    \end{column}
  \end{columns}
  \begin{center}
    \mprompt{
      \prompt{mv astronomie.txt Etoiles}{}
    }
  \end{center}
}
% Exemple d'uitlisation
\expl[Renommer un répertoire]{
  \begin{columns}
    \begin{column}{6cm}
      État Initial de l'arborescence:\\
      \dirtree{%
        .1 \DTd{{\color{solarizedBlue}moi}}\DTfcomment{{\color{red}Répertoire courant}}.
        .2 \DTd{{\color{red}Etoiles}}\DTfcomment{{\color{red}Répertoire Source}}.
        .3 astronomie.txt.
      }
    \end{column}
    \begin{column}{6cm}
      État final de l'arborescence:\\
      \dirtree{%
        .1 \DTd{{\color{solarizedBlue}moi}}\DTfcomment{{\color{red}Répertoire courant}}.
        .2 \DTd{{\color{solarizedGreen}Relativite}}\DTfcomment{{\color{solarizedGreen}Répertoire Renommé}}.
        .3 astronomie.txt.
      }
    \end{column}
  \end{columns}
  \begin{center}
    \mprompt{
      \prompt{mv Etoiles Relativite}{}
    }
  \end{center}
}
