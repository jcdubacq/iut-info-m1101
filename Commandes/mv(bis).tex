			% Exemple d'uitlisation
			\expl[Déplacer un Répertoire]{
				\begin{columns}
					\begin{column}{6cm}
						État Initial de l'arborescence:\\
						\dirtree{%
							.1 \DTd{{\color{solarizedBlue}chez\_moi}}\DTfcomment{{\color{red}Répertoire Courant}}.
							.2 {\color{red} astronomie.txt}\DTfcomment{{\color{red}Fichier Source}}.
							.2 \DTd{{\color{solarizedBlue}Etoiles}}\DTfcomment{{\color{solarizedBlue}Répertoire Cible}}.
						}
					\end{column}
					\begin{column}{6cm}
						État Final de l'arborescence:\\
						\dirtree{%
							.1 \DTd{{\color{solarizedBlue}chez\_moi}}\DTfcomment{{\color{red}Répertoire Courant}}.
							.2 \DTd{{\color{solarizedBlue}Etoiles}}\DTfcomment{{\color{solarizedBlue}Répertoire Cible}}.
							.3 {\color{solarizedGreen} astronomie.txt}\DTfcomment{{\color{solarizedGreen}Fichier Déplacé}}.
						}
					\end{column}
				\end{columns}
				\begin{center}
					\mprompt{
						\prompt{mv astronomie.txt Etoiles}{}
					}
				\end{center}
			}
			% Exemple d'uitlisation
			\expl[Renommer un Répertoire]{
				\begin{columns}
					\begin{column}{6cm}
						État Initial de l'arborescence:\\
						\dirtree{%
							.1 \DTd{{\color{solarizedBlue}chez\_moi}}\DTfcomment{{\color{red}Répertoire Courant}}.
							.2 \DTd{{\color{red}Etoiles}}\DTfcomment{{\color{red}Répertoire Source}}.
							.3 astronomie.txt.
						}
					\end{column}
					\begin{column}{6cm}
						État Final de l'arborescence:\\
						\dirtree{%
							.1 \DTd{{\color{solarizedBlue}chez\_moi}}\DTfcomment{{\color{red}Répertoire Courant}}.
							.2 \DTd{{\color{solarizedGreen}Relativite}}\DTfcomment{{\color{solarizedGreen}Répertoire Renommé}}.
							.3 astronomie.txt.
						}
					\end{column}
				\end{columns}
				\begin{center}
					\mprompt{
						\prompt{mv Etoiles Relativite}{}
					}
				\end{center}
			}
