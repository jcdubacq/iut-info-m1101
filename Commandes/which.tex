% Syntaxe d'utilisation
\synt{which}{}{nom\_de\_la\_commande}

% Description de la commande
\desc{\begin{itemize}
  \item Affiche le chemin du fichier correspondant à une commande.
  \item Parcours successivement les répertoires de la variable \lin{\$PATH}. Dès qu'il trouve un fichier correspondant au nom de la commande il renvoie son chemin.
  \end{itemize}
}

% Exemple d'uitlisation
\expl{%
  \begin{columns}
    \begin{column}{6cm}
      \dirtree{%
        .1 \DTd{}\DTfcomment{{\color{solarizedRed}Répertoire Racine}}.
        .2 \DTd{bin}.
        .3 ls\DTfcomment{{\color{solarizedGreen}Exécutable \#1}}.
        .3 \dots.
        .2 \DTd{home}.
        .3 \DTd{chez\_moi}\DTfcomment{{\color{solarizedRed}Répertoire Courant}}.
        .4 \DTd{bin}.
        .5 ls\DTfcomment{{\color{solarizedGreen}Exécutable \#2}}.
      }
    \end{column}
    \begin{column}{6cm}
      \mpromptS{
        \promptS[/home/chez\_mo]{echo \$PATH}{/bin\NoAutoSpaceBeforeFDP:/usr/bin\NoAutoSpaceBeforeFDP:/usr/local/bin\NoAutoSpaceBeforeFDP:/home/chez\_moi/bin}
        \promptS[/home/chez\_moi]{which ls}{/bin/ls}
      }
    \end{column}
  \end{columns}
}
