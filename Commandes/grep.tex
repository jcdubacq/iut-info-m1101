			% Syntaxe d'utilisation
			\synt{grep}{}{"motif" fichier}
			
			% Description de la commande
			\desc{\begin{itemize}
					\item Affiche les lignes du fichier qui comportent le \lin{"motif"}.
					\item Les lignes sont affichées dans leur ordre d'apparition dans le fichier.
				\end{itemize}
				}
			
			% Exemple d'uitlisation
			\expl{
				\begin{columns}
					\begin{column}{4cm}
						Soit le fichier \lin{planetes.txt} contenant les lignes suivantes:\\
						\fileform[3cm]{planetes.txt}{%
						\# Premier groupe\\
						1\hspace{1em}Mercure\hspace{1em}Tellurique\\
						2\hspace{1em}Venus\hspace{2em}Tellurique\\
						3\hspace{1em}Terre\hspace{2em}Tellurique\\
						4\hspace{1em}Mars\hspace{2.5em}Tellurique\\
						\# Deuxième groupe\\
						1\hspace{1em}Jupiter\hspace{1em}Gazeuse\\
						2\hspace{1em}Saturne\hspace{1em}Gazeuse\\
						3\hspace{1em}Uranus\hspace{1.5em}Gazeuse\\
						4\hspace{1em}Neptune\hspace{1em}Gazeuse
						}
					\end{column}
					\begin{column}{7cm}
						Commandes:\\ \vspace{5pt}
						\mpromptM{
							\promptM{grep 'Tellurique' planetes.txt}{%
								1\hspace{1em}Mercure\hspace{1em}Tellurique\\
								2\hspace{1em}Venus\hspace{2em}Tellurique\\
								3\hspace{1em}Terre\hspace{2em}Tellurique\\
								4\hspace{1em}Mars\hspace{2.5em}Tellurique
							}
							\promptM{grep '1' planetes.txt}{%
								1\hspace{1em}Mercure\hspace{1em}Tellurique\\
								1\hspace{1em}Jupiter\hspace{1em}Gazeuse\
							}
							\promptM{\cursor}
						}
					\end{column}
				\end{columns}
			}
