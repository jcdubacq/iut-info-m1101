% Syntaxe d'utilisation
\synt{chmod}{}{droit fichier}

% Description de la commande
\desc{\begin{itemize}
  \item Modifie les droits et permissions accordés par le propriétaire aux différents utilisateurs du système.
  \end{itemize}
}

% Exemple d'utilisation
\expl{
  \begin{columns}
    \begin{column}{6cm}
      Retire au propriétaire le droit d'écriture sur le fichier \lin{cv\_2011.pdf}.\\
      \mpromptS{
        \promptS{chmod u-w cv\_2011.pdf}{}
      }
      Ajoute au propriétaire et aux membres de son groupe le droit d'exécution sur le fichier \lin{listing.bash}.\\
      \mpromptS{
        \promptS{chmod ug+x listing.bash}{}
      }
    \end{column}
    \begin{column}{6cm}
      Retire aux utilisateurs qui ne sont ni le propriétaire ni membre de son groupe les droits de lecture, d'écriture et d'exécution.\\
      \mpromptS{
        \promptS{chmod o-rwx listing.bash}{}
      }
      Ajoute à tous les utilisateurs, tous les droits.
      \mpromptS{
        \promptS{chmod a+rwx listing.bash}{}
      }
    \end{column}
  \end{columns}
}
