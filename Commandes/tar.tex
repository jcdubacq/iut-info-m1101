% Syntaxe d'utilisation
\synt{tar}{cvf}{nom\_archive fichier\_ou\_repertoire [autres\_sources]}

% Description de la commande
\desc{\begin{itemize}
  \item Crée un fichier archive dont le nom (chemin) est donné en premier argument et porte classiquement l'extension \lin{.tar}.
  \item Les fichiers sources qui servent a créer l'archive sont préservés par la commande \lin{tar}.
  \item L'option \lin{c} (comme \textbf{c}reate), indique que la commande \lin{tar} doit utiliser un algorithme d'archivage.
  \item L'option \lin{v} (\textbf{v}erbose), permet d'afficher le déroulement de l'archivage.
  \item L'option \lin{f} (\textbf{f}ile), permet de préciser juste derrière un fichier d'archivage.
  \end{itemize}
}

% Exemple d'utilisation
\expl{
  \dirtree{%
    .1 \DTd{moi}\DTfcomment{{\color{red}Répertoire courant}}.
    .3 astronomie.txt.
    .2 \DTd{Images}.
    .3 soleil2.jpg.
    .3 Terre1.jpg.
    .2 espace.tar\DTfcomment{{\color{solarizedGreen}Créé par la commande \#1}}.
  }
  Regroupe dans la même archive \lin{espace.tar} le fichier \lin{astronomie.txt} et le répertoire \lin{Images/} et son contenu:\\\vspace{5pt}
  \begin{center}
    \mprompt{
      \prompt{tar cvf espace.tar astronomies.txt Images}{astronomie.txt\\%
        Images/\\%
        Images/soleil2.jpg\\%
        Images/Terre1.jpg
      }
    }
  \end{center}
}
