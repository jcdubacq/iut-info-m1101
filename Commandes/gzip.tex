			% Syntaxe d'utilisation
			\synt{gzip}{}{fichier <fichier\_2 \dots>}
			
			% Description de la commande
			\desc{\begin{itemize}
					\item Compresse un ou plusieurs fichiers dont le nom est passé en paramètre.
					\item Le fichier source (initial non compressé) est supprimé et seul subsiste le fichier compressé.
					 \item Le fichier compressé produit porte le même nom que le fichier initial auquel l'extension .gz  a été ajoutée.
				\end{itemize}
				}
			
			% Exemple d'utilisation
			\expl{
				\begin{columns}
					\begin{column}{6cm}
						\dirtree{%
						.1 \DTd{chez\_moi}\DTfcomment{{\color{red}Répertoire Courant}}.
						.2 tellurique.tsv.
						.2 {\color{red}astronomie.txt}\DTfcomment{{\color{red}Avant gzip}}.
						}
					\end{column}
					\begin{column}{6cm}
						\dirtree{%
						.1 \DTd{chez\_moi}\DTfcomment{{\color{red}Répertoire Courant}}.
						.2 tellurique.tsv.
						.2 {\color{solarizedGreen} astronomie.txt.gz}\DTfcomment{{\color{solarizedGreen}Après gzip}}.
						}
					\end{column}
				\end{columns}
				\begin{center}
					\mprompt{
						\prompt{ls}{astronomie.txt telluriques.tsv}
						\prompt{gzip astronomie.txt}{}
						\prompt{ls}{astronomie.txt.gz telluriques.tsv}
					}
				\end{center}
			}
			
