% Syntaxe d'utilisation
\dsynt{tar}{tvf}{nom\_archive}{tar}{xvf}{nom\_archive [chemin(s) dans l'archive]}

% Description de la commande
\desc{\begin{itemize}
  \item Examine une archive ou crée des fichiers à partir de l'archive.
  \item Le fichier archive est préservé par la commande \lin{tar}.
  \item L'option \lin{x} (e\textbf{x}tract), permet de désarchiver.
  \item L'option \lin{t} (lis\textbf{t}), permet de lister le contenu d'une archive
  \end{itemize}
}

% Exemple d'utilisation
\expl{
  \begin{columns}
    \begin{column}{55mm}
      \dirtree{%
        .1 \DTd{moi}\DTfcomment{{\color{red}Répertoire courant}}.
        .2 espace.tar.
        .2 {\color{solarizedGreen}astronomie.txt}\DTfcomment{{\color{solarizedGreen}{créé par la commande \#2}}}.
        .2 {\color{solarizedGreen}Images}\DTfcomment{{\color{solarizedGreen}{créé par la commande \#3}}}.
        .3 {\color{solarizedGreen}soleil2.jpg}\DTfcomment{{\color{solarizedGreen}{créé par la commande \#3}}}.
        .3 {\color{solarizedGreen}Terre1.jpg}\DTfcomment{{\color{solarizedGreen}{créé par la commande \#3}}}.
      }
    \end{column}%
    \begin{column}{6cm}
      \mpromptS{
        \promptS{tar tvf espace.tar}{astronomie.txt\\%
          Images/\\%
          Images/soleil2.jpg\\%
          Images/Terre1.jpg
        }
        \promptS{tar xvf espace.tar astronomie.txt}{astronomie.txt}
        \promptS{tar xf espace.tar}{}
        \promptS{\cursor}{}
      }
    \end{column}
  \end{columns}
}
