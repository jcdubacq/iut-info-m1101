			% Syntaxe d'utilisation
			\synt{less}{}{fichier}
			
			% Description de la commande
			\desc{\begin{itemize}
					\item Affiche le contenu d'un fichier,
					\item Permet de naviguer en avant et en arrière dans le fichier.
					\item Permet d'effectuer des recherches de mot(if)s.
				\end{itemize}
				La commande ouvre une interface dans la fenêtre du terminal. Contrairement à la commande \lin{more}, on ne revient pas à la ligne de commande lorsqu'on atteint la fin du fichier, pour cela il faut quitter l'application.
				}
			
			% Exemple d'uitlisation
			\expl{
				Pour avoir une description complète des commandes de navigation dans l'interface de visualisation \lin{less}, reportez-vous aux pages de \lin{man}. Les commandes les plus utilisées sont:
				\begin{columns}
					\begin{column}{6cm}
						\begin{tabular}{rl}
							\hline
							Combinaison\\de touches&Action\\
							\hline
							\keystroke{H}&Affiche l'aide (abrégé des commandes)\\
							\keystroke{F}&Avancer d'une page (forward)\\
							\keystroke{B}&Reculer d'une page (backward)\\
							\keystroke{E}&Avancer d'une ligne\\
							\keystroke{Y}&Reculer d'une ligne\\
							\hline
						\end{tabular}
					\end{column}
					\begin{column}{6cm}
						\begin{tabular}{rl}
							\hline
							Combinaison\\de touches&Action\\
							\hline
							\keystroke{Q}&Quiter\\
							\keystroke{G}&Aller à la première ligne\\
							\Shift+\keystroke{G}&Aller à la dernière ligne\\
							\keystroke{num}+\keystroke{G}&Aller à la ligne numéro \lin{num}\\										&\\
							\hline
						\end{tabular}
					\end{column}
				\end{columns}
			}
