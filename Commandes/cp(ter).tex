			% Syntaxe d'utilisation
			\synt{cp}{-r}{source <source\_2 \dots> cible}
			
			% Description de la commande
			\desc{\begin{itemize}
					\item L'option \lin{-r} (\textbf{R}écursif) permet de copier un répertoire et son contenu si il apparait dans le(s) source(s).
				\end{itemize}
				}
			
			% Exemple d'uitlisation
			\expl{
				\dirtree{%
				.1 \DTd{chez\_moi}\DTfcomment{{\color{red}Répertoire Courant}}.
				.2 astronomie.txt.
				.2 \DTd{{\color{red}Galaxie}}\DTfcomment{{\color{red}Fichier Source Commande \#3}}.
				.3 Andromede.pdf.
				.2 \DTd{{\color{solarizedBlue}Etoiles}}\DTfcomment{{\color{solarizedBlue}Répertoire Cible \#3}}.
				.3 soleil.jpg.
				.3 \DTd{{\color{solarizedGreen}Galaxie}}\DTfcomment{{\color{solarizedGreen}Copié/Créé par la Commande \#3}}.
				.4 {\color{solarizedGreen}Andromede.pdf}\DTfcomment{{\color{solarizedGreen}Copié/Créé par la Commande \#3}}.
				.2 aldebaran.gif.
				}
				Commande \#3:
				\mprompt{
					\prompt{cp -r Galaxies Etoiles}{}
				}
			}
