% Syntaxe d'utilisation
\synt{cp}{-r}{source <source\_2 \dots> cible}

% Description de la commande
\desc{\begin{itemize}
  \item L'option \lin{-r} (\textbf{R}écursif) permet de copier un répertoire et son contenu si il apparait dans le(s) source(s).
  \item Attention : si le répertoire n'existe pas et qu'on copie un répertoire, il y a renommage
  \end{itemize}
}

% Exemple d'uitlisation
\expl{
  \dirtree{%
    .1 \DTd{moi}\DTfcomment{{\color{red}Répertoire Courant}}.
    .2 \DTd{{\color{red}Galaxie}}\DTfcomment{{\color{red}Source commandes}}.
    .3 Andromede.pdf.
    .2 \DTd{{\color{solarizedBlue}Etoiles}}\DTfcomment{{\color{solarizedBlue}Répertoire cible \#1}}.
    .3 \DTd{{\color{solarizedGreen}Galaxie}}\DTfcomment{{\color{solarizedGreen}Copié/créé par la commande \#1}}.
    .4 {\color{solarizedGreen}Andromede.pdf}\DTfcomment{{\color{solarizedGreen}Copié/créé par la commande \#1}}.
    .2 \DTd{{\color{solarizedGreen}Top10}}\DTfcomment{{\color{solarizedGreen}Copié/créé par la commande \#2 (renommage)}}.
    .3 {\color{solarizedGreen}Andromede.pdf}\DTfcomment{{\color{solarizedGreen}Copié/créé par la commande \#2}}.
  }
  \mprompt{
    \prompt{cp -r Galaxie Etoiles}{}
    \prompt{cp -r Galaxie Top10}{}
  }
}
