% Syntaxe d'utilisation
\synt{mv}{}{source cible}

% Description de la commande
\desc{Déplace/renomme un fichier ou répertoire.
  \begin{itemize}
  \item modifie le chemin d'accès à la \lin{source} qui devient le chemin \lin{cible}.
  \item Le chemin \lin{source} disparait et le chemin \lin{cible} est créé.
  \item Le fichier ou répertoire pointé reste le même.
  \item La cible doit être un chemin non occupé ou un répertoire.
  \end{itemize}}


% Exemple d'uitlisation
\expl[Renommer un fichier]{
  \begin{columns}
    \begin{column}{6cm}
      État Initial de l'arborescence:\\
      \dirtree{%
        .1 \DTd{{\color{solarizedBlue}moi}}\DTfcomment{{\color{red}Répertoire courant}}.
        .2 {\color{red} AstroNomIe.TXT}\DTfcomment{{\color{red}Fichier Source}}.
      }
    \end{column}
    \begin{column}{6cm}
      État Final de l'arborescence:\\
      \dirtree{%
        .1 \DTd{{\color{solarizedBlue}moi}}\DTfcomment{{\color{red}Répertoire courant}}.
        .2 {\color{solarizedGreen} astronomie.txt}\DTfcomment{{\color{solarizedGreen}Fichier Renommé}}.
      }
    \end{column}
  \end{columns}
  \begin{center}
    \mprompt{
      \prompt{mv AstroNomIe.TXT astronomie.txt}{}
    }
  \end{center}
}
