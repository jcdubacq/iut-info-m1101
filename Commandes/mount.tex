% Syntaxe d'utilisation
\synt{mount}{}{[[périphérique] point\_de\_montage]}

% Description de la commande
\desc{\begin{itemize}
  \item \lin{périphérique} correspond à périphérique
    (\lin{/dev/xxx}). Il y a plusieurs syntaxes possibles.
  \item \lin{point\_de\_montage} correspond à un nom de répertoire
    valide dans l'arborescence principale donnant accès au contenu de
    l'arborescence du périphérique.
  \item Sans argument, la commande liste tous les montages en cours
  \end{itemize}
}

% Exemple d'uitlisation
\expl{
  \mprompt{
    \prompt[/home]{mount /dev/sde1 /mnt/usb}{}	
  }
  \begin{columns}
    \begin{column}{6cm}
      \dirtree{%
        .1 \DTd{}\DTfcomment{{\color{red}Répertoire racine du périphérique \texttt{/dev/sde1}}}.
        .2 \DTd{photo}.
        .3 {\color{solarizedGreen}\dots}\DTfcomment{{\color{solarizedGreen}Contenu...}}.
        .2 {\color{solarizedGreen}CV.pdf}.
      }
      \hrule
      \dirtree{%
        .1 \DTd{}\DTfcomment{{\color{red}Répertoire Racine}}.
        .2 \DTd{mnt}.
        .3 \DTd{usb}\DTfcomment{{\color{solarizedBlue}Répertoire normal}}.
      }
    \end{column}$\rightarrow$
    \begin{column}{55mm}
      \dirtree{%
        .1 \DTd{}\DTfcomment{{\color{red}Répertoire racine}}.
        .2 \DTd{mnt}.
        .3 \DTd{usb}\DTfcomment{{\color{solarizedBlue}Point de montage}}.
        .4 \DTg{photo}\DTfcomment{{\color{solarizedGreen}Contenu du périphérique}}.
        .5 {\color{solarizedGreen}\dots}\DTfcomment{{\color{solarizedGreen}Contenu du périphérique}}.
        .4 {\color{solarizedGreen}CV.pdf}\DTfcomment{{\color{solarizedGreen}Contenu du périphérique}}.
      }
    \end{column}%
  \end{columns}
}
