			% Syntaxe d'utilisation
			\synt{mount}{}{périphérique  point\_de\_montage}
			
			% Description de la commande
			\desc{\begin{itemize}
					\item \lin{périphérique} correspond soit à un fichier de périphérique (\lin{/dev/xxx}), soit à l'adresse d'un disque (\lin{nom\_réseau\_du\_disque:répertoire\_du\_disque}).
					\item \lin{point\_de\_montage} correspond à un nom de répertoire valide dans l'arborescence principale donnant accès au contenu de l'arborescence du périphérique.
				\end{itemize}
				}
			
			% Exemple d'uitlisation
			\expl{
				
				\begin{center}
					\mpromptM{
								\promptM[/home]{mount /dev/sda1 /mnt/usb}{}	
					}
				\end{center}
				\dirtree{%
				.1 \DTd{}\DTfcomment{{\color{red}Répertoire Racine}}.
				.2 \DTd{mnt}.
				.3 \DTd{usb}\DTfcomment{{\color{Blue}Point de Montage}}.
				.4 \DTg{photo}\DTfcomment{{\color{Green}Contenu du périphérique}}.
				.5 {\color{Green}\dots}\DTfcomment{{\color{Green}Contenu du périphérique}}.
				.4 {\color{Green}CV.pdf}\DTfcomment{{\color{Green}Contenu du périphérique}}.
				.2 \DTd{home}\DTfcomment{{\color{red}Répertoire Courant}}.
				.3 \dots.
				}
			}
