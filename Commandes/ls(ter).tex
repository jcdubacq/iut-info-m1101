% Syntaxe d'utilisation
\synt{ls}{-l}{<source>}

% Description de la commande
\desc{\begin{itemize}
  \item Affiche le contenu d'un répertoire en format long.
  \item Le format long donne le nom du propriétaire et son groupe, ainsi que les droits des différentes classes d'utilisateurs sur les fichiers et répertoires.
  \end{itemize}
}

% Exemple d'uitlisation
\expl{
  \dirtree{%
    .1 \DTd{chez\_moi}\DTfcomment{{\color{solarizedRed}Répertoire Courant}}.
    .2 \DTd{public\_html}.
    .3 index.html.
    .3 astronomie.txt.
  }
  \begin{center}
    \mprompt{
      \prompt{ls -l}{%
        \begin{tabular}{@{ }r@{ }r@{ }r@{ }r@{ }r@{ }r@{ }r@{ }r@{ }l}
          \multicolumn{9}{l}{total 32}\\
          {\color{solarizedGreen}drwxr-xr-x}&2&{\color{solarizedRed}santini}&{\color{solarizedBlue}ensinfo}&4096&20&jui&15:50&public\_html\\%
          {\color{solarizedGreen}-rw-r-{}-r-{}-}&1&{\color{solarizedRed}santini}&{\color{solarizedBlue}ensinfo}&25&20&jui&15:49&astronomie.txt
        \end{tabular}
      }
    }
  \end{center}
  Ici, le nom de l'utilisateur est \lin{\textcolor{solarizedRed}{santini}}, nom du groupe est \lin{\textcolor{solarizedBlue}{ensinfo}} et les droits sont colorés en {\color{solarizedGreen}vert}.
}
