% Syntaxe d'utilisation
\synt{touch}{}{chemin [chemin\_2 \dots]}

% Description de la commande
\desc{\begin{itemize}
  \item Si le chemin est occupé par un fichier ou un répertoire, mise à jour de la date de dernière modification.
  \item Sinon, création d'un ou de plusieurs fichiers vides à l'endroit spécifié par le chemin.
  \end{itemize}
}

% Exemple d'uitlisation
\expl{
  \dirtree{%
    .1 \DTd{moi}\DTfcomment{{\color{red}Répertoire Courant}}.
    .2 astronomie.txt.
    .2 \color{solarizedGreen}{lisezmoi.txt}\DTfcomment{{\color{solarizedGreen}Création Commande \#1}}.
    .2 \DTd{Stars}.
    .3 \color{solarizedGreen}{TCeti.txt}\DTfcomment{{\color{solarizedGreen}Création Commande \#2}}.
    .3 \color{solarizedGreen}{ACentauri.txt}\DTfcomment{{\color{solarizedGreen}Création Commande \#2}}.
  }
  \mprompt{
    \prompt{touch lisezmoi.txt}{}
    \prompt{touch Stars/TCeti.txt Stars/ACentauri.txt}{}
  }
}
