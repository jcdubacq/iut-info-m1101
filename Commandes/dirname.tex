			% Syntaxe d'utilisation
			\synt{dirname}{chemin}{}
			
			% Description de la commande
			\desc{\begin{itemize}
					\item Ne conserve que la partie répertoire d'un chemin d'accès.
					\item Il n'est pas nécessaire que le chemin existe dans l'arborescence. Le chemin est traité comme une chaîne de caractères.
				\end{itemize}
				}
			
			% Exemple d'uitlisation
			\expl{
				\begin{center}
					\mprompt{
						\prompt{dirname Documents}{.}
						\prompt{dirname Documents/cv.txt}{Documents}
						\prompt{dirname Documents/Photos/}{Documents}
						\prompt{dirname Documents/Photos/Soleil.jpg}{Documents/Photos}
						\prompt{\cursor}{}
					}
				\end{center}
			}
