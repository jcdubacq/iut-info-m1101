			% Syntaxe d'utilisation
			\synt{cut}{-d 'sep' -f n}{fichier}
			
			% Description de la commande
			\desc{\begin{itemize}
					\item Affiche une colonne du fichier.
					\item L'option \lin{<-d 'sep'>} permet de changer le séparateur par défaut qui est la tabulation. Le séparateur est donné entre guillement simples. 
					\item L'option \lin{<-f n>} indique que la commande doit afficher la n$\up{ème}$ colonne.
				\end{itemize}
				}
			
			% Exemple d'uitlisation
			\expl{
				\begin{columns}
					\begin{column}{6cm}
						Cas\#1: les mots (les champs) sont séparés par des tabulations:\\
						\fileform[4cm]{tellur.tsv}{1\hspace{3em}Mercure\hspace{3em}Venus\\2\hspace{3em}Terre\hspace{4em}Mars}\\
						Commande \#1\\ \vspace{3pt}
						\mpromptS{
							\promptS{cut -f 2 tellur.tsv}{Mercure\\Terre}
							\promptS{\cursor}
						}
					\end{column}
					\begin{column}{6cm}
						Cas\#2: les mots (les champs) sont séparés par le caractère \lin{=}:\\
						\fileform[4cm]{jov.txt}{1=Jupiter=Saturne\\1=Uranus=Neptune}\\
						Commande \#2\\ \vspace{3pt}
						\mpromptS{
							\promptS{cut -d '=' -f 3 jov.txt}{Saturne\\Neptune}
							\promptS{\cursor}
						}
					\end{column}
				\end{columns}
			}
