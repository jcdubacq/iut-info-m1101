% Syntaxe d'utilisation
\synt{ln}{}{source [cible]}

% Description de la commande
\desc{\begin{itemize}
  \item Crée un lien dur entre la référence source et le chemin cible.
  \end{itemize}
}

% Exemple d'utilisation
\expl{
  \begin{center}
    \mprompt{
      \prompt{ln astronomie.txt Etoiles/lisezmoi.txt}{}
    }
  \end{center}
  Le lien sur un fichier crée une deuxième entrée pointant vers le même inode.
  \begin{columns}
    \begin{column}{6cm}
      \dirtree{%
        .1 \DTd{moi}\DTfcomment{{\color{red}Répertoire courant}}.
        .2 astronomie.txt\DTfcomment{{\color{red}Référence source}}.
        .2 \DTd{{\color{solarizedBlue}Etoiles}}.
        .3 soleil.jpg.
        .2 aldebaran.gif.
      }
    \end{column}
    \begin{column}{6cm}
      \dirtree{%
        .1 \DTd{moi}\DTfcomment{{\color{red}Répertoire courant}}.
        .2 astronomie.txt.
        .2 \DTd{Etoiles}.
        .3 soleil.jpg.
        .3 lisezmoi.txt\DTfcomment{{\color{solarizedGreen}Nouveau chemin}}.
        .2 aldebaran.gif.
      }
    \end{column}
  \end{columns}
}

