			% Syntaxe d'utilisation
			\synt{tail}{< -int >}{fichier}
			
			% Description de la commande
			\desc{\begin{itemize}
					\item Affiche par défaut les 10 dernières lignes d'un fichier.
					\item Si un entier \lin{n} précède le nom du fichier, la commande affiche les n dernières lignes du fichier.
				\end{itemize}
				}
			
			% Exemple d'uitlisation
			\expl{
				\begin{columns}
					\begin{column}{6cm}
						Soit le fichier \lin{planetes.txt} contenant les lignes suivantes:\\
						\fileform[3cm]{planetes.txt}{%
							\# Premier groupe\\
							1\hspace{1em}Mercure\hspace{1em}Tellurique\\
							2\hspace{1em}Venus\hspace{2em}Tellurique\\
							3\hspace{1em}Terre\hspace{2em}Tellurique\\
							4\hspace{1em}Mars\hspace{2.5em}Tellurique\\
							\# Deuxième groupe\\
							1\hspace{1em}Jupiter\hspace{1em}Gazeuse\\
							2\hspace{1em}Saturne\hspace{1em}Gazeuse\\
							3\hspace{1em}Uranus\hspace{1.5em}Gazeuse\\
							4\hspace{1em}Neptune\hspace{1em}Gazeuse
						}
					\end{column}
					\begin{column}{6cm}
						La commande suivante affiche les 4 dernières lignes du fichier:
						\begin{center}
							\mpromptS{
								\promptS{tail -4 planetes.txt}{%
							1\hspace{1em}Jupiter\hspace{1em}Gazeuse\\
							2\hspace{1em}Saturne\hspace{1em}Gazeuse\\
							3\hspace{1em}Uranus\hspace{1.5em}Gazeuse\\
							4\hspace{1em}Neptune\hspace{1em}Gazeuse
					}
								\promptS{\cursor}{}
							}
						\end{center}
					\end{column}
				\end{columns}
			}
