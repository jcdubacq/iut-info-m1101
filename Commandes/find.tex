			% Syntaxe d'utilisation
			\synt{find}{}{depart -iname "motif"}
			
			% Description de la commande
			\desc{\begin{itemize}
					\item Recherche dans les répertoires et sous-répertoires les fichiers dont le nom correspond au motif en partant du point de l'arborescence spécifié par le \lin{depart}.
					\item L'option \lin{-iname} indique que le motif sera recherché sans tenir compte des majuscules et minuscules.
				\end{itemize}
				}
			
			% Exemple d'utilisation
			\expl{
				\begin{columns}
					\begin{column}{4cm}
						\dirtree{%
							.1 \DTd{chez\_moi}\DTcomment{{\color{red}Répertoire courant}}.
							.2 \DTd{Mes\_Images}.
							.3 \DTd{JPG}.
							.4 soleil.jpg.
							.4 lune.jpg.
							.3 \DTd{GIF}.
							.4 alphacentauri.gif.
							.4 terre.gif.
							.2 \DTd{Mes\_Photos}.
							.3 etacentauri.jpg.
							.3 phobos.gif.
						}
					\end{column}
					\begin{column}{8cm}
						\mpromptM{%
							\promptM{find . -iname *.gif}{./Mes\_Images/GIF/alphacentauri.gif\\./Mes\_Images/GIF/terre.gif\\./Mes\_Photos/phobos.gif}
							\promptM{find . -iname *centauri*}{./Mes\_Images/GIF/alphacentauri.gif\\./Mes\_Photos/etacentauri.jpg}

							\promptM{find Mes\_Images/ -iname *e.*}{Mes\_Images/GIF/terre.gif\\Mes\_Images/JPG/lune.jpg}
							\promptM{\cursor}{}
						}
					\end{column}
				\end{columns}
			}
