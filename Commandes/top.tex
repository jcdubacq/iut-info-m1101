			% Syntaxe d'utilisation
			\synt{top}{}{}
			
			% Description de la commande
			\desc{\begin{itemize}
					\item Permet de suivre dynamiquement (temps réel) les ressources matériel utilisées par chaque processus.
					\item Ouvre un interface dans la ligne de commande qui peut être quittée en pressant la touche \keystroke{Q}
					\item Donne pour chaque processus en autres choses, le PID, le nom du propriétaire, la date de lancement du processus, les \%CPU et \%MEM utilisés.
				\end{itemize}
				}
			
			% Exemple d'uitlisation
			\expl{
				\begin{center}
				\mprompt{%
				\lin{\tiny{
					\begin{tabular}{@{}r@{ }r@{ }r@{ }r@{ }r@{ }r@{ }r@{ }r@{ }r@{ }r@{ }r@{ }r@{ }}
\multicolumn{12}{l}{Tasks:  85 total,   1 running,  84 sleeping,   0 stopped,   0 zombie}\\
\multicolumn{12}{l}{Cpu(s):  5.7\%us,  0.0\%sy,  0.0\%ni, 93.6\%id,  0.0\%wa,  0.7\%hi,  0.0\%si,  0.0\%st}\\
\multicolumn{12}{l}{Mem:    772068k total,   231864k used,   540204k free,     2412k buffers}\\
\multicolumn{12}{l}{Swap:   995992k total,        0k used,   995992k free,   161316k cached}\\
\multicolumn{12}{l}{}\\
PID&USER&PR&NI&VIRT&RES&SHR&S&\%CPU&\%MEM&TIME+&COMMAND\\
5116&root&20&0&33832&22m&6576&S&5.7&3.0&0:19.49&X\\
5879&santini&20&0&16060&7344&6116&S&0.3&1.0&0:01.06&xfce4-netload-p\\
1&root&20&0&1664&568&496&S&0.0&0.1&0:02.95&init\\
2&root&20&0&0&0&0&S&0.0&0.0&0:00.00&kthreadd\\
3&root&RT&0&0&0&0&S&0.0&0.0&0:00.00&migration/0
					\end{tabular}
					}}
				}
				\end{center}
			}
