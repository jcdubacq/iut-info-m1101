			% Syntaxe d'utilisation
			\synt{cd}{}{<cible>}
			
			% Description de la commande
			\desc{\begin{itemize}
					\item Change le répertoire courant (permet de naviguer dans l'arborescence).
					\item Si le chemin du répertoire cible est omit, le répertoire courant redevient par défaut le répertoire personnel.
				\end{itemize}
				}
			
			% Exemple d'uitlisation
			\expl{
				\dirtree{%
				.1 \DTd{}\DTfcomment{{\color{red}Répertoire Racine}}.
				.2 \DTd{home}\DTfcomment{{\color{red}Répertoire Courant Initial}}.
				.3 \DTd{chez\_moi}\DTfcomment{{\color{red}Répertoire Courant Final \#1}}.
				.4 astronomie.txt.
				.4 \DTd{Etoiles}\DTfcomment{{\color{red}Répertoire Courant Final \#2}}.
				}
				\begin{columns}
					\begin{column}{5cm}
						Commande \#1:
						\mpromptS{
							\promptS[/home]{cd}{}
							\promptS{\cursor}
						}
					\end{column}
					\begin{column}{5cm}
						Commande \#2:
						\mpromptS{
							\promptS[/home]{cd chez\_moi/Etoile}{}
							\promptS[\~{}/Etoile]{\cursor}
						}
					\end{column}
				\end{columns}
			}
