% Syntaxe d'utilisation
\synt{cd}{}{<cible>}

% Description de la commande
\desc{\begin{itemize}
  \item Change le répertoire courant (permet de naviguer dans
    l'arborescence).
  \item Si le chemin du répertoire cible est omis, le répertoire courant
    redevient par défaut le \emph{répertoire personnel}.
  \item[\dialogwarning] Ce n'est pas une commande, mais une
    fonctionnalité du shell.
  \end{itemize}
}

% Exemple d'uitlisation
\expl{
  \dirtree{%
    .1 \DTd{}\DTfcomment{{\color{red}Répertoire Racine}}.
    .2 \DTd{home}\DTfcomment{{\color{red}Répertoire courant initial}}.
    .3 \DTd{moi}\DTfcomment{{\color{red}Répertoire courant cas \#1}}.
    .4 \DTd{Etoiles}\DTfcomment{{\color{red}Répertoire courant cas \#2}}.
  }
  \begin{columns}
    \begin{column}{5cm}
      Commande cas \#1:
      \mpromptS{
        \promptS[/home]{cd}{}
        \promptS{\cursor}
      }
    \end{column}
    \begin{column}{5cm}
      Commande cas \#2:
      \mpromptS{
        \promptS[/home]{cd moi/Etoiles}{}
        \promptS[\~{}/Etoile]{\cursor}
      }
    \end{column}
  \end{columns}
}
