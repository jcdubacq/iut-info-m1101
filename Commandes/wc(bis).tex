			% Syntaxe d'utilisation
			\synt{wc}{-l}{fichier <fichier\_2 \dots>}
			
			% Description de la commande
			\desc{\begin{itemize}
					\item L'option \lin{-l} indique que l'on affiche que le nombre de lignes.
				\end{itemize}
				}
			
			% Exemple d'utilisation
			\expl{
				\begin{columns}
					\begin{column}{6cm}
						Soit le fichier suivant:\\
						\begin{center}
							\fileform[4cm]{tellur.tsv}{1 Mercure Venus\\2 Terre Mars}
						\end{center}
					\end{column}
					\begin{column}{6cm}
						Commande \#1:\\
						\begin{center}
							\mpromptS{
							\promptS{wc -l tellur.tsv}{\hspace{3em}2\hspace{3em}tellur.tsv}
							\promptS{\cursor}{}
							}
						\end{center}
					\end{column}
				\end{columns}
				L'affichage produit indique que le fichier \lin{tellur.tsv} comporte:
				\begin{itemize}
					\item 2 lignes.
				\end{itemize}
			}
