% Syntaxe d'utilisation
\synt{cp}{}{source cible}

% Description de la commande
\desc{\begin{itemize}
  \item Copie le fichier source vers la cible.
  \item La source doit être un fichier ordinaire (pas un répertoire),
  \item Si la source est un répertoire la commande produit un message d'erreur.
  \item Si la cible :
    \begin{itemize}
    \item est le chemin d'un répertoire existant, le fichier sera copié dans ce répertoire et conservera son nom,
    \item ne correspond pas à un répertoire existant, le fichier sera copié avec le nom (chemin) \lin{cible}.
    \end{itemize}
  \end{itemize}
}

% Exemple d'uitlisation
\expl{
  \dirtree{%
    .1 \DTd{moi}\DTfcomment{{\color{red}Répertoire courant}}.
    .2 {\color{red}astronomie.txt}\DTfcomment{{\color{red}Fichier source commandes \#1 et \#2}}.
    .2 \DTd{Etoiles}\DTfcomment{{\color{solarizedBlue}Répertoire cible commande \#1}}.
    .3 {\color{solarizedGreen}astronomie.txt}\DTfcomment{{\color{solarizedGreen}Copié/Créé par la commande \#1}}.
    .3 {\color{solarizedGreen}info.txt}\DTfcomment{{\color{solarizedGreen}copié/créé par la commande \#2}}.
    .2 cv.pdf.
  }
  \mprompt{
    \prompt{cp astronomie.txt Etoiles}{}
    \prompt{cp astronomie.txt Etoiles/info.txt}{}
  }
}
