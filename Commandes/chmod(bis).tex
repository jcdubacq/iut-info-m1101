% Description de la commande
\desc{ Il existe plusieurs notations des droits.
  \begin{itemize}
  \item La notation alphanumérique:\lin{(ugoa) (+/-) (rwx)}
  \item La notation octale
    \begin{itemize}
    \item Calcul des droits pour un utilisateur (u, g ou o):
      \begin{tabular}{|c|c|c|c|c|c|c|c|c|}
        \hline
        Droit&\lin{-{}-{}-}&\lin{-{}-x}&\lin{-w-}&\lin{-wx}&\lin{r-{}-}&\lin{r-x}&\lin{rw-}&\lin{rwx}\\
        \hline
        Binaire&000&001&010&011&100&101&110&111\\
        \hline
        Octale&0&1&2&3&4&5&6&7\\
        \hline
      \end{tabular}
    \item Exemple de notation octale des droits d'un fichier
      \begin{tabular}{|c|c|c|c|c|c|c|c|c|c|}
        \hline
        &\multicolumn{3}{|c|}{\textbf{U}ser}&\multicolumn{3}{|c|}{\textbf{G}roup}&\multicolumn{3}{|c|}{\textbf{O}ther}\\
        \hline
        Alphabétique&\lin{r}&\lin{w}&\lin{x}&\lin{r}&\lin{-}&\lin{x}&\lin{-}&\lin{-}&\lin{x}\\
        \hline
        Binaire&1&1&1&1&0&1&0&0&1\\
        \hline
        Octale&\multicolumn{3}{c|}{7}&\multicolumn{3}{c|}{5}&\multicolumn{3}{c|}{1}\\
        \hline
      \end{tabular}\\\vspace{10pt}
    \end{itemize}
  \end{itemize}
}
% Exemple d'utilisation
\expl{
  \begin{columns}
    \begin{column}{3cm}
      \begin{tabular}{ll}
        Alph.&Oct.\\
        \hline
        \lin{-{}-{}- -{}-{}- -{}-{}-}&000\\
        \lin{rw- -{}-{}- -{}-{}-}&600\\
        \lin{rw- r-{}- r-{}-}&644\\
        \lin{rw- rw- rw-}&666
      \end{tabular}
    \end{column}
    \begin{column}{3cm}
      \begin{tabular}{ll}
        Alph.&Oct.\\
        \hline
        \lin{rwx -{}-{}- -{}-{}-}&700\\
        \lin{rwx r-x r-x}&755\\
        \lin{rwx rwx rwx}&777\\
      \end{tabular}
    \end{column}
    \begin{column}{6cm}
      \mpromptS{
        \promptS{chmod 700 dir\_parano}{}
        \promptS{chmod 644 fichier\_pub}{}
      }	
    \end{column}
  \end{columns}
}

