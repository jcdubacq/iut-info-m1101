			% Syntaxe d'utilisation
			\synt{wc}{}{fichier <fichier\_2 \dots>}
			
			% Description de la commande
			\desc{\begin{itemize}
					\item Affiche des statistiques sur le nombre de lignes, de mots et de caractères (comptés en nombre d'octets) contenus dans le fichier dont le chemin est donné en paramètre.
				\end{itemize}
				}
			
			% Exemple d'utilisation
			\expl{
				\begin{columns}
					\begin{column}{6cm}
						Soit le fichier suivant:\\
						\begin{center}
							\fileform[4cm]{tellur.tsv}{1 Mercure Venus\\2 Terre Mars}
						\end{center}
					\end{column}
					\begin{column}{6cm}
						Commande \#1:\\
						\begin{center}
							\mpromptS{
							\promptS{wc tellur.tsv}{\hspace{3em}2\hspace{3em}6\hspace{3em}29\hspace{1em}tellur.tsv}
							\promptS{\cursor}{}
							}
						\end{center}
					\end{column}
				\end{columns}
				L'affichage produit indique que le fichier \lin{tellur.tsv} comporte:
				\begin{itemize}
					\item 2 lignes,
					\item 6 mots et
					\item 29 caractères. La taille du fichier texte est donc de 29 octets \dots
				\end{itemize}
			}

