% Syntaxe d'utilisation
\synt{ln}{-s}{source cible}

% Description de la commande
\desc{\begin{itemize}
  \item Crée un lien symbolique entre la référence source et le chemin cible.
  \end{itemize}
}

% Exemple d'utilisation
\expl{
  \begin{center}
    \mprompt{
      \prompt{ln -s Galaxies Etoiles/Galaxies}{}
    }
  \end{center}
  Le lien symbolique sur un répertoire donne également accès à toutes les références contenues dans le répertoire pointé par le lien. Ainsi, le fichier \lin{\~{}/Galaxie/Andromede.pdf} est aussi accessible par le chemin \lin{\~{}/Etoiles/Galaxie/Andromede.pdf}.
  \begin{columns}
    \begin{column}{6cm}
      \dirtree{%
        .1 \DTd{chez\_moi}\DTfcomment{{\color{red}Répertoire Courant}}.
        .2 astronomie.txt.
        .2 \DTd{{\color{red}Galaxie}}\DTfcomment{{\color{red}Référence Source}}.
        .3 Andromede.pdf.
        .2 \DTd{{\color{solarizedBlue}Etoiles}}\DTfcomment{{\color{solarizedBlue}Répertoire Cible}}.
        .3 soleil.jpg.
        .2 aldebaran.gif.
      }
    \end{column}
    \begin{column}{6cm}
      \dirtree{%
        .1 \DTd{chez\_moi}\DTfcomment{{\color{red}Répertoire Courant}}.
        .2 astronomie.txt.
        .2 \DTd{{\color{red}Galaxie}}\DTfcomment{{\color{red}Référence Source}}.
        .3 Andromede.pdf.
        .2 \DTd{Etoiles}.
        .3 soleil.jpg.
        .3 \DTd{{\color{solarizedGreen}Galaxie}}\DTfcomment{{\color{solarizedGreen}Nouveau chemin}}.
        .4 {\color{solarizedGreen}Andromede.pdf}\DTfcomment{{\color{solarizedGreen}Nouveau chemin}}.
        .2 aldebaran.gif.
      }
    \end{column}
  \end{columns}
}
