% Syntaxe d'utilisation
\synt{mkdir}{}{chemin [chemin\_2 \dots]}

% Description de la commande
\desc{\begin{itemize}
  \item Création d'un ou de plusieurs répertoires aux endroits spécifiés par les chemins.
  \item Si le chemin est occupé par un fichier ou un répertoire, il y a un message d'erreur.
  \item Si le chemin n'est pas déjà créé à part le dernier élément, il y a un message d'erreur.
  \end{itemize}
}

% Exemple d'uitlisation
\expl{
  \dirtree{%
    .1 \DTd{moi}\DTfcomment{{\color{red}Répertoire courant}}.
    .2 \DTd{{\color{solarizedGreen}Systeme\_Solaire}}\DTfcomment{{\color{solarizedGreen}Création commande \#1}}.
    .2 \DTd{Etoiles}.
    .3 \DTd{{\color{solarizedGreen}Rouges}}\DTfcomment{{\color{solarizedGreen}Création commande \#2}}.
    .3 \DTd{{\color{solarizedGreen}Bleues}}\DTfcomment{{\color{solarizedGreen}Création commande \#2}}.
  }
  \mprompt{
    \prompt{mkdir Systeme\_Solaire}{}
    \prompt{mkdir Etoiles/Rouges Etoiles/Bleues}{}
    \prompt{mkdir Galaxies/M91}{mkdir: impossible de créer le répertoire « Galaxies/M91 »: Aucun fichier ou dossier de ce type}
  }
}
