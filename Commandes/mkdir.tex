			% Syntaxe d'utilisation
			\synt{mkdir}{}{chemin <chemin\_2 \dots>}
			
			% Description de la commande
			\desc{\begin{itemize}
					\item Création d'un ou de plusieurs répertoires aux endroits spécifiés par les chemins.
					\item Si le chemin est occupé par un fichier ou un répertoire, il y a un message d'erreur.
				\end{itemize}
				}
			
			% Exemple d'uitlisation
			\expl{
				\dirtree{%
				.1 \DTd{chez\_moi}\DTfcomment{{\color{red}Répertoire Courant}}.
				.2 astronomie.txt.
				.2 \DTd{{\color{solarizedGreen}Systeme\_Solaire}}\DTfcomment{{\color{solarizedGreen}Création Commande \#1}}.
				.2 \DTd{Etoiles}.
				.3 \DTd{{\color{solarizedGreen}Rouges}}\DTfcomment{{\color{solarizedGreen}Création Commande \#2}}.
				.3 \DTd{{\color{solarizedGreen}Bleues}}\DTfcomment{{\color{solarizedGreen}Création Commande \#3}}.
				.2 \DTd{{\color{solarizedGreen}Galaxies}}\DTfcomment{{\color{solarizedGreen}Création Commande \#3}}.
				}
				\begin{columns}
					\begin{column}{5cm}
						Commande \#1:
						\mpromptS{
							\promptS{mkdir Systeme\_Solaire}{}
						}
					\end{column}
					\begin{column}{5cm}
						Commande \#2:
						\mpromptS{
							\promptS{mkdir Etoiles/Rouges}{}
						}
					\end{column}
				\end{columns}
						Commande \#3:
						\mprompt{
							\prompt{mkdir Galaxies Etoiles/Bleues}{}
						}
			}
