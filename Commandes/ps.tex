			% Syntaxe d'utilisation
			\synt{ps}{<-eu>}{}
			
			% Description de la commande
			\desc{\begin{itemize}
					\item Affiche les processus en cours d'exécution.
					\item L'option \lin{<-e>} indique que tous les processus doivent être affichés,
					\item L'option \lin{<-u>} restreint l'affichage aux processus de l'utilisateur.
				\end{itemize}
				}
			
			% Exemple d'uitlisation
			\expl{
				\begin{center}
				\mprompt{
					\prompt{ps -eu}{%
					\begin{tabular}{@{}r@{ }r@{ }r@{ }r@{ }r@{ }r@{ }r@{ }r@{ }r@{ }r@{ }l@{ }}
						\multicolumn{11}{l}{Warning: bad ps syntax, perhaps a bogus '-'? See http://procps.sf.net}\\
USER&PID&\%CPU&\%MEM&VSZ&RSS&TTY&STAT&START&TIME&COMMAND\\
santini&5905&0.0&0.2&4824&1656&pts/1&Ss&09:27&0:00&-bash LC\_ALL=fr\_FR.UTF\\
santini&5962&0.0&0.1&3884&896&pts/1&R+&09:48&0:00&ps -eu MANPATH=/etc/jav
					\end{tabular}
					}
					\prompt{\cursor}{}
				}
				\end{center}
			}
