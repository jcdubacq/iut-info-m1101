			% Syntaxe d'utilisation
			\synt{sed}{}{'s/motif/new/g' fichier}
			
			% Description de la commande
			\desc{
				La commande \lin{sed} est une commande qui permet de faire de nombreuses opérations. Nous ne verrons ici que la syntaxe permettant de substituer un motif dans un texte.
				\begin{itemize}
					\item Affiche le contenu du fichier après avoir remplacé les occurrences du \lin{motif} par \lin{new}.
			\end{itemize}
				}
			
			% Exemple d'uitlisation
			\expl{
				\begin{columns}
					\begin{column}{4cm}
						Soit le fichier:\\
						\fileform[3cm]{dialogue.txt}{- C'est par ici!!!\\-Où ça, "ici" ?}
					\end{column}
					\begin{column}{7cm}
						\mpromptM{
							\promptM{sed 's/ici/là/' dialogue.txt}{- C'est par là!!!\\-Où ça, "là" ?}
							\promptM{\cursor}{}
						}
					\end{column}
				\end{columns}
			}
