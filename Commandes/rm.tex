% Syntaxe d'utilisation
\synt{rm}{}{chemin [chemin\_2 \dots]}

% Description de la commande
\desc{\begin{itemize}
  \item La commande supprime le fichier pointé par le(s) chemin(s).
  \item Si le chemin pointe sur un répertoire, la commande affiche un message d'erreur.
  \end{itemize}
}

% Exemple d'uitlisation
\expl{
  \dirtree{%
    .1 \DTd{moi}\DTfcomment{{\color{red}Répertoire Courant}}.
    .2 {\color{solarizedGreen}astronomie.txt}\DTfcomment{{\color{solarizedGreen}Supprimé par la commande \#1}}.
    .2 \DTd{Etoiles}.
    .3 {\color{solarizedGreen}soleil.jpg}\DTfcomment{{\color{solarizedGreen}Supprimé par la commande \#2}}.
    .2 {\color{solarizedGreen}aldebaran.gif}\DTfcomment{{\color{solarizedGreen}Supprimé par la commande \#2}}.
  }
  \mprompt{
    \prompt{rm astronomie.txt}{}
    \prompt{rm aldebaran.gif Etoiles/soleil.jpg}{}
  }
}
