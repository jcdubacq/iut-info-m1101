			% Syntaxe d'utilisation
			\synt{rm}{}{chemin <chemin\_2 \dots>}
			
			% Description de la commande
			\desc{\begin{itemize}
					\item La commande supprime le fichier pointé par le(s) chemin(s).
					\item Si le chemin pointe sur un répertoire, la commande affiche un message d'erreur.
				\end{itemize}
				}
			
			% Exemple d'uitlisation
			\expl{
				\dirtree{%
				.1 \DTd{chez\_moi}\DTfcomment{{\color{red}Répertoire Courant}}.
				.2 {\color{solarizedGreen}astronomie.txt}\DTfcomment{{\color{solarizedGreen}Supprimé par la Commande \#1}}.
				.2 \DTd{Etoiles}.
				.3 {\color{solarizedGreen}soleil.jpg}\DTfcomment{{\color{solarizedGreen}Supprimé par la Commande \#2}}.
				.2 {\color{solarizedGreen}aldebaran.gif}\DTfcomment{{\color{solarizedGreen}Supprimé par la Commande \#2}}.
				}
				Commande \#1:
				\mprompt{
					\prompt{rm astronomie.txt}{}
				}\\
				Commande \#2:
				\mprompt{
					\prompt{rm aldebaran.gif Etoiles/soleil.jpg}{}
				}
			}
