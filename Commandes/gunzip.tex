			% Syntaxe d'utilisation
			\synt{gunzip}{}{fichier <fichier\_2 \dots>}
			
			% Description de la commande
			\desc{\begin{itemize}
					\item Décompresse un ou plusieurs fichiers (au format gzip) dont le nom est passé en paramètre.
					\item Le fichier source (compressé) est supprimé et seul subsiste le fichier décompressé.
					 \item Le fichier décompressé qui apparaît porte le même nom que le fichier initial auquel l'extension .gz a été supprimée.
				\end{itemize}
				}
			
			% Exemple d'utilisation
			\expl{
				\begin{columns}
					\begin{column}{6cm}
						\dirtree{%
						.1 \DTd{chez\_moi}\DTfcomment{{\color{red}Répertoire Courant}}.
						.2 tellurique.tsv.
						.2 {\color{solarizedGreen} astronomie.txt.gz}\DTfcomment{{\color{solarizedGreen}Avant gunzip}}.
						}
					\end{column}
					\begin{column}{6cm}
						\dirtree{%
						.1 \DTd{chez\_moi}\DTfcomment{{\color{red}Répertoire Courant}}.
						.2 tellurique.tsv.
						.2 {\color{red}astronomie.txt}\DTfcomment{{\color{red}Après gunzip}}.
						}
					\end{column}
				\end{columns}
				\begin{center}
					\mprompt{
						\prompt{ls}{astronomie.txt.gz telluriques.tsv}
						\prompt{gunzip astronomie.txt.gz}{}
						\prompt{ls}{astronomie.txt telluriques.tsv}
					}
				\end{center}
			}
