			% Syntaxe d'utilisation
			\synt{wget}{chemin}{}
			
			% Description de la commande
			\desc{\begin{itemize}
					\item Client HTTP, HTTPS et FTP .
					\item Permet de récupérer du contenu d'un serveur serveur Web ou FTP (télécharger).
				\end{itemize}
				}
			
			% Exemple d'uitlisation
			\expl{
				\begin{center}
					\mprompt{
						\prompt{wget http://www-lipn.univ-paris13.fr/\~{}santini/intro\_syste\\me/2011\_2012\_S1D\_cours\_1.pdf .}{%
						Résolution de www-lipn.univ-paris13.fr... 10.10.0.68\\
Connexion vers www-lipn.univ-paris13.fr|10.10.0.68|:80... connecté.\\
requête HTTP transmise, en attente de la réponse... 200 OK\\
Longueur: 4568618 (4,4M) [application/pdf]\\
Sauvegarde en : «2011\_2012\_S1D\_cours\_1.pdf»\\
\\
100\%[======================================>] 4 568 618   10,4M/s   ds 0,4s\\
\\
2012-01-02 16:02:59 (10,4 MB/s) - «2011\_2012\_S1D\_cours\_1.pdf» sauvegardé [4568618/4568618]
}
						\prompt{ls -l ./2011\_2012\_S1D\_cours\_1.pdf}{-rw-r--r-- 1 santini users 4,4M 2011-12-14 10:33 ./2011\_2012\_S1D\_cours\_1.pdf
}
					}
				\end{center}
			}
