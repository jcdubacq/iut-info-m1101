\documentclass[a4paper]{iutvexam}
% compile-latex options:--jobname controle20170317
% compile-latex options:--jobname correction20170317
\usepackage{textcomp}
\usepackage{dirtree}
\usepackage{listings}
\usepackage{caption}
\newcommand{\DTd}[1]{\textbf{#1/}}
\newcommand{\DTfcomment}[1]{\DTcomment{\emph{#1}}}

\title{Contrôle 1}
\date{8/10/2013}
\begin{document}
\conditions{ Vous disposez de 60 minutes pour faire ce contrôle. Aucun
  document autorisé. Toute tentative de communication avec un voisin ou
  l'extérieur peut être sanctionnée. Toutes les réponses doivent être
  faites sur l'énoncé. La taille de la réponse attendue dépend de la
  taille allouée pour répondre. }
\begin{questions}
  \titledquestion{Codage}
  \begin{parts}
    \part[1] Quel est le nombre \textbf{minimal} de bits nécessaire
    pour coder l'information suivante: un code de bière décomposé en un numéro de brasserie irlandaise (il y en a 4096), une année de production (il n'y a que les 16 dernières années qui comptent), une chaîne de production dans la brasserie (le maximum est de 64 sortes de bière différentes pour une même brasserie) ?
    \begin{solutionordottedlines}[.25in]
      
    \end{solutionordottedlines}
    \part[\half] Quel est le plus grand entre $48\times 10^6$
    bits et $6$ Mio ? Prouvez-le
    \begin{solutionordottedlines}[.5in]
      $48\times10^6 bits < 6Mio$ car $48\times10^6 bits = 6\times10^6 octets = 6Mo$ or $6Mo < 6Mio$ car \\
      $1Mo=10^6=(1000)^2$ et $1Mio=2^{20}=(2^10)^2=1024^2$ et
      $1000<1024$ (éventuellement moins de détails)
    \end{solutionordottedlines}
    \part[1\half] Convertissez en hexadécimal les 3 nombres suivants:
    38, 194, 6000.
    \begin{solutionordottedlines}[.75in]
      0x26, 0xC2, 0x1770
    \end{solutionordottedlines}
    \part[2] Convertissez en C2 sur 8 bits le nombre $88$ et le nombre $-88$.
    \begin{solutionordottedlines}[.25in]
      0xC3 ou $1100\,0011$
    \end{solutionordottedlines}
    \part[1] Convertissez en décimal le nombre codé en C2 écrit en hexadécimal sur 12 bits 0xFCC.
    \begin{solutionordottedlines}[.25in]
      111111001100 --- 000000110011 --- abs(x+1)=51 --- x=-52.
    \end{solutionordottedlines}
    \part[1] Faites l'opération suivante:
    $0b1101\,1011+0b1111\,0101+1$. Donnez le résultat en binaire puis en
    décimal.
    \begin{solutionordottedlines}[.25in]
      $0b1\,1101\,0001=465_{10}$
    \end{solutionordottedlines}
  \end{parts}
  \titledquestion{Flottants courts} On veut modifier la norme IEEE
  754. On définit une nouvelle catégorie de nombres flottants sur 8
  bits:
  \begin{itemize}
  \item 1 bit de signe;
  \item 3 bits pour noter $E$ avec $E=e+3$;
  \item 4 bits pour la partie fractionnaire $M$ de la valeur $v$.
  \end{itemize}
  On a $x=(-1)^s\times v\times 2^e$. On fait les mêmes
  \textbf{exceptions} que dans la norme IEEE 754: pour
  $E=\texttt{0b000}$, on a $x=0$ et pour $E=\texttt{0b111}$, on a
  $x=\pm\infty$.
  \begin{parts}
    \part[1] Codez en \emph{flottant court} le nombre 3,5.
    \begin{solutionordottedlines}[.5in]%
      $3,5=11,1=1,11\times2$ donc $0100 0110 0000$ soit 0x460.
    \end{solutionordottedlines}
    \part[1] Quel est le plus grand nombre \textbf{non infini} que l'on peut
    représenter? Donnez son \emph{codage} sous forme hexadécimale, et sa
    valeur en décimal.
    \begin{solutionordottedlines}[.5in]%
      Codage $01101111=0x77F$ (1/2), $E=0b110=6$, $e=3$ et
      $v=0b1,1111$. Soit $x=0b1111\,1=15,5$ (1/2). Notes partielles
      si essentiellement bon.
    \end{solutionordottedlines}
    \part[1] Soit deux nombres $a$ et $b$, flottants courts, qui ont le
    même exposant $e$. Quelle est la \textbf{plus petite différence} possible non
    nulle entre ces deux nombres ?
    \begin{solutionordottedlines}[.5in]%
      La différence de $v$ vaut 0.0000001 donc différence totale
      de $2^e\times2^{-7}$ donc $2^{e-7}$.
    \end{solutionordottedlines}
    \part[1] Transformez en décimal les deux nombres \emph{flottants
      courts} codés par les chaînes de bits suivantes:
    $\texttt{0b0011\,1100}$ et
    $\texttt{0b0110\,0100}$.
    \begin{solutionordottedlines}[.5in]%
      1,75 et 10,0
    \end{solutionordottedlines}
    \part[2] Faites la conversion du nombre 7,25 en \emph{flottant court}. Est-ce qu'on peut convertir 7,125 facilement en \emph{flottant court} ? Pourquoi ?
    \begin{solutionordottedlines}[1in]%
    $\texttt{0b0101\,1101}$. On ne peut pas convertir 7,125 exactement.
    \end{solutionordottedlines}
  \end{parts}

  \titledquestion{Programme étrange}%

  \begin{lstlisting}[numbers=left,language=C,caption=Programme]
    fonction(unsigned int a) {
      unsigned int b=0;
      while (a!=0) {
        b=b+(a&1);
        a=a>>1;
      }
      return(b);
    }
  \end{lstlisting}

  Les entiers sont sur 32 bits dans cette machine. L'opération
  \verb|>>n| représente le décalage à droite, qui déplace chaque bit
  dans celui qui est $n$ colonnes plus à droite (le dernier est perdu, les valeurs non spécifiés sont remplacées par 0).

  \begin{parts}
    \part[\half] Donnez en binaire le résultat de l'opération \verb|18>>2| (sur 8 bits).
    \begin{solutionordottedlines}[.25in]
      Le décalage à droite n'introduit que des zéros dans le nombre a,
      donc a finit par être nul (division entière par 2).
    \end{solutionordottedlines}
    \part[\half] Pourquoi ce programme s'arrête-t-il ? Expliquez-bien.
    \begin{solutionordottedlines}[.5in]
      Le décalage à droite n'introduit que des zéros dans le nombre a,
      donc a finit par être nul (division entière par 2).
    \end{solutionordottedlines}
    \part[1] Que fait la ligne 5 si $a$ est impair ? Et si $a$ est pair ? Conclure.
    \begin{solutionordottedlines}[.75in]
    \end{solutionordottedlines}
    \part[1] On appelle la fonction avec la valeur 21. Détaillez les
    valeurs successives de \texttt{a} et \texttt{b} jusqu'à l'arrêt de
    la fonction.

    \begin{minipage}{.3\linewidth}
      \begin{solutionordottedlines}[1in]
        21,0 (facultatif)\\10,1
      \end{solutionordottedlines}
    \end{minipage}\hfill
    \begin{minipage}{.3\linewidth}
      \begin{solutionordottedlines}[1in]
        5,1\\2,2
      \end{solutionordottedlines}
    \end{minipage}\hfill
    \begin{minipage}{.3\linewidth}
      \begin{solutionordottedlines}[1in]
        1,2\\0,3
      \end{solutionordottedlines}
    \end{minipage}
    \part[1] Que calcule cette fonction ?
    \begin{solutionordottedlines}[.25in]
      Le nombre de bits à 1 dans le codage binaire de l'entier a.
    \end{solutionordottedlines}
  \end{parts}
\end{questions}
\end{document}
