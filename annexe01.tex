  \subsection{Le jeu du fakir}
  \begin{frame}[label=fakira]
    \frametitle{Le jeu du fakir (1)}
    Est-ce que le nombre choisi est impair ?
  \end{frame}
  \begin{frame}[label=fakirb]
    \frametitle{Le jeu du fakir (2)}
    Est-ce que le nombre choisi fait partie de ceux-ci:
    \begin{tabular}{|c|c|c|c|c|c|c|c|c|c|}\hline
      2&3&6&7&10&11&14&15\\\hline
      18&19&22&23&26&27&30&31\\\hline
      34&35&38&39&42&43&46&47\\\hline
      50&51&54&55&58&59&62&63\\\hline
      66&67&70&71&74&75&78&79\\\hline
      82&83&86&87&90&91&94&95\\\hline
      98&99&&&&&&\\\hline
    \end{tabular}
  \end{frame}
  \begin{frame}[label=fakirc]
    \frametitle{Le jeu du fakir (3)}
    Est-ce que le nombre choisi fait partie de ceux-ci:
    \begin{tabular}{|c|c|c|c|c|c|c|c|c|c|}\hline
      4&5&6&7&12&13&14&15\\\hline
      20&21&22&23&28&29&30&31\\\hline
      36&37&38&39&44&45&46&47\\\hline
      52&53&54&55&60&61&62&63\\\hline
      68&69&70&71&76&77&78&79\\\hline
      84&85&86&87&92&93&94&95\\\hline
      100&&&&&&&\\\hline
    \end{tabular}
  \end{frame}
  \begin{frame}[label=fakird]
    \frametitle{Le jeu du fakir (4)}
    Est-ce que le nombre choisi fait partie de ceux-ci:
    \begin{tabular}{|c|c|c|c|c|c|c|c|c|c|}\hline
      8&9&10&11&12&13&14&15\\\hline
      24&25&26&27&28&29&30&31\\\hline
      40&41&42&43&44&45&46&47\\\hline
      56&57&58&59&60&61&62&63\\\hline
      72&73&74&75&76&77&78&79\\\hline
      88&89&90&91&92&93&94&95\\\hline
    \end{tabular}
  \end{frame}
  \begin{frame}[label=fakire]
    \frametitle{Le jeu du fakir (5)}
    Est-ce que le nombre choisi fait partie de ceux-ci:
    \begin{tabular}{|c|c|c|c|c|c|c|c|c|c|}\hline
      16&17&18&19&20&21&22&23\\\hline
      24&25&26&27&28&29&30&31\\\hline
      48&49&50&51&52&53&54&55\\\hline
      56&57&58&59&60&61&62&63\\\hline
      80&81&82&83&84&85&86&87\\\hline
      88&89&90&91&92&93&94&95\\\hline
    \end{tabular}
  \end{frame}
  \begin{frame}[label=fakirf]
    \frametitle{Le jeu du fakir (6)}
    Est-ce que le nombre choisi fait partie de ceux-ci:
    \begin{tabular}{|c|c|c|c|c|c|c|c|c|c|}\hline
      32&33&34&35&36&37&38&39\\\hline
      40&41&42&43&44&45&46&47\\\hline
      48&49&50&51&52&53&54&55\\\hline
      56&57&58&59&60&61&62&63\\\hline
      96&97&98&99&100&&&\\\hline
    \end{tabular}
  \end{frame}
  \begin{frame}[label=fakirg]
    \frametitle{Le jeu du fakir (7)}
    Est-ce que le nombre choisi est strictement plus grand que 63 ?\pause

    {\large Le nombre est...}
    \hyperlink{mesure}{\beamergotobutton{Retour}}
  \end{frame}
\end{presentationonly}


\end{document}

% Local Variables:
% TeX-master: "archi01"
% TeX-PDF-mode: t
% fill-column: 78
% coding: utf-8-unix
% mode-require-final-newline: t
% mode: latex
% mode: flyspell
% ispell-local-dictionary: "francais"
% End:
