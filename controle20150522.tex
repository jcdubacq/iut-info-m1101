\documentclass[a4paper]{iutvexam}
% compile-latex options:--jobname controle20150522
% compile-latex options:--jobname correction20150522
\usepackage{textcomp}
\usepackage{dirtree}
\usepackage{listings}
\usepackage{caption}
\newcommand{\DTd}[1]{\textbf{#1/}}
\newcommand{\DTfcomment}[1]{\DTcomment{\emph{#1}}}

\title{Contrôle 1}
\date{22/5/2015}
\begin{document}
\conditions{ Vous disposez de 2 heures pour faire ce contrôle. Aucun
  document autorisé. Toute tentative de communication avec un voisin ou
  l'extérieur peut être sanctionnée. Toutes les réponses doivent être
  faites sur l'énoncé. La taille de la réponse attendue dépend de la
  taille allouée pour répondre. }
\begin{questions}
  \titledquestion{Système}

  Lorsqu'une commande vous est demandée, \textbf{donnez un exemple
    d'utilisation} qui répond à la question.
  \begin{parts}
    \part[\half] Quelle est la commande qui donne un texte d'aide sur
    une commande Linux?
    \begin{solutionordottedlines}[.25in]%
      \texttt{man man}
    \end{solutionordottedlines}
    \part[\half] Quelle est la commande qui permet de récupérer les dernières lignes d'un fichier ou de l'entrée standard?
    \begin{solutionordottedlines}[.25in]%
      \texttt{tail -n 20}
    \end{solutionordottedlines}
    \part[\half] Quelle est la commande qui permet de filter par un motif l'entrée (ou un fichier) ?
    \begin{solutionordottedlines}[.25in]%
      \texttt{grep motif}
    \end{solutionordottedlines}
    \part[\half] Quelle est la commande qui permet d'obtenir toutes les informations d'identité (identifiants numériques, groupes, etc.) de l'utilisateur courant ?
    \begin{solutionordottedlines}[.25in]%
      \texttt{id}
    \end{solutionordottedlines}
    \part[1] Quelle sont les deux commandes qui permettent (avec ou sans options) d'obtenir toutes les informations à propos d'un fichier (droits, taille, dates importantes) ?
    \begin{solutionordottedlines}[.25in]%
      \texttt{ls} et \texttt{stat}
    \end{solutionordottedlines}
    \part[\half] Quelle est la commande qui permet de trier les lignes d'un fichier ou de l'entrée standard par ordre alphabétique ?
    \begin{solutionordottedlines}[.25in]%
      \texttt{sort}
    \end{solutionordottedlines}
    \part[1] Quelle est l'interprétation du symbole spécial \verb|~|
    dans le shell ? Donnez \textbf{deux} sens (légèrement) différents de
    ce symbole.
    \begin{solutionordottedlines}[.5in]%
      On peut désigner son propre répertoire (symbole seul) ou celui de
      quelqu'un d'autre.
    \end{solutionordottedlines}
    \part[1] Par quelle ligne de code commence un script shell écrit en \texttt{bash} ?
    \begin{solutionordottedlines}[.25in]%
      \verb|#!/bin/bash| (accepter aussi avec sh).
    \end{solutionordottedlines}
    \part[1] Citez au moins deux ressources gérées par le système
    d'exploitation.
    \begin{solutionordottedlines}[.5in]%
      Le système d'exploitation arbitre au moins l'accès au processeur
      et l'accès à la mémoire. Les périphériques aussi, mais pas la ROM
      ou le GRUB. .25 par ressource.
    \end{solutionordottedlines}
    \part[\half] Quelle est la commande qui permet de visualiser une liste
    détaillée des processus existants sur un système Linux ?
    \begin{solutionordottedlines}[.25in]%
      \texttt{ps -ef} (\texttt{ps -eu} est dans le cours et est accepté
      aussi). J'ai aussi accepté \texttt{top}.
    \end{solutionordottedlines}
  \end{parts}

  \titledquestion{Système 2}
  \begin{parts}
    \part[2] Syntaxe de variables : on exécute la ligne suivante. Que s'affiche t'il ?\\
    \verb|A=2;B="$A++";C='X $A';D=$(echo coucou);echo "$A $B $C $D"| %$
    \begin{solutionordottedlines}[.5in]%
      \verb|2 2++ X $A coucou|
    \end{solutionordottedlines}
    \part[1] Est-ce que deux processus peuvent être exécutés réellement
    en même temps sur un processeur simple ? Expliquez
    \begin{solutionordottedlines}[1in]%
      Le processeur étant partagé, pas vraiment; il y a exécution concurrente avec priorités des 2 processus; Alternance rapide; Illusion de simultanéité
    \end{solutionordottedlines}
    \part[1] Quelles sont les 4 étapes du cycle du processeur ?
    \begin{solutionordottedlines}[.25in]%
      Voir cours
    \end{solutionordottedlines}
    \part[1] Comment le shell trouve-t-il où est une commande qui n'est
    pas un chemin absolu ou relatif (par exemple \texttt{cat fichier},
    où est situé \texttt{cat} ?)
    \begin{solutionordottedlines}[1in]%
      Il utilise la variable \texttt{PATH} qui est une liste de
      répertoires séparés par des deux-points. Il cherche dans ces
      répertoires si un fichier du nom de la commande existe et s'arrête
      au premier trouvé.
    \end{solutionordottedlines}
  \end{parts}

  \titledquestion{Arborescence}

  \framebox{\begin{minipage}{.5\linewidth}%
      \dirtree{%
        .1 \DTd{}.  .2 \DTd{bin}.  .3 (\dots)\DTfcomment{(1)}.  .2
        \DTd{home}.  .3 \DTd{jvaljean}\DTfcomment{Votre répertoire
          personnel}.  .4 \DTd{Documents}\DTfcomment{Vous êtes ici}.  .5
        \DTd{Paris}. .6 greve.jpg.  .5
        condamnation.pdf.  .3 \DTd{gavroche}.  .4 avis.txt.  .2
        (\dots)\DTfcomment{(2)}.  }
    \end{minipage}}\hfill%
  \begin{minipage}{.45\linewidth}
    Voilà ci-contre un extrait de l'arborescence de votre système. Il y
    a d'autres utilisateurs : \textit{\texttt{gavroche}} est dans votre
    groupe, \textit{\texttt{thenardier}} et \textit{\texttt{cosette}} ne
    le sont pas. Les \emph{fichiers} sont tous lisibles par tous initialement
  \end{minipage}
  \begin{parts}
    \part[3] Comment être sûr que seul vous (\texttt{jvaljean}) puissiez
    lire le document \texttt{condamnation.pdf}, tout en laissant la
    possibilité à \textit{\texttt{gavroche}} de regarder les photos que
    vous avez mis de l'ancien Paris dans le répertoire \texttt{Paris}
    (il connaît le chemin), et que ni \textit{\texttt{cosette}} ni
    \textit{\texttt{thenardier}} ne puissent même savoir qu'un document
    \texttt{condamnation.pdf} existe.
    \begin{solutionordottedlines}[1in]%
    \end{solutionordottedlines}
    \part[1] Quels droits doit donner \textit{\texttt{gavroche}} sur
    le fichier de son répertoire pour que \textit{\texttt{cosette}}
    puisse le lire, mais pas le modifier et que
    \textit{\texttt{jvaljean}} puisse y ajouter des choses ?
    \begin{solutionordottedlines}[.25in]%
    \end{solutionordottedlines}
  \end{parts}
  \newpage
  \titledquestion{Codage}
  \begin{parts}
    \part[1] Quel est le nombre \textbf{minimal} de bits nécessaire
    pour coder l'information suivante: le nom d'un roi du Danemark de la
    maison d'Oldenbourg (de Christian I\ier à Christian VIII, de
    Frédéric I\ier à Fréderic VII, et Jean).
    \begin{solutionordottedlines}[.25in]
      16 possibilités, donc 4 bits
    \end{solutionordottedlines}
    \part[1] Quel est le plus grand entre $48\times 10^6$
    bits et $6$ Mio ? Prouvez-le
    \begin{solutionordottedlines}[.5in]
      $48\times10^6 bits < 6Mio$ car $48\times10^6 bits = 6\times10^6 octets = 6Mo$ or $6Mo < 6Mio$ car \\
      $1Mo=10^6=(1000)^2$ et $1Mio=2^{20}=(2^10)^2=1024^2$ et
      $1000<1024$ (éventuellement moins de détails)
    \end{solutionordottedlines}
    \part[2] Convertissez en hexadécimal les 4 nombres suivants:
    16, 48, 226, 2251.
    \begin{solutionordottedlines}[.75in]
      0x10, 0x30, 0xE2, 0x8CB
    \end{solutionordottedlines}
    \part[2] Convertissez en C2 sur 8 bits le nombre $-59$.
    \begin{solutionordottedlines}[.25in]
      0xC5 ou $1100\,0101$\\
    \end{solutionordottedlines}
    \part[2] Faites l'opération suivante:
    $0b1101\,1011+0b1111\,0101+1$. Donnez le résultat en binaire puis en
    décimal.
    \begin{solutionordottedlines}[.25in]
      $0b1\,1101\,0001=465_{10}$
    \end{solutionordottedlines}
    \part[2] Voici le récapitulatif du tableau de conversion UTF8:
    \begin{center}\renewcommand{\tabcolsep}{1mm}
      \begin{tabular}{|c|c|c|c|}\hline
        Valeurs & Écriture binaire & Codage UTF-8 (binaire) & octets\\\hline
        0x0--0x7F & abc\,defg & 0abc\,defg & 1 \\
        0x80--0x7FF & abc\,defg\,hijk & 110a\,bcde~10fg\,hijk & 2 \\
        0x800--0xFFFF & abcd\,efgh\,ijkl\,mnop & 1110\,abcd~10ef\,ghij~10kl\,mnop & 3 \\
        0x10000--0x1FFFFF& a\,bcde\,fghi\,jklm\,nopq\,rstu & 1111\,0abc~10de\,fghi~10jk\,lmno~10pq\,rstu &	4 \\\hline
      \end{tabular}
    \end{center}
    Donnez la séquence d'octets correspondant au texte «~1€~», dont les
    valeurs unicodes sont 0x49 et 0x20AC.
    \begin{solutionordottedlines}[.25in]
      0x49, puis 0xE2 0x82 0xAC.\\
      $0x49\in [0x0;0x7F]$ $\rightarrow$ codage sur 1 octet. Le code ascii est le code UTF-8 solution:$0x49$\\
      $0x20AC\in [0x800;0xFFFF]$ $\rightarrow$ codage sur 3 octets selon le motif $111\,abcd\,\,10ef\,ghij\,\,10kl\,mnop$ avec $abcd\,efgh\,ijkl\,mnop=0x20AC=0011\,0000\,1010\,1100$ soit: solution: $1110\,0010\,\,1000\,0010\,\,1010\,1100=0xE282AC$
    \end{solutionordottedlines}
  \end{parts}

  \titledquestion{Programme étrange}%
  Considérez le programme suivant:

  \begin{lstlisting}[language=C,caption=Programme]
    int main() {
      char a=1;
      while (a!=0) {
        a=a+1;
      }
    }
  \end{lstlisting}

  \begin{parts}
    \part[3] Est-ce que ce programme s'arrête ? Expliquez-bien.
    \begin{solutionordottedlines}[.75in]
    \end{solutionordottedlines}
    \part[1] Combien de fois la boucle est-elle exécutée ?
    \begin{solutionordottedlines}[.25in]
      256
    \end{solutionordottedlines}
  \end{parts}
\end{questions}
\end{document}
