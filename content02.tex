\section{Représenter un nombre}
\subsection{Les systèmes de numération}
\begin{frame}{Représenter les nombres}
  \begin{itemize}
  \item Objet (abstrait) qui admet de nombreuses représentations.
  \item L'idée de quantité et une représentation visuelle précèdent sans
    doute l'écriture \emph{(unaire)}.
  \item Un jeu de règles de représentation des nombres sous forme de
    signes écrits est un système de numération.
  \end{itemize}
  \begin{example}[Plusieurs représentations]
    On représente aussi les nombres sur d'autres \emph{supports} que
    l'écrit: représentations par sons, par objets (nombre de bougies sur
    un gâteau). Cela ne change pas le nombre (information), 27 bougies
    représentent bien 27 éléments (années écoulées, ici) autant que
    «~2~» collé à «~7~», ou que %
    \begin{tikzpicture}[fill=solarizedRebase0,draw=solarizedRebase0,line width=.07,scale=.16]
      \draw[fill] (1.322,0)--(.188,.850)--(1.322,1.700)--(.566,1.133)--(.566,.566)--cycle;
      \draw[fill] (1.700,0)--(.566,.850)--(1.700,1.700)--(.944,1.133)--(.944,.566)--cycle;
      \begin{scope}[rotate=180,shift={(-6,-2.6)}]
        \draw[fill] (2.551,.944)--(2.834,1.228)--(3.118,.944)--cycle;
        \draw[fill] (1.984,.944)--(2.267,1.228)--(2.551,.944)--cycle;
        \draw[fill] (3.118,.944)--(3.401,1.228)--(3.685,.944)--cycle;
        \draw[fill] (1.984,1.228)--(2.267,1.511)--(2.551,1.228)--cycle;
        \draw[fill] (2.551,1.228)--(2.834,1.511)--(3.118,1.228)--cycle;
        \draw[fill] (3.118,1.228)--(3.401,1.511)--(3.685,1.228)--cycle;
        \draw[fill] (2.551,1.511)--(2.834,1.795)--(2.834,2.645)--(2.834,2.645)--(2.834,1.795)--(3.118,1.511)--cycle;
      \end{scope}
    \end{tikzpicture}
    (numération babylonienne) ou XXVII (numération romaine).
  \end{example}
\end{frame}
\begin{frame}{La numération positionnelle}
  \begin{definition}[Système de numération positionelle]
    Un ensemble fini de symboles $\mathcal{B}$ (appelés chiffres)
    auxquels est associé une valeur entière de $0$ à $B-1$, où $B$ est
    le nombre d'éléments de $\mathcal{B}$. $B$ est la \emph{base}.
    
    La valeur d'une suite finie de $k$ chiffres
    $\alpha_{k-1}\alpha_{k-2}...\alpha_{0}$ est la somme:
    $$\alpha_{k-1}B^{k-1}+\dots+\alpha_{1}B+\alpha_{0}=
    \sum_{i=0}^{k-1}\alpha_{i}B^i.$$
  \end{definition}

  \begin{itemize}
  \item Le mot chiffre vient de l'arabe \transtrue \<'a.s-.sifr> et
    désignait le zéro.
  \item Exemple en base 5: le nombre $132_5$ vaut $1\times
    5^2+3\times5^1+2\times 5^0$, soit $25+15+2=42_{10}$.
  \item $B^i$ est le \emph{poids} du $i$-ième\ chiffre (en comptant de 0
    à droite).
  \end{itemize}
\end{frame}
\begin{frame}{Les autres systèmes de numération}{un peu de culture générale...}
  \begin{itemize}
  \item Systèmes de numération additifs (chiffres grecs, égyptiens):
    $\cap\cap|||||||$, par exemple. Chaque poids est représenté par un
    symbole distinct, la position n'est pas importante. À un détail
    près, les chiffres romains le sont aussi.
  \item Systèmes hybrides (numérotation chinoise ou japonaise, français)
    : on utilise des chiffres fixes, mais on intercale un symbole
    (écrit) ou un mot (oral) différent pour chaque poids.
  \item Des systèmes de numération exotiques: les poids ne sont pas $1$,
    $B$, $B^2$, $B^3$, etc. mais les valeurs d'une suite (strictement
    croissante): par exemple, numération de Fibonacci.
  \end{itemize}
  
  {\small  Cette page est inspirée de Wikipedia \emph{Système de numération},
    ainsi que les dessins de chiffres babyloniens.}
\end{frame}
\begin{frame}{La base 10}
  \begin{itemize}
  \item Système décimal, utilisé depuis le cinquième siècle en Inde, apporté par
    les Arabes en Europe dans le X\ieme\ siècle.
  \item $\mathcal{B}=\left\{0,1,2,3,4,5,6,7,8,9\right\}$, et $B=10$
  \item Par exemple: mille cinq cent quatre-vingt-quatre se représente par
    $1684_{10}$, qui s'interprète comme
    $$1\times10^3+6\times10^2+8\times10+4$$
  \end{itemize}
  \begin{tikzpicture}[fill=solarizedRebase0,draw=solarizedRebase0,scale=.5]
    \only<1|handout:0>{
      \foreach \xx in {0,1,...,83} {
        \foreach \yy in {0,1,2,3,4,5,6,7,8,9,11,12,...,20} {
          \fill ({.2*\xx},{.2*\yy}) rectangle ++(.1,.1);
        }
      }
    }
    \only<2|handout:0>{
      \foreach \xx in {0,1,...,83} {
        \foreach \yy in {0,11} {
          \fill ({.2*\xx},{.2*\yy}) rectangle ++(.1,1.9);
        }
      }
    }
    \only<3|handout:0>{
      \foreach \xx in {0,10,...,70} {
        \foreach \yy in {0,11} {
          \fill ({.2*\xx},{.2*\yy}) rectangle ++(1.9,1.9);
        }
      }
    }
    \only<1-|handout:1>{
      \foreach \xx in {84} {
        \foreach \yy in {0,1,2,3} {
          \fill ({.2*\xx},{.2*\yy}) rectangle ++(.1,.1);
        }
      }
    }
    \only<3-|handout:1>{
      \foreach \xx in {80,81,82,83} {
        \foreach \yy in {0,11} {
          \fill ({.2*\xx},{.2*\yy}) rectangle ++(.1,1.9);
        }
      }
    }
    \only<4|handout:1>{
      \foreach \xx in {0} {
        \foreach \yy in {0} {
          \fill ({.2*\xx},{.2*\yy}) rectangle ++(9.9,4.1);
        }
      }
      \foreach \xx in {50,60,70} {
        \foreach \yy in {0,11} {
          \fill ({.2*\xx},{.2*\yy}) rectangle ++(1.9,1.9);
        }
      }
    }
  \end{tikzpicture}
\end{frame}
\begin{frame}{Représenter les nombres en informatique}
  \begin{definition}[Les bases les plus utilisées sont 2, 8, 10 et 16]
    \begin{tabular}{|c|>{$}c<{$}|>{\ttfamily}c|>{\ttfamily}c|>{\tiny}r|}
      \hline
      Base &\multicolumn{1}{c}{Chiffres}&& Exemple&Usage\\\hline
      2 &\{0,1\}&0b&0b10110&\only<1->{Codages bas-niveau}\\\hline
      8 &\{0,1,2,3,4,5,6,7\}&0&026&\only<2->{peu utilisé}\\\hline
      16&\{0,1,2,3,4,5,6,7,8,9,&0x&0x16&\only<3->{Écriture compacte}\\
      &A,B,C,D,E,F\}&&&\only<3->{d'octets}\\\hline
      10&\{0,1,2,3,4,5,6,7,8,9\}&&22&\only<4->{Nombres courants}\\\hline
    \end{tabular}
  \end{definition}
  \begin{itemize}
  \item<1-> {$1\times2^4+1\times2^2+1\times2=22_{10}$\pause}
  \item<2-> {$2\times8+6=22_{10}$\pause}%
  \item<3-> {$1\times16+6=22_{10}$\pause}%
  \item<4-> {$2\times10+2=22_{10}$}%
  \item<5-> En binaire, un chiffre est désigné par le terme \emph{bit} (aussi).
  \end{itemize}
\end{frame}
\begin{frame}{De la base x à la base 10}
  On peut toujours convertir un nombre de la façon suivante.
  \begin{methode}[recalcul]
    Si en base $x$, il s'écrit $\alpha\beta\gamma\delta$, il vaut (par
    définition):
    $$\alpha\times x^3+\beta\times x^2+\gamma\times x+\delta$$
  \end{methode}
  \begin{example}[conversion de 0x4D7]
    Le nombre 0x4D7 (hexadécimal) est égal à $4\times16^2+D\times16+7$, donc à 
    $4\times256+13\times16+7=1239$ en base 10.
  \end{example}
  \begin{example}[puissance de la base]
    $B^n$ s'écrit toujours $1$ suivi de $n$ zéros (par exemple, $2^6$
    s'écrit \texttt{0b1\,000\,000})
  \end{example}
\end{frame}
\def\mm#1{\mbox{\texttt{#1}}}
\begin{frame}{De la base 10 à la base 2}
  \begin{columns}[c]
    \begin{column}{0.3\linewidth}
      \begin{methode}[divisions successives]
        On divise le nombre par la base (2). Le reste est le dernier chiffre du
        nombre dans la base 2, on recommence avec le résultat de la division.

        \bigskip

        Ceci fonctionne avec toutes les bases, diviser par $B$ au lieu de 2.
      \end{methode}\pause
    \end{column}
    \begin{column}{0.65\linewidth}
      \begin{example}[divisions successives]
        \uncover<3->{$13/2=6$, reste 1;}
        \uncover<4->{$6/2=3$, reste 0;}
        \uncover<5->{$3/2=1$, reste 1;}
        \uncover<6->{$1/2=0$, reste 1;}
        ~\\
        $13=%
        \uncover<3->{6\times2+\mbox{\texttt{0b1}}}
        \uncover<4->{=3\times2\times2+\mbox{\texttt{0b01}}}
        \uncover<5->{=1\times2\times2\times 2+\mbox{\texttt{0b101}}}
        \uncover<6->{=\mbox{\texttt{0b1101}}}
        $
      \end{example}
      \begin{methode}<7->[soustractions successives, rapide]

        Puissances de 2: \alert<18>{1}, \alert<17>{2}, \alert<16>{4},
        \alert<15>{8}, \alert<14>{16}, \alert<13>{32}, \alert<12>{64},
        \alert<11>{128}, \alert<10>{256}, \alert<9>{512},
        \alert<8>{1024}...

        $216\only<11->{-128=88}\only<12->{-64=24}\only<14->{-16=8}\only<15->{-8=0}$\\
        $216_{10}=\rule[-3.5ex]{0ex}{3.5ex}%
        \only<beamer|beamer:1-10>{216_{10}}
        \only<beamer|beamer:11>{\mm{0b}\underbrace{\mm1}_{128}\mm{0000000}+88_{10}}
        \only<beamer|beamer:12>{\mm{0b1}\underbrace{\mm1}_{64}\mm{000000}+24_{10}}
        \only<beamer|beamer:13>{\mm{0b11}\underbrace{\mm0}_{32}\mm{00000}+24_{10}}
        \only<beamer|beamer:14>{\mm{0b110}\underbrace{\mm1}_{16}\mm{0000}+8_{10}}
        \only<beamer|beamer:15>{\mm{0b1101}\underbrace{\mm1}_{8}\mm{000}}
        \only<beamer|beamer:16>{\mm{0b11011}\underbrace{\mm0}_{4}\mm{00}}
        \only<beamer|beamer:17>{\mm{0b110110}\underbrace{\mm0}_{2}\mm{0}}
        \only<beamer|beamer:18>{\mm{0b1101100}\underbrace{\mm0}_{1}}
        \only<beamer|beamer:15-18>{+0_{10}}%
        \only<19->{\mm{0b11011000}}
        $
      \end{methode}
    \end{column}
  \end{columns}
\end{frame}
\begin{frame}{De la base 2 à 8 et 16 (et inversement)}
  \begin{itemize}
  \item Base 2 vers 8 ou 16 ou inverse: substitution mécanique!
  \item Compléter par des $0$ devant si nécessaire (octal: 3 chiffres, hexadécimal: 4); 
  \item Connaître les correspondances pour chaque chiffre;\par
    \centerline{\small\ttfamily\begin{tabular}{|l|cccccccc|}\hline
        Hex./Octal & 0 & 1 & 2 & 3 & 4 & 5 & 6 & 7 \\\hline
        Binaire & 0 & 1 & 10 & 11 & 100 & 101 & 110 & 111 \\\hline\hline
        Hex. & 8 & 9 & A & B & C & D & E & F\\\hline
        Binaire & 1000 & 1001 & 1010 & 1011 & 1100 & 1101 & 1110 & 1111 \\\hline
      \end{tabular}
    }
  \item Base $8=2^3$: $\mm{0b\only<beamer|beamer:2>{0}\only<beamer|beamer:1-2>{11101}}\only<3->{\underbrace{\mm{011}}_{3}\underbrace{\mm{101}}_5}\only<4->{=\mm{035}}\phantom{\underbrace{\mm{~}}_{D}}$
  \item Base $16=2^4$: $\mm{0b\only<beamer|beamer:2>{000}\only<beamer|beamer:1-2>{11101}}\only<3->{\underbrace{\mm{0001}}_{1}\underbrace{\mm{1101}}_D}\only<4->{=\mm{0x1D}}\phantom{\underbrace{\mm{~}}_{D}}$
  \item De 16 ou 8, vers 2, procédure inverse:
    $\mm{0x\alert<beamer|beamer:5>{3}\alert<beamer|beamer:6>{A}}=\mm{0b\alert<beamer|beamer:5>{0011}\alert<beamer|beamer:6>{1010}}$
  \item \textbf{\textsc{Apprenez ces tables par c\oe ur !}}
  \end{itemize}
\end{frame}

\begin{exercice}
  \begin{exercicelet}{Puissances de 2}
    \begin{questions}
    \item Écrivez la liste de toutes les puissances de 2, de $2^{-4}$ à
      $2^{16}$.
      \begin{xcorrection} 0,06125 -- 0,125 -- 0,25 -- 0,5 -- 1 -- 2 -- 4 -- 8
        -- 16 -- 32 -- 64 -- 128 -- 256 -- 512 -- 1024 -- 2048 -- 4096 -- 8192
        -- 16384 -- 32768 -- 65536

        Et jusqu'à 1024, c'est à savoir par c\oe ur. Aucune discussion à
        avoir.
      \end{xcorrection}
    \item Écrivez une table de conversion des chiffres hexadécimaux et
      octaux vers le codage naturel écrit en binaire (4 bits ou 3 bits).
      \begin{xcorrection} 0 -- 0000, 1 -- 0001, 2 -- 0010, 3 -- 0011, 4 --
        0100, 5 -- 0101, 6 -- 0110, 7 -- 0111, 8 -- 1000, 9 -- 1001, A -- 1010,
        B -- 1011, C -- 1100, D -- 1101, E -- 1110, F -- 1111. Et pour les
        octaux, la même chose de 0 à 7 (on peut mettre seulement sur 3 bits pour
        l'octal, en supprimant le 0 initial).

        C'est à savoir par c\oe ur. Aucune discussion à avoir.
      \end{xcorrection}
    \end{questions}
  \end{exercicelet}
  \begin{exercicelet}{Conversions}
    \begin{questions}
    \item Écrivez en binaire et en hexadécimal les nombres décimaux
      suivants: 28; 149; 1285.
      \begin{xcorrection}
        1 1100; 1001 0101; 101 0000 0101 1001.
        0x1C; 0x95; 0x505.
      \end{xcorrection}
    \item Convertissez en décimal les nombres suivants: 0x48; 0xA1C; 0b1010010010011111.
      \begin{xcorrection}
        72; 2588; 42143.
      \end{xcorrection}
    \item Comment trouver midi à quatorze heures ?
      \begin{xcorrection}En base 8. Ne pas insister sur cette
        question.\end{xcorrection}
    \end{questions}
  \end{exercicelet}
\end{exercice}
\subsection{Des entiers naturels aux réels}
\begin{frame}{Bases, entiers relatifs et réels}
  \begin{itemize}
  \item Pour les entiers relatifs, il faut une information
    supplémentaire: le signe;
  \item Représentation classique: un signe -- pour les négatifs.
  \item Réels: la virgule (séparateur décimal) est à droite du chiffre
    de poids 1 (exposant 0) \emph{(représentation en virgule fixe)}.
  \end{itemize}
  \begin{example}
    $$\begin{array}{rl}
      -10,11_{2}&=-\left(1\times2^1+1\times2^{-1}+1\times2^{-2}\right)\\
      &=-2,75_{10}\\
      -47,2_{8}&=-\left(4\times8^1+7\times8^{0}+2\times8^{-1}\right)\\
      &=-39,25_{10}\end{array}$$
  \end{example}
\end{frame}
\begin{frame}{Convertir un réel d'une base dans une autre}%
  \begin{columns}[c]%
    \begin{column}{0.2\linewidth}
      \begin{theorem}
        \sloppy On peut toujours convertir d'un côté la partie
        fractionnaire d'un nombre, de l'autre sa partie entière.
      \end{theorem}\pause
    \end{column}
    \begin{column}{0.75\linewidth}
      \begin{methode}[multiplications successives]
        On multiplie par $B$ la partie fractionnaire, la partie entière
        du résultat donne le premier chiffre après la virgule.
        \uncover<3->{$0,375\times 2=\alert<3>{0},75$;} \uncover<4->{$0,75\times2=\alert<4>{1},5$;}
        \uncover<5->{$\alert<5>{0},5\times2=\alert<5>{1}$;}\uncover<6->{...développement fini ! Pas toujours !}
        ~\\
        $0,375=%
        \only<3>{\mm{0b0,\alert{0}}}%
        \only<4>{\mm{0b0,0\alert{1}}}%
        \only<5>{\mm{0b0,01\alert{1}}}
        \only<6->{\mm{0b0,011}}
        $
      \end{methode}
      \begin{methode}<7->[soustractions successives, rapide]

        Puissances de 2:
        \alert<8>{0,5},~\alert<9>{0,25},~\alert<10>{0,125},~%
        \alert<11>{0,0625},...

        $0,8125\only<8->{-0,5=0,3125}\only<9->{-0,25=0,0625}\only<11->{-0,0625=0}$\\
        $0,8125_{10}=\rule[-4ex]{0ex}{4ex}%
        \only<beamer|beamer:8>{\mm{0b0,}\underbrace{\mm1}_{1/2}+0,3125_{10}}
        \only<beamer|beamer:9>{\mm{0b0,1}\underbrace{\mm1}_{1/4}+0,0625_{10}}
        \only<beamer|beamer:10>{\mm{0b0,11}\underbrace{\mm0}_{1/8}+0,0625_{10}}
        \only<beamer|beamer:11>{\mm{0b0,110}\underbrace{\mm1}_{1/16}+0_{10}}
        \only<12->{\mm{0b0,1101}}
        $
      \end{methode}
    \end{column}
  \end{columns}
\end{frame}
\begin{exercice}
  \begin{exercicelet}{Changements de base}
    \begin{questions}
    \item Écrivez en binaire et en hexadécimal les nombres décimaux
      suivants: 0,3125; 164,3125.
      \begin{xcorrection}
        0,0101; 1010 0100,0101.
        0x0,5; 0xA4,5.
      \end{xcorrection}
    \item Convertissez en décimal le nombre suivant: 0b1010,0011.
      \begin{xcorrection}
        10,1875.
      \end{xcorrection}
    \end{questions}
  \end{exercicelet}
\end{exercice}

\subsection{Codage des entiers}
\begin{frame}{Le codage}
  \begin{itemize}
  \item Plutôt que d'écrire des nombres, on est souvent amené à les
    \emph{coder}, c'est-à-dire à les écrire sur une taille fixe.
  \item On écrit ces codes en binaire (ou en hexadécimal pour gagner de la
    place) avec un nombre déterminé à l'avance de bits.
  \item Avec un nombre fixé $k$ de bits, on peut écrire uniquement un nombre
    fixé de nombres ($2^k$).
  \end{itemize}
  \begin{definition}[Codage naturel des entiers --- NAT]
    Le codage naturel consiste à écrire l'entier en base $2$ et à
    compléter l'écriture par des $0$ à gauche jusqu'à atteindre la taille
    désirée. Exemple: $27_{10}$=\mm{0b11011} se code $0001\,1011$ en NAT 8 bits.

    Avec $n$ bits, on code les entiers de $0$ à $2^n-1$. Usuellement, on
    utilise des tailles multiples de 8.
  \end{definition}
\end{frame}
\begin{frame}{Le codage des entiers relatifs (1)}
  \framesubtitle{VA+S et C1, peu usités}
  \begin{definition}[Codage valeur absolue+signe --- VA+S]
    On écrit l'entier en base $2$ et on complète l'écriture par des $0$ à
    gauche jusqu'à atteindre la taille désirée \emph{moins $1$}, et de coder le
    signe devant par $0$ (positif) ou $1$ (négatif).

    Exemple: $-12_{10}$=\mm{0b1100} se code $1000\,1100$ en VA+S 8 bits.
  \end{definition}
  \begin{definition}[Codage complément à 1 --- C1]
    L'entier écrit en base $2$ est complété par des $0$
    à gauche jusqu'à la taille désirée moins $1$. \emph{Si le nombre
      est négatif,} on \textbf{complémente} alors chacun des chiffres
    ($0\leftrightarrow1$). Puis signe devant ($+\rightarrow0$).

    Exemple: $-12_{10}$=\mm{0b1100} se code $1111\,0011$ en C1 8 bits.
  \end{definition}
  Avec $n$ bits, on code les entiers de $-2^{n-1}+1$ à $2^{n-1}-1$ (pour VA+S et C1).
\end{frame}
\begin{frame}{Le codage des entiers relatifs (2)}
  \framesubtitle{C2, le plus utilisé}
  \begin{definition}[Codage complément à 2 --- C2]
    Si le nombre est positif, on le complète par des $0$ à gauche jusqu'à
    la taille désirée moins $1$. \emph{Si le nombre est négatif,} on lui
    ajoute $1$, on l'écrit en base $2$ (après soustraction), on le
    complète par des $0$ à gauche jusqu'à la taille désirée moins
    $1$, et on \textbf{complémente} alors chacun des chiffres
    ($0\leftrightarrow1$). Enfin (positif ou négatif), on met le signe
    devant ($+\rightarrow0,-\rightarrow1$).

    Exemple: $-12_{10}$=\mm{0b1100} se code $1111\,0100$ en C2 8 bits.
  \end{definition}

  Deuxième méthode: si négatif, on ajoute $1$ (donc on enlève $1$ à la
  valeur absolue), on complète jusqu'à la taille désirée (et non pas
  taille moins $1$), on complémente (si négatif), et c'est fini.

  Troisième méthode: on complète jusqu'à la taille désirée, si négatif,
  on complémente et on \emph{ajoute} $1$, et c'est fini.
\end{frame}
\begin{frame}{Le codage des entiers relatifs (3)}
  \framesubtitle{C2 (deuxième étape)}
  \begin{itemize}
  \item Avec $n$ bits, on code les entiers de $-2^{n-1}$ à $2^{n-1}-1$.
  \item Dans l'autre sens (de codage C2 vers valeur binaire), il faut faire les
    opérations dans l'ordre inverse.
  \item Pour les positifs, les quatre codages sont identiques.
  \end{itemize}
  \begin{example}[Codage C2 sur 8 bits]
    $28=0b1\,1100$ donc codage $001\,1100\rightarrow0001\,1100$\\
    $-28=0b1\,1100$ donc codage
    $001\,1011\rightarrow110\,0100\rightarrow1110\,0100$\\
    Codage C2: $10100111$, donc valeur négative:
    $0100111\rightarrow1011000\rightarrow1011001$, soit 89 en décimal;
    donc -89.\\
    Codage C2: $00100111$, donc valeur positive: $0b100111$, soit 39 en
    décimal; donc 39.
  \end{example}
\end{frame}
\begin{exercice}
  \begin{exercicelet}{Codage d'entiers}
    \begin{questions}
    \item Ce tableau comporte des cases inutilisées. Complétez-le:
      \begin{center}
        \begin{tabular}{|>{$}c<{$}|>{$}c<{$}|c|>{$}c<{$}|>{\tt}c|}\hline
          \mbox{\textbf{Décimal}}&\mbox{\textbf{Écriture}}&\textbf{Type de}&\mbox{\textbf{Codage}}&\mbox{Codage}\\
          \mbox{}&\mbox{\textbf{Binaire}}&\textbf{codage}&\mbox{(binaire)}&\mbox{(hexa)}\\\hline
          -18&-1\,0010&VA+S (8 bits)& 1001\,0010&0x92\\\hline
          424&\begin{correction}1 1010 1000\end{correction}&NAT (16 bits)&\begin{correction}0000 0001 1010 1000\end{correction}&\begin{correction}0x01A8\end{correction}\\\hline
          -138&\begin{correction}-1000\,1010\end{correction}&C2 (16 bits)&\begin{correction}1111 1111 0111 0110\end{correction}&\begin{correction}0xFF76\end{correction}\\\hline
          \begin{correction}-115\end{correction}&-111\,0011&C1 (8 bits)&\begin{correction}10001100\end{correction}&\begin{correction}0x8C\end{correction}\\\hline
          -4197&\begin{correction}-1 0000\end{correction}&VA+S (24 bits)&\begin{correction}1000 0000 0001\end{correction}&\begin{correction}0x801065\end{correction}\\
          {}&\begin{correction}0110 0101\end{correction}&&\begin{correction}0000 0110 0101\end{correction}&\\\hline
          -84&\begin{correction}-101\,0100\end{correction}&\begin{correction}C1 (8 bits)\end{correction}&\begin{correction}1010\,1011\end{correction}&0xAB\\\hline
          341&\begin{correction}101010101\end{correction}&NAT (8 bits)&\begin{correction}Impossible!\end{correction}&\begin{correction}XXX\end{correction}\\\hline
        \end{tabular}
      \end{center}
      \begin{xcorrection}
        Rappel: pour décoder un complément à 2, il faut: inverser les bits si
        signe=1, ajouter 1 à la valeur obtenue, on a la valeur absolue. Pour
        encoder un complément à 2, il faut soustraire 1 à la valeur à encoder,
        inverser les bits (si le signe est égal à 1). Si le signe est positif,
        rien à faire de différent de NAT.
      \end{xcorrection}
    \end{questions}
  \end{exercicelet}
\end{exercice}
\subsection{Codage des réels}
\begin{frame}{Représentation en virgule flottante}
  \begin{itemize}
  \item Décomposition en quatre parties d'un réel: signe $s$, valeur
    $v$, base $B$ et exposant $e$;
  \item Ex: $-325=-3,25\times10^2$, $-0b101,1=-0b1,011\times2^2$;
  \item Contrainte: valeur=réel $x$, tq $1\leq x<B$;
  \item Un seul chiffre avant la virgule!
  \item $e$ entier, signe usuel;
  \item $x=(-1)^s\times v\times B^e$
  \item Choix de $B$ donne une décomposition unique si $x\neq0$;
  \item En binaire, le premier chiffre de $v$ est forcément $1$;
  \item Exception pour $0$.
  \end{itemize}
\end{frame}
\begin{frame}{Normalisation IEEE 754}
  \begin{itemize}
  \item Réduire la quantité d'info redondante;
  \item format 32 bits pour simple précision, 64 et 80 pour double et
    étendue à partir de codages simples côte-à-côte;
  \item Stockage de $s$, $E=e+127$, $M$ (partie fractionnaire de $v$);
  \item Tout ceci en codage NAT car tout positif!
  \item Format sur 32 bits: \begin{tabular}{|c|c|c|}%
      \multicolumn{1}{c}{1}&\multicolumn{1}{c}{8}&\multicolumn{1}{c}{23}\\\hline%
      $s$&\quad$E$\quad&\qquad$M$\qquad\\\hline%
    \end{tabular};
  \item Exception: pour $0$, $E=00000000$, pour $\infty$, $E=11111111$;
  \item Intervalle de valeurs (32 bits): $2^{-126}$ à
    $(2-2^{-23})\times2^{127}$, soit de $1,8\times 10^{-38}$ à
    $3,4\times10^{38}$.
  \end{itemize}
\end{frame}
\begin{exercice}
  \begin{exercicelet}{Codage IEEE754}
    \begin{questions}
    \item Ce tableau comporte des cases inutilisées. Complétez-le:
      \begin{center}\renewcommand{\tabcolsep}{1mm}
        \begin{tabular}{|c|c|c|c|c|c|>{$}c<{$}|>{\small}c|}\hline
          \multicolumn{1}{|c|}{\textbf{Décimal}}&\multicolumn{1}{c|}{\textbf{Binaire}}&\multicolumn{1}{c|}{\textbf{Virgule flottante}}&\multicolumn{1}{c|}{\textbf{E}}&\multicolumn{3}{c|}{\textbf{Codage IEEE754}}&\textbf{Hexa}\\\cline{5-7}
          &&&&S&E (8b)&\multicolumn{1}{c|}{\null\qquad M (23b)\qquad\null}&\\\hline
          19,5&10011,1&1,00111$\times 2^4$&131&0&10000011&\mbox{00111}\underbrace{\mbox{0\ldots0}}_{\hbox to 0pt {18 fois}}&419C0000\\\hline
          -7,5&\begin{correction}111,1\end{correction}&\begin{correction}1,111$\times 2^2$\end{correction}&\begin{correction}129\end{correction}&\begin{correction}1\end{correction}&\begin{correction}10000001\end{correction}&\begin{correction}\mbox{111}\underbrace{\mbox{0\ldots0}}_{\mbox{20 fois}}\end{correction}&\begin{correction}C0F00000\end{correction}\\\hline
          -46,25&\begin{correction}101110,01\end{correction}&\begin{correction}1,0111001$\times 2^5$\end{correction}&\begin{correction}132\end{correction}&\begin{correction}1\end{correction}&\begin{correction}10000100\end{correction}&\begin{correction}\mbox{0111001}\underbrace{\mbox{0\ldots0}}_{\mbox{16 fois}}\end{correction}&\begin{correction}C2390000\end{correction}\\\hline
          0,3125&\begin{correction}0,0101\end{correction}&\begin{correction}1,01$\times 2^{-2}$\end{correction}&\begin{correction}125\end{correction}&\begin{correction}0\end{correction}&\begin{correction}01111101\end{correction}&\begin{correction}\mbox{01}\underbrace{\mbox{0\ldots0}}_{\mbox{21 fois}}\end{correction}&\begin{correction}3EA00000\end{correction}\\\hline
          \begin{correction}-0,1875\end{correction}&\begin{correction}0,0011\end{correction}&\begin{correction}1,1$\times 2^{-3}$\end{correction}&\begin{correction}124\end{correction}&\begin{correction}1\end{correction}&\begin{correction}01111100\end{correction}&\begin{correction}\mbox{1}\underbrace{\mbox{0\ldots0}}_{\mbox{22 fois}}\end{correction}&BE400000\\\hline
          \begin{correction}$+\infty$\end{correction}&\begin{correction}---\end{correction}&\begin{correction}---\end{correction}&\begin{correction}255\end{correction}&\begin{correction}0\end{correction}&\begin{correction}11111111\end{correction}&\begin{correction}\underbrace{\mbox{0\ldots0}}_{\mbox{23 fois}}\end{correction}&7F800000\\\hline
          0&\begin{correction}0\end{correction}&\begin{correction}---\end{correction}&\begin{correction}0\end{correction}&\begin{correction}0\end{correction}&\begin{correction}00000000\end{correction}&\begin{correction}\underbrace{\mbox{0\ldots0}}_{\mbox{23 fois}}\end{correction}&\begin{correction}00000000\end{correction}\\\hline
          -26,375&\begin{correction}-11010,011\end{correction}&\begin{correction}-1,1010011\end{correction}&\begin{correction}171\end{correction}&\begin{correction}1\end{correction}&\begin{correction}10101011\end{correction}&\begin{correction}1010011\smash{\underbrace{\mbox{0\ldots0}}_{\mbox{16 fois}}}\end{correction}&\begin{correction}D5D30000\end{correction}\\
          $\times2^{40}$&\begin{correction}$\times 2^{40}$\end{correction}&\begin{correction}$\times 2^{44}$\end{correction}&&&&&\\
          \hline
        \end{tabular}
      \end{center}
    \end{questions}
  \end{exercicelet}
\end{exercice}
% Local Variables:
% TeX-master: "archi02"
% TeX-PDF-mode: t
% fill-column: 78
% coding: utf-8-unix
% mode-require-final-newline: t
% mode: latex
% mode: flyspell
% ispell-local-dictionary: "francais"
% End:
