\documentclass[a4paper]{iutvexam}
% compile-latex options:--jobname controle20131008
% compile-latex options:--jobname correction20131008
\usepackage{textcomp}
\title{Contrôle 1}
\date{8/10/2013}
\begin{document}
\conditions{ Vous disposez de 2 heures pour faire ce contrôle. Aucun
  document autorisé. Toute tentative de communication avec un voisin ou
  l'extérieur peut être sanctionnée. Toutes les réponses doivent être
  faites sur l'énoncé.  }
\begin{questions}
  \titledquestion{Codage de coordonnées polaires}\textbf{Codage de coordonnées polaires}\\
  \begin{parts}
    \part[1] Sur combien de bits sera stockée un angle ? Une distance ?
    Une coordonnée ?
    \begin{solutionordottedlines}[.25in]%
      12 bits pour un angle (0.5, 0.25 si 9 bits), 32 bits pour une distance (0.25), 44 bits pour une coordonnée (0.25 pour la somme des deux).
    \end{solutionordottedlines}
    \part[1] Existe t-il des points qui ont plusieurs coordonnées polaires avec la définition
    \begin{solutionordottedlines}[.25in]%
      Oui (point origine, ou angle et angle+180 avec distance=-distance)
    \end{solutionordottedlines}
    \part[2] Expliquez en détail comment coder le point -1/60\textdegree.
    \begin{solutionordottedlines}[1in]%
      1 en IEEE754 = $2^0$, donc $E=127=0111\,1111$, donc \texttt{0x37800000} et \texttt{0b10 0101 1000}
    \end{solutionordottedlines}
    \part[4] Codez les autres points du graphique.
    \begin{solutionordottedlines}[1in]%
      0/0: Que des zéros\\
      0.75/0: \texttt{0x37400000}/0\\
      1.5/310: \texttt{0x37C00000}/1100 0001 1100 (1 0011 0110 si 9 bits)\\
      2/45: \texttt{0x40000000}/1 1100 0010 (101101 si 9 bits)\\
      1.4375/135: \texttt{0x3FB80000}/101 0100 0110 (1000 0111 si 9 bits)
      0.5 points par réponse juste dans la limite de 4 points (j'avais oublié que j'avais mis 5 points).
    \end{solutionordottedlines}
  \end{parts}
\end{questions}
\end{document}
