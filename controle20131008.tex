\documentclass[a4paper]{iutvexam}
% compile-latex options:--jobname controle20131008
% compile-latex options:--jobname correction20131008
\usepackage{textcomp}
\usepackage{dirtree}
\usepackage{listings}
\usepackage{caption}
\newcommand{\DTd}[1]{\textbf{#1/}}
\newcommand{\DTfcomment}[1]{\DTcomment{\emph{#1}}}

\title{Contrôle 1}
\date{8/10/2013}
\begin{document}
\conditions{ Vous disposez de 2 heures pour faire ce contrôle. Aucun
  document autorisé. Toute tentative de communication avec un voisin ou
  l'extérieur peut être sanctionnée. Toutes les réponses doivent être
  faites sur l'énoncé. La taille de la réponse attendue dépend de la
  taille allouée pour répondre. }
\begin{questions}
  \titledquestion{Système}

  Lorsqu'une commande vous est demandée, \textbf{donnez un exemple
    d'utilisation} qui répond à la question.
  \begin{parts}
    \part[\half] Quelle est la commande qui donne un texte d'aide sur
    une commande Linux?
    \begin{solutionordottedlines}[.25in]%
      \texttt{man man}
    \end{solutionordottedlines}
    \part[1] Quelle est l'interprétation du symbole spécial \verb|~|
    dans le shell ? Donnez \textbf{deux} sens (légèrement) différents de
    ce symbole.
    \begin{solutionordottedlines}[.5in]%
      On peut désigner son propre répertoire (symbole seul) ou celui de
      quelqu'un d'autre.
    \end{solutionordottedlines}
    \part[1] Par quelle ligne de code commence un script shell écrit en \texttt{bash} ?
    \begin{solutionordottedlines}[.25in]%
      \verb|#!/bin/bash| (accepter aussi avec sh).
    \end{solutionordottedlines}
    \part[1] Citez au moins deux ressources gérées par le système
    d'exploitation.
    \begin{solutionordottedlines}[.5in]%
      Le système d'exploitation arbitre au moins l'accès au processeur
      et l'accès à la mémoire.
    \end{solutionordottedlines}
    \part[\half] Quelle est la commande qui permet de visualiser une liste
    détaillée des processus existants sur un système Linux ?
    \begin{solutionordottedlines}[.25in]%
      \texttt{ps -ef} (\texttt{ps -eu} est dans le cours et est accepté
      aussi)
    \end{solutionordottedlines}
  \end{parts}
  \titledquestion{Système 2}
  \begin{parts}
    \part[1] Quelle est la différence entre une variable du shell et une variable d'environnement ?
    \begin{solutionordottedlines}[.5in]%
    \end{solutionordottedlines}    
    \part[1] Est-ce que deux processus peuvent être exécutés réellement
    en même temps sur un processeur simple ? Expliquez
    \begin{solutionordottedlines}[1in]%
      
    \end{solutionordottedlines}
    \part[1] Citez au moins quatre systèmes d'exploitation différent
    \begin{solutionordottedlines}[.25in]%
      Linux, Windows, OS X, Android, ...
    \end{solutionordottedlines}
    \part[1] Comment le shell trouve-t-il où est une commande ?
    \begin{solutionordottedlines}[1in]%
      Il utilise la variable \texttt{PATH} qui est une liste de
      répertoires séparés par des deux-points. Il cherche dans ces
      répertoires si un fichier du nom de la commande existe et s'arrête
      au premier trouvé.
    \end{solutionordottedlines}
  \end{parts}

  \newpage
  \titledquestion{Arborescence}

  \framebox{\begin{minipage}{.5\linewidth}%
      \dirtree{%
        .1 \DTd{}.  .2 \DTd{bin}.  .3 (\dots)\DTfcomment{(1)}.  .2
        \DTd{home}.  .3 \DTd{jvaljean}\DTfcomment{Votre répertoire
          personnel}.  .4 \DTd{Documents}\DTfcomment{Vous êtes ici}.  .5
        \DTd{Ancien}. .6 \DTd{Paris}.  .7 greve.jpg.  .5
        condamnation.pdf. .5 document1.txt.  .5 doc2.txt.  .5
        facture.txt.  .5 mission.pdf.  .5 mission.jpg.  .4 \DTd{Coffre}.
        .3 \DTd{gavroche}.  .4 avis.txt.  .2 (\dots)\DTfcomment{(2)}.  }
    \end{minipage}}\hfill%
  \begin{minipage}{.45\linewidth}
    Voilà ci-contre un extrait de l'arborescence de votre système. Il y
    a d'autres utilisateurs : \textit{\texttt{gavroche}} est dans votre
    groupe, \textit{\texttt{thenardier}} et \textit{\texttt{cosette}} ne
    le sont pas.

    Vous devrez faire tout cet exercice sans \textbf{jamais} utiliser la
    commande \texttt{cd}. Dans l'extrait d'arborescence, le répertoire
    courant est indiqué. Il est possible que \textbf{deux} commandes
    soient nécessaires pour certaines manipulations.
  \end{minipage}
  \begin{parts}
    \bonuspart[\half] Répondez à \textnormal{toutes} les questions avec
    une seule commande pour ce bonus!
    \part[\half] Listez le contenu du répertoire \texttt{Ancien}
    \begin{solutionordottedlines}[.25in]%
      \texttt{ls Ancien}
    \end{solutionordottedlines}
    \part[\half] Créez un répertoire \texttt{Nouveau/Vacances}
    \begin{solutionordottedlines}[.25in]%
      \texttt{mkdir -p Nouveau/Vacances} ou en deux commandes (sinon 0)
    \end{solutionordottedlines}
    \part[1] Lisez le contenu du fichier \texttt{avis.txt} chez
    l'utilisateur \textit{\texttt{gavroche}}.
    \begin{solutionordottedlines}[.25in]%
      \texttt{cat \~gavroche/avis.txt} ou autre solution à base de \texttt{..}
    \end{solutionordottedlines}
    \part[1] Déplacez tous les documents du répertoire courant dans \texttt{Coffre} sauf \texttt{condamnation.pdf}
    \begin{solutionordottedlines}[.25in]%
      \texttt{mv *.txt mission.pdf} ou autre solution à base de \texttt{..}
    \end{solutionordottedlines}
    \part[1] Copiez \texttt{condamnation.pdf} et le répertoire \texttt{Paris} dans le répertoire \texttt{Nouveau/Vacances}
    \begin{solutionordottedlines}[.25in]%
      \texttt{cp -r condamnation.txt Ancien/Paris Nouveau/2013} (.5 pour le -r).
    \end{solutionordottedlines}
    \part[1] Sachant que les fichiers étaient lisibles par tout le
    monde, changez les modes de \texttt{Coffre} pour que
    \texttt{\textit{gavroche}} puisse connaître le contenu du
    répertoire, que \texttt{\textit{thenardier}} ne puisse pas, et que
    \texttt{\textit{cosette}} puisse (si vous lui en donnez le chemin)
    lire un des fichiers. Utilisez les notations \textbf{numériques}
    pour cette commande.
    \begin{solutionordottedlines}[.25in]%
      \texttt{chmod 751 \~/Coffre}
    \end{solutionordottedlines}
    \part[\half] Le mode de \texttt{Ancien} est au départ
    \texttt{rwxr-xr-x}. En utilisant les notations \textbf{symboliques}
    et non numériques pour cette commande, faites en sorte que plus
    personne (même vous) ne puisse savoir quoi que ce soit sur le
    répertoire \texttt{Paris} stocké dans \texttt{Ancien}.
    \begin{solutionordottedlines}[.25in]%
      \texttt{chmod a-rwx Ancien/Paris}
    \end{solutionordottedlines}
    \part[\half] Donnez ici deux exemples de répertoires ou fichiers que l'on trouve traditionnellement dans les parties non listées numéro 1 et 2.
    \begin{solutionordottedlines}[.25in]%
      1. ls et 2. usr
    \end{solutionordottedlines}
  \end{parts}
  \newpage
  \titledquestion{Codage}
  \begin{parts}
    \part[\half] Quel est le nombre \textbf{minimal} de bits nécessaire
    pour coder l'information suivante: un numéro de page de l'annexe
    d'un livre, qui vaut entre 1400 et 1912.
    \begin{solutionordottedlines}[.25in]
    \end{solutionordottedlines}
    \part[\half] Quel est le plus grand entre $48\times 10^6$
    bits et $6$ Mio ? Prouvez-le
    \begin{solutionordottedlines}[.5in]
      ...TODO
    \end{solutionordottedlines}
    \part[1\half] Convertissez en hexadécimal les 3 nombres suivants:
    48, 226, 2251.
    \begin{solutionordottedlines}[.75in]
      ...TODO
    \end{solutionordottedlines}
    \part[1] Convertissez en C2 sur 8 bits le nombre $-61$.
    \begin{solutionordottedlines}[.25in]
      ...TODO
    \end{solutionordottedlines}
    \part[1] Faites l'opération suivante:
    $0b1101\,1011+0b1111\,0101+1$. Donnez le résultat en binaire puis en
    décimal.
    \begin{solutionordottedlines}[.25in]
      ...TODO
    \end{solutionordottedlines}
    \part[1] Voici le récapitulatif du tableau de conversion UTF8:
    \begin{center}\renewcommand{\tabcolsep}{1mm}
      \begin{tabular}{|c|c|c|c|}\hline
        Valeurs & Écriture binaire & Codage UTF-8 (binaire) & octets\\\hline
        0x0--0x7F & abc\,defg & 0abc\,defg & 1 \\
        0x80--0x7FF & abc\,defg\,hijk & 110a\,bcde~10fg\,hijk & 2 \\
        0x800--0xFFFF & abcd\,efgh\,ijkl\,mnop & 1110\,abcd~10ef\,ghij~10kl\,mnop & 3 \\
        0x10000--0x1FFFFF& a\,bcde\,fghi\,jklm\,nopq\,rstu & 1111\,0abc~10de\,fghi~10jk\,lmno~10pq\,rstu &	4 \\\hline
      \end{tabular}
    \end{center}
    Donnez la séquence d'octets correspondant au texte «~1€~», dont les
    valeurs unicodes sont 0x49 et 0x20AC.
    \part[\half] Un signal audio a une fréquence d'échantillonage de
    8000 Hz. Quelle est la fréquence maximale qui peut être reproduite
    fidèlement avant quantification ? Justifiez
    \begin{solutionordottedlines}[.25in]
      ...TODO
    \end{solutionordottedlines}
    \part[1] Un signal audio (mono-voie) à 8000 Hz utilise 1024 niveaux
    d'intensité par échantillon. Combien de bits sont nécessaires pour
    chaque échantillon ? (justifiez) En déduire la taille totale d'un
    fichier qui code une minute de signal.
    \begin{solutionordottedlines}[.25in]
      $8000\times 10\times 60=4\,800\,000$ bits ou bien 600 ko.
    \end{solutionordottedlines}
  \end{parts}
  \titledquestion{Flottants courts} On veut modifier la norme IEEE
  754. On définit une nouvelle catégorie de nombres flottants sur 12
  bits:
  \begin{itemize}
  \item 1 bit de signe;
  \item 4 bits pour noter $E$ avec $E=e+7$;
  \item 7 bits pour la partie fractionnaire $M$ de la valeur $v$.
  \end{itemize}
  On a $x=(-1)^s\times v\times 2^e$. On fait les mêmes
  \textbf{exceptions} que dans la norme IEEE 754: pour
  $E=\texttt{0b0000}$, on a $x=0$ et pour $E=\texttt{0b1111}$, on a
  $x=\pm\infty$. NB: $2^{-6}=0,015\,625, 2^{-12}=0,000\,244\,140\,625$.
  \begin{parts}
    \part[1] Codez en \emph{flottant court} le nombre 3,5.
    \begin{solutionordottedlines}[.5in]%
      ...TODO
    \end{solutionordottedlines}
    \part[1] Quel est le plus grand nombre non infini que l'on peut
    représenter? Donnez son \emph{codage} sous forme hexadécimale, et sa
    valeur en décimal.
    \begin{solutionordottedlines}[.5in]%
      Codage $011101111111=0x77F$ (1/2), $E=0b1110=14$, $e=7$ et
      $v=0b1,1111111$. Soit $x=0b1111\,1111=255$ (1/2). Notes partielles
      si essentiellement bon.
    \end{solutionordottedlines}
    \part[1] Soit deux nombres $a$ et $b$, flottants courts, qui ont le
    même exposant $e$. Quelle est la plus petite différence possible non
    nulle entre ces deux nombres ?
    \begin{solutionordottedlines}[.5in]%
      La différence de $v$ vaut 0.0000001 donc différence totale
      de $2^e\times2^{-7}$ donc $2^{e-7}$.
    \end{solutionordottedlines}
    \part[1] Transformez en décimal les deux nombres \emph{flottants
      courts} codés par les chaînes de bits suivantes:
    $\texttt{0b0011\,1010\,0000}$ et
    $\texttt{0b0110\,0101\,1010}$.
    \begin{solutionordottedlines}[.5in]%
      ...TODO
    \end{solutionordottedlines}
  \end{parts}

  \titledquestion{Programme étrange}%
  Considérez le programme suivant:

  \begin{lstlisting}[language=C,caption=Programme]
fonction(int a) {
  int b=0;
  while (a!=0) {
    a=a>>1;
    b=b+(a&1)
  }
  return(b);
}
  \end{lstlisting}

  \begin{parts}
    \part[1] Pourquoi ce programme s'arrête-t-il ? Expliquez-bien.
    \begin{solutionordottedlines}[.75in]
      TODO
    \end{solutionordottedlines}
    \part[1] On appelle la fonction avec la valeur 21. Détaillez les
    valeurs successives de \texttt{a} et \texttt{b} jusqu'à l'arrêt de
    la fonction.

    \begin{minipage}{.3\linewidth}
      \begin{solutionordottedlines}[1in]
        TODO
      \end{solutionordottedlines}
    \end{minipage}\hfill
    \begin{minipage}{.3\linewidth}
      \begin{solutionordottedlines}[1in]
        TODO
      \end{solutionordottedlines}
    \end{minipage}\hfill
    \begin{minipage}{.3\linewidth}
      \begin{solutionordottedlines}[1in]
        TODO
      \end{solutionordottedlines}
    \end{minipage}
    \part[1] Que calcule cette fonction ?
    \begin{solutionordottedlines}[.25in]
      TODO
    \end{solutionordottedlines}
  \end{parts}
\end{questions}
\end{document}
